\chapter{Suite de fonctions ou d'application - Convergence uniforme}
\section{Convergence simple}
\begin{de}
Soit A une ensemble (en général, A est un intervalle), et $(E,\parallel~\parallel)$ un K espace vectoriel normé.\\
Soit $(f_n)$ une suite d'application de $A\rightarrow E$.\\
On dit que cette suite converge simplement sur A si : 
$$\forall x \in A~ (f_n(x)) \mbox{ converge dans } (E,\parallel~\parallel)$$
Lorsque que c'est le cas, la limite de $f_n(x)$ quand n tend vers $\infty$ dépend a priori de x.\\
On peut donc définir une nouvelle application : 
$$f : A \rightarrow E$$
$$x \mapsto f(x) = \lim_{n\mapsto\infty}f_n(x)$$
On dit que f est la limite simple de $f_n$ sur A.
\end{de}
\section{Convergence uniforme d'une suite d'applications}
\begin{prop}
L'ensemble B(A,E) des applications bornées d'un ensemble A dans un K espace vectoriel normes (E,$\parallel~\parallel$) est un K sous espace vectoriel des applications de A dans E. De plus : 
$$B(A,E) \rightarrow \Re^+$$
$$f \mapsto \parallel f\parallel_{\infty,A}=\underset{A}Sup\parallel f\parallel$$
est une norme 
\end{prop}
\begin{de}
Soit $(f_n)$ une suite d'application de A, qui est un ensemble de E, un K espace vectoriel normé $(E,\parallel~\parallel)$, et f une application de A dans E.\\
On dit que $(f_n)$ converge uniformement vers f sur A si :
$$\exists n_0 \in N~ tq~ \forall n\geq n_0~ f_n-f \in B(A,E) \mbox{ (bornée) }$$
$$\parallel f_n-f\parallel_{\infty,A}\underset{n\mapsto\infty}\rightarrow0$$
En général, la première condition est évidente. Il faut donc se concentrer sur la deuxième propriété. En pratique, E=$\Re$ et A = I c $\Re$.
\end{de}
\begin{prop}
Soit $(f_n)$ une suite d'application de A, un ensemble, dans (E,$\parallel~\parallel$), un K espace vectoriel normée.\\
Si $(f_n)$ converge uniformement vers f sur A:
$$f : A \rightarrow E$$
Alors $(f_n)$ converge simplement vers f sur A.
\end{prop}
\begin{prop}
Cette propriété est utile pour prouver la non convergence uniforme.\\
Soit $(f_n)$ une suite d'application de A, un ensemble, dans (E,$\parallel~\parallel$) un K espace vectoriel normée.\\
Si $(f_n)$ convergent uniformement vers f, application de A dans E, alors :
$$\forall (x_n) \in A^N~ f_n(x_n) - f(x_n) \underset{n \rightarrow \infty}\rightarrow \overrightarrow{0}$$
De plus, $(x_n)$ n'est pas necessairement convergente et cette notion n'a meme pas de sens si il n'y a pas de distance de défini sur A.
\end{prop}
\section{Théorème classique sous les hypothèses de convergence uniforme}
\subsection{Théorème de continuité}
\begin{theo}
Soit $(f_n)$ une suite d'application de $I\mapsto K$, avec I un intervalle de $\Re$, convergent uniformement vers f, application de I dans K, sur tout le segment $[a,b]$ inclu dans I.\\
Si $\forall n \in N$, $f_n$ est continue en $x_0 \in I$, alors f est continue en $x_0$.\\
Si $\forall n \in N$, $f_n$ est continue sur I, alors f est continue sur I.
\end{theo}
\begin{gene}
Soit $(f_n)$ une suite d'application de A dans E, avec A une partie non vide d'un espace vectoriel normée, E un K espace vectoriel.\\
Si : \\
\begin{itemize}
 \item[$\rightarrow$] $\forall n \in N$ $f_n$ est continue sur A, respectivement en $a \in A$.\\
 \item[$\rightarrow$] $(f_n)$ converge uniformement vers $f : A \rightarrow E$ sur tout compact $\in A$\\
\end{itemize}
Alors, f est continue sur A, respectivement en a.
\end{gene}
\begin{de}
Un compact est une extension du théorème de Bolzano-Weistrass, qui dit que de toute suite convergente on peut extraire une suite croissante.
\end{de}
\subsection{Théorème d'interversion des limites, ou théorème de la double limite}
\begin{theo}
Soit $(f_n)$ une suite d'application de I dans K, avec I un intervalle non vide de $\Re$, convergent uniformement sur I, vers un application f de I dans K.\\
Soit $a \in \bar{\Re}$ un point de I ou une extrémité de I.\\
Si $\forall n \in N$ :
$$f_n(x) \underset{x \rightarrow a}\rightarrow l_n \in K$$
avec $x \in a$, alors : \\
\begin{itemize}
 \item[$\rightarrow$] $(l_n)$ converge vers une limite $l \in K$.\\
 \item[$\rightarrow$] $f(x) \underset{x \rightarrow a}\rightarrow l$, avec $x \in a$.\\
\end{itemize}
\end{theo}
\begin{de}
On dit qu'un espace vectoriel normée qu'il est complet si toute suite de cette espace vérifiant le critère de cauchy converge.\\
Les espaces vectoriel normée complet sont appellé espace de Banach
\end{de}
\begin{gene}
Soit $(f_n)$ une suite d'application de A dans E, avec A une partie non vide d'un espace vectoriel normée, et E un K espace vectoriel normée complet.\\
Si $(f_n)$ converge uniformement sur A vers une application f de A dans E, si $a \in \bar{A}$ c'est à dire si tout voisinage de a rencontre A, et si $\forall n \in N~ f_n(x) \underset{x \rightarrow a}\rightarrow l_n \in E$, avec $x \in A$, alors : \\
\begin{itemize}
 \item[$\rightarrow$] ($l_n$) converge vers $l \in E$.\\
 \item[$\rightarrow$] De plus $f(x) \underset{x \rightarrow a}\rightarrow l$, avec $x \in A$. On peut écrire ceci sous la forme suivante :  
$$\lim_{x \rightarrow a,~ x \in A} \lim_{n\rightarrow \infty} f_n(x) = \lim_{n\rightarrow \infty} \lim_{x \rightarrow a,~ x \in A} f_n(x)$$
\end{itemize}
On peut inverser les limites dans ce cas.
\end{gene}
\subsection{Théorème d'integration sur un segment sous les hypothèses de convergence uniforme}
\begin{theo}
Soit $(f_n)$ une suite d'application continue sur un segment [a,b], à valeur dans K, convergent uniformement sur [a,b] vers f, application de [a,b] dans K.\\
Alors : \\
\begin{itemize}
 \item[$\rightarrow$] f est continue sur [a,b]\\
 \item[$\rightarrow$] $\int_a^b f_n = \int_a^b f$ quand $n\rightarrow\infty$ \\
\end{itemize}
On peut écrire la deuxième conclusion sous la forme : 
$$\lim_{n\rightarrow\infty} \int_a^b f_n = \int_a^b \lim_{n\rightarrow\infty} f_n$$
\end{theo}
\begin{gene}
Ce théorème reste valable lorsqu'on remplace l'ensemble d'arrivé par un K espace vectoriel normée complet E. Ceci suppose d'avoir, au préalable défini $\int_a^b g$ pour g fonction de [a,b] dans E, un espace de banach, au moins continue par morceaux.
\end{gene}
\subsection{Thérorème de classe $C^1$}
\begin{theo}
Soit $(f_n)$ une suite d'application de I, un intervalle de $\Re$, dans K. On suppose que : \\
\begin{itemize}
 \item[$\rightarrow$] $\forall n \in N$ $f_n$ est $C^1$ sur I\\
 \item[$\rightarrow$] La suite $(f_n)$ converge simplement sur I vers une application f de I dans K\\
 \item[$\rightarrow$] La suite $(f'_n)$ converge uniformement sur tout segment [a,b] inclu I vers une application g de I dans \\
\end{itemize}
Alors : \\
\begin{itemize}
 \item[$\rightarrow$] f est de classe $C^1$ sur I\\
 \item[$\rightarrow$] f'=g\\
 \item[$\rightarrow$] $f_n$ converge uniformement vers f sur tout segment inclu dans I.\\
\end{itemize}
La conclusion 2 peut aussi s'écrire sous la forme : 
$$\dfrac{d}{dx} \lim_{n\mapsto\infty} f_n = \lim_{n\mapsto\infty} \dfrac{d}{dx} f_n$$
\end{theo}
\begin{gene}
Le théorème précédent reste vrai quand on remplace l'ensemble d'arrivé par un espace vectoriel normée complet, un espace de Barach  
\end{gene}
\subsection{Théorème de classe $C^p$}
\begin{theo}
Soit $(f_n)$ une suite d'application de I, un intervalle de $\Re$, dans K. Soit $p \in N^*$.\\
On suppose que : \\
\begin{itemize}
 \item[$\rightarrow$] $\forall n \in N$ $f_n$ est $C^p$ sur I
 \item[$\rightarrow$] Les suites $(f_n),(f'_n),...,(f_n^{(p-1)}$ converge simplement sur I vers une application f de I dans K
 \item[$\rightarrow$] La suite $(f_n^{(p)})$ converge uniformement sur tout segment [a,b] inclu I vers une application g de I dans K
\end{itemize}
Alors : \\
\begin{itemize}
 \item[$\rightarrow$] La limite simple f de la suite $(f_n)$ est $C^p$ sur I.
 \item[$\rightarrow$] $\forall k \in \textlbrackdbl\ 1,p \textrbrackdbl$ $f^{(k)}$ est la limite simple de la suite $(f_n^{(k)})$
 \item[$\rightarrow$] $\forall k \in \left[0,p\right]$ $(f_n^{(k)})$ converge uniformement vers $f^{(k)}$ sur tout segment de I.
\end{itemize}
\end{theo}
\begin{gene}
Ce théorème reste valable quand on remplace l'ensemble d'arrivé par un espace vectoriel complet.
\end{gene}
\subsection{Théorème de classe $C^{\infty}$}
\begin{de}
Une application de I, un intervalle inclu dans $\Re$, dans K est dite de classe $C^{\infty}$ sur I si $\forall p \in N^*$ elle est $C^p$ sur I.
\end{de}
On en déduit facilement du théorème $C^p$ précédent que si $(f_n)$ est une suite d'application de I dans K, et si : \\
\begin{itemize}
 \item[$\rightarrow$] $\forall n \in N$ $f_n$ est de classe $C^{\infty}$ sur I\\
 \item[$\rightarrow$] $\forall k \in N$ $(f_n^{(k)})$ converge uniformement sur tout segment inclu dans I\\
\end{itemize}
Alors :\\
\begin{itemize}
 \item[$\rightarrow$] La limite simple f de la suite $(f_n)$ est $C^{\infty}$ sur I.
 \item[$\rightarrow$] $\forall k \in N$ $(f_n^{(k)})$ converge uniformement vers $f^{(k)}$ sur tout segment inclu dans I.
\end{itemize}
Pour définir le classe $C^p$ dans le cas d'un espace E, on utilise le taux de variation
\section{Théorème de convergence monotone et convergence dominé}
\subsection{Théorème de convergence monotone}
\begin{de}
La suite $(f_n)$ d'application de I dans $\Re$ est dite monotone si $\forall x \in I$, $(f_n(x))$ est monotone
\end{de}
\begin{theo}
Soit I un intervalle quelconque de $\Re$ (pas forcément un segment), et $(f_n)$ une suite d'application continue par morceaux sur I, à valeur dans $\Re$, convergent simplement sur I vers f, application de I dans $\Re$, également continue par morceaux. Si : \\
\begin{itemize}
 \item[$\rightarrow$] La suite $(f_n)$ est monotone\\
 \item[$\rightarrow$] Si $f_0$ et f sont intégrable sur I\\
\end{itemize}
Alors : \\
\begin{itemize}
 \item[$\rightarrow$] $\forall n \in N$ $f_n$ est intégrable sur I
 \item[$\rightarrow$] $\int_I f_n \rightarrow \int_I f$ quand $n\rightarrow\infty$
\end{itemize}
La deuxieme conclusion peut s'écrire sous la forme : 
$$\lim_{n\rightarrow\infty} \int_I f_n = \int_I \lim_{n\rightarrow\infty} f_n$$
\end{theo}
\subsection{Théorème de la convergence dominée}
\begin{theo}
Soit I un intervalle quelconque inclu dans $\Re$, et $(f_n)$ une suite d'application de I dans K, continue par morceaux, convergent simplement vers f, application de I dans K, également continue par morceaux.\\
Si $\exists$ g, application de I dans $\Re^+$, continue par morceaux et intégrale sur I, telque (condition de domination): 
$$\forall n \in N,~ \forall x \in I,~ |f_n(x)|\leq g(x)$$
Alors :\\ 
\begin{itemize}
 \item[$\rightarrow$] Les $f_n$ et f sont intégrable sur I
 \item[$\rightarrow$] $\int_I f_n = \int_I f$ quand $n \rightarrow \infty$.
\end{itemize}
On peut écrire cette dernière conséquence sous la forme suivante : 
$$\lim_{n \rightarrow \infty} \int_I f_n = \int_I \lim_{n \rightarrow\infty} f_n$$
\end{theo}