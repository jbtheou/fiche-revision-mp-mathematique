\chapter{Série numérique}
\section{Définitions}
\subsection{Définitions générales}
\begin{de}
Soit ($u_n$) une suite à valeur dans K.\\
On appelle série de terme général $u_n$, et on note $\underset{n}\Sigma u_n$ cette nouvelle suite, de terme générale :
$$S_n = \sum_{k=0}^n u_k = u_0 +u_1 + ... +u_n$$
$S_n$ est appelé somme partiel de rang n de la série $\underset{n}\Sigma u_n$
\end{de}
\begin{de}
On dit que la série $\underset{n}\Sigma u_n$ converge si la suite ($S_n$) des sommes partiel converge dans K.\\
On note alors :
$$\sum_{k=0}^{\infty} u_k$$
cette limite.\\
En général, on note également S cette limite, et on appelle S somme de la série.
\end{de}
\begin{prop}
Si la série $\underset{n}\Sigma u_n$ converge, alors $u_n \rightarrow 0$ quand $ n \rightarrow +\infty$.
\end{prop}
\subsection{Reste d'ordre n}
\begin{de}
Si $\underset{n}\Sigma u_n$ converge, on peut définir $R_n$, le reste d'ordre n de la série $\underset{n}\Sigma u_n$, par :
$$R_n = \lim_{p \rightarrow \infty} \sum_{k=n+1}^p u_k$$
\end{de}
\begin{prop}
On obtient les relations suivantes : 
$$S_n + R_n = S$$
$$\lim_{n \rightarrow \infty} R_n \rightarrow 0$$
\end{prop}
\section{Quelques propriétés générales}
Dans tout ce chapitre, ($u_n$),($v_n$),... sont des suites à valeurs dans K
\begin{prop}
Supposons que $\underset{n}\Sigma u_n$ et $\underset{n}\Sigma v_n$ converge, alors la série de terme général $w_n = u_n + v_n$ converge, et on as :
$$\sum_{k = 0 }^{\infty} w_k = \left(\sum_{k = 0 }^{\infty} u_k\right) + \left(\sum_{k = 0 }^{\infty} v_k\right)$$
\end{prop}
\begin{prop}
Soit $\lambda \in K$.\\
Si $\underset{n}\Sigma u_n$ converge, il en est de même de la série de terme général $w_n = \lambda u_n$, et alors : 
$$\sum_{k = 0 }^{\infty} w_k = \lambda\left(\sum_{k = 0 }^{\infty} u_k\right)$$
\end{prop}
\begin{prop}
Soit $(z_n)$ est une suite à valeur dans C.\\
Si $(x_n) = Re(z_n)$ et $(y_n)= Im(z)$, donc $z_n = x_n + iy_n$, alors : 
$$(\underset{n}\sum z_n \mbox{ converge }) \Leftrightarrow (\sum_n x_n~ et~ \sum_n y_n \mbox{ converge })$$
\end{prop}
\begin{de}
On dit que la série de terme général $u_n$ converge absolument, ou que la suite ($u_n$) est sommable, si la série $\underset{n}\Sigma |u_n|$ converge.
\end{de}
\begin{theo}
L'absolu convergence de la série de terme général $u_n$ implique la converge de la série de terme général $u_n$.\\
Dans ce cas, on as : 
$$|\sum_{k=0}^{\infty} u_k| \leq \sum_{k=0}^{\infty} |u_k|$$
\end{theo}
\begin{prop}
Pour tous $n_0$ entier naturel, on peut modifier les $n_0$ premier termes de la suite ($u_n$) sans modifier la convergence de la série de terme général $u_n$.\\
On peut écrire ceci sous la forme :\\
Si $(v_n)$ vérifie à partir du rang $n_0$ $v_n = u_n$, alors :
$$(\underset{n}\Sigma v_n \mbox{ converge }) \Rightarrow (\underset{n}\Sigma u_n \mbox{ converge })$$
\end{prop}
\begin{prop}
Toute suite $(a_n)$ est une somme partielle d'une série $\underset{n} u_n$, à un terme constant près.
$$\forall n \in N ~a_n = \sum_{k=1}^n (a_k -a_{k-1}) + a_0$$
\end{prop}
\section{Séries à termes réels positifs (ou de signe constant)}
\begin{propfond}
Soit $(u_n)$ une suite à valeur réel positive.\\
Alors la suite des sommes partielle $S_n$ est croissante. Donc : 
\begin{itemize}
 \item[$\rightarrow$] Soit $(S_n)$ est majorée, et alors elle converge vers S = $Sup S_n$
 \item[$\rightarrow$] Soit $(S_n)$ n'est pas majorée, alors $S_n \underset{\infty}\rightarrow \infty$. On ecrit :
$$\sum_{k=0}^{\infty} u_k = +\infty$$
\end{itemize}
\end{propfond}
\begin{prop}
Si $(u_n) $ et $(v_n)$ sont à terme réels positifs telque 
$$0 \leq u_n \leq v_n$$
Alors : 
\begin{itemize}
 \item[$\rightarrow$] $\underset{n}\Sigma v_n$ converge $\Rightarrow$ $\underset{n}\Sigma u_n$ converge et :
$$0 \leq \Sigma_{k=0}^{\infty} u_k  \leq \Sigma_{k=0}^{\infty} v_k$$
 \item[$\rightarrow$] $\underset{n}\Sigma u_n$ diverge $\Rightarrow$ $\underset{n}\Sigma v_n$ diverge
\end{itemize}
\end{prop}
\begin{prop}
Soient $(u_n)$ et $(v_n)$ deux suites à termes réel positifs.\\
Si $u_n = \underset{\infty}= O(v_n)$ (Grand O), et si $\underset{n}\sum v_n$ converge, alors $\underset{n}\sum u_n$ converge
\end{prop}
\begin{prop}
Soient $(u_n)$ et $(v_n)$ deux suites à termes réels positifs, ou simplement de signe constant.\\
Si $u_n \underset{\infty}\sim v_n$, alors:
$$(\underset{n}\sum u_n \mbox{ converge })\Leftrightarrow(\underset{n}\sum v_n \mbox{ converge }) $$
On dit que $\underset{n}\sum u_n$ et $\underset{n}\sum u_n$ ont même nature.
\end{prop}
\subsection{Convergence des séries de Riemann}
\begin{de}
On appelle série de Riemann les séries de terme général, avec $\alpha \in \Re$ :
$$u_n = \dfrac{1}{n^{\alpha}}$$
\end{de}
\begin{prop}
Soit ($u_n$) la suite de terme général :
$$u_n = \dfrac{1}{n^{\alpha}}$$
Alors :
$$(\underset{n}\sum u_n \mbox{converge}) \Leftrightarrow (\alpha > 1)$$
\end{prop}
\begin{prop}
La comparaison série-intégrale (Qui consiste à encadrer la somme partiel, ou le reste partiel, par des intégrales de la fonction) précédent permet d'obtenir un équivalent simple, quand le cas de série de terme général $u_n = \dfrac{1}{n^{\alpha}}$ :
\begin{itemize}
 \item{$\rightarrow$} De $S_n$ dans le cas divergent
 \item{$\rightarrow$} De $R_n$, dans le cas convergent.
\end{itemize}
\end{prop}

\subsection{Règle de Riemann}
\begin{prop}
Soit $(u_n)$ suite à valeur réelle ou complexe.\\
Si $\exists \alpha > 1$ telque $n^{\alpha}u_n \underset{\infty}\rightarrow 0$, alors $\underset{n}\sum |u_n|$ converge. On dit aussi que la suite $(u_n)$ est sommable
\end{prop}
\begin{prop}
Soit $(u_n)$ suite à valeur réelle.\\
Si $nu_n \underset{\infty}\rightarrow \infty$, alors $\underset{n}\sum u_n$ diverge. On a meme : 
$$\sum_{k=0}^{\infty} u_n = \infty$$
\end{prop}
\subsection{Série de Bertrand}
\begin{de}
Ce sont les séries de terme général :
$$u_n = \dfrac{1}{n^{\alpha}.ln(n)^{\beta}}$$
\end{de}
\begin{prop}
Une série de Bertrand converge si et seulement si : 
$$\alpha > 1~ ou~ \alpha=1~ et~ \beta>1$$
\end{prop}
\subsection{Propriété de Cauchy}
Soit u, la fonction définie $\forall n_0 \in N$, par : 
$$u : \left[n_0,\infty \right[ \rightarrow \Re$$
$$t \rightarrow u(t)$$
u est une application continue par morceaux et monotone. On obtient que :
$$(\sum_n u(n) \mbox{ converge }) \Leftrightarrow (\int_{n_0}^{\infty} u \mbox{ converge })$$
On dit que la série et l'intégrale ont même nature.
\subsection{Règle de d'Alembert}
\begin{prop}
Soit $(u_n)$ une suite de réels telque $\forall n \Leftrightarrow \geq n_0$, $u_n > 0$, et telque :
$$\dfrac{u_{n+1}}{u_n}\underset{\infty}\rightarrow l\in \bar{\Re^+}$$
Alors : 
\begin{itemize}
 \item{$\rightarrow$} Si l > 1, alors la série de terme général $u_n$ diverge grossièrement
 \item{$\rightarrow$} Si l < 1, alors la série converge
\end{itemize}
\end{prop}
\section{Séries alternées}
\begin{de}
Soit $(u_n)$ une suite de réels.\\
On dit que $(u_n)$ est alternée si elle est du type :
$$\forall n \in N,~ u_n = (-1)^n.a_n$$
Ou du type : 
$$\forall n \in N,~ u_n = (-1)^{n+1}.a_n$$
Avec $(a_n)$, suite de réels positive. Dan ces deux cas : 
$$a_n = |u_n|$$
La série $\underset{n}\sum u_n$ est dites alternée sur la suite $(u_n)$ est alternée.
\end{de}
\subsection{Critère de Leibniz ou Critère spécial des séries alternée}
\begin{enon}
Soit $\underset{n}\sum u_n$ une série alternée.\\
Si : 
\begin{itemize}
 \item{$\rightarrow$} La suite $(u_n) \underset{\infty}\rightarrow 0$
 \item{$\rightarrow$} (|$u_n$|) est une suite décroissante
\end{itemize}
Alors : 
\begin{itemize}
  \item{$\rightarrow$} La série $\underset{n} \sum u_n$ converge
 \item{$\rightarrow$} $\forall n \in N$, le signe de $R_n$ est le signe de son premier terme, et : 
$$|R_n|\leq |u_{n+1}|$$
\end{itemize}
Ce résultat est valable aussi si on considère que S, la limite de la série, est le reste d'ordre $-1$.
\end{enon}
\begin{prop}
 Sous les hypothèses de la série alternée, la somme S est comprise entre deux sommes partielle consécutive, $S_n$ et $S_{n+1}$, et ceci $\forall n \in N$
\end{prop}
\begin{prop}
Si la série n'est alternée, et ne vérifie le fait que la suite ($|u_n|$) n'est décroissante qu'a partir d'un rang $n_0$, alors la série converge toujours, et le regle sur le reste reste valable, à partir du rang $n_0$.
\end{prop}
Lors d'exercice, le point le plus souvent difficile est de démontrer que la suite des valeurs absolu décroit.
\section{Quelques espaces remarquables}
\begin{prop}
L'ensembles des suites $(u_n) \in K^{N}$ telque $\underset{n}\sum u_n$ converge est un sous espace vectoriel de $K^N$. (Stable par addition et par multiplication par un scalaire). 
\end{prop}
\subsection{Ensemble $l^1(K)$}
\begin{de}
L'ensemble $l^1(K)$ est l'ensemble des suites sommables à valeurs dans K. Cette ensemble est un sous espace vectoriel de l'espace vectoriel vu ci dessus.
\end{de}
\begin{prop}
Soit $(u_n)$ une suite à valeur dans K.\\
Si la suite $(u_n)$ est sommable, la serie $\underset{n}\sum u_n$ converge aussi (L'absolu convergence implique la convergence). $l^1(K)$ est donc inclu dans l'espace vectoriel vu au début de ce chapitre.
\end{prop}
\begin{prop}
Supposons que u=$(u_n)$ et $v=(v_n)$ soient sommable. L'inégalité suivante : 
$$0 \leq |u_n + v_n|\leq|u_n|+|v_n|$$
Montre que u+v est également sommable.
\end{prop}
\begin{prop}
Soit $\lambda$ un scalaire.
$$u \in l^1(K) \Rightarrow \lambda.u \in l^1(K)$$
\end{prop}
\subsection{Ensemble $l^2(K)$}
\begin{de}
L'ensemble $l^2(K)$ est l'ensemble des suites de carrées sommable, c'est à dire telque :
$$\sum_n |u_n^2|$$
converge.\\
On montre que l'ensemble précédent est inclu dans cette ensemble.
\end{de}
\section{Sommation par paquets ou associativité de la sommation}
\begin{de}
Soit $(u_n)$ une suite à valeur dans K telque $\underset{n}\sum u_n$ converge.\\
Soit :
$$S = \sum_{k=0}^{\infty} u_k$$
Donnons nous une application $\phi$ défini par :
$$\phi : N \rightarrow N$$
telque $\phi$ soit une application strictement croissante.\\
Une telle application étant donnée, définisoons à partir de $(u_n)$ et de $\phi$ une nouvelle suite $(v_n)$ de terme général : 
$$v_n = u_{\phi_{(n-1)+1}}+...+u_{\phi_n}$$
\end{de}
\begin{prop}
Avec les notations précédentes, le fait que la série de terme général $u_n$ converge implique que la série de terme général $v_n$ converge, et vers S aussi.
\end{prop}

