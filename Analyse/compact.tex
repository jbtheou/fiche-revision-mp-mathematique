\chapter{Compact}
\section{Définitions et propriétés}
\subsection{Définition de Bolzano-Weierstrass}
\begin{de}
Un sous ensemble C d'un espace vectoriel normée E, munie de la norme $\parallel~\parallel$, est dit compact si de toute suite $(\gamma_n)$ à valeur dans C, on peut extraire une sous-suite $(\gamma_{\phi(n)})$, avec $\phi$ une application strictement croissante strictement de N dans N, qui converge vers un éléments de C.
\end{de}
\subsection{Théorème}
\begin{theo}
Tout segment [a,b] inclu dans $\Re$ est compact, c'est à dire que de toute suite bornée on peut extraire une suite qui converge.
\end{theo}
\begin{theo}
Soit $\Phi$ une fonction de E dans E', avec E et E' deux K espaces vectoriels munie respectivement des normes $\parallel~\parallel$ et $\parallel~\parallel'$, définie et continue sur un compact C de E. Alors : 
\begin{itemize}
 \item[$\rightarrow$] $\Phi(C)$ est un compact de E'
 \item[$\rightarrow$] $\Phi$ est bornée sur C, c'est à dire : $$\exists M \in R^+~ tq~ \forall \gamma \in C,~ \parallel \Phi(\gamma)\parallel' \leq M$$
 \item[$\rightarrow$] $\Phi$ est uniformement continue sur C
 \item[$\rightarrow$] Si E'=$\Re$, alors $\Phi$ atteint ces bornes.
\end{itemize}
\end{theo}
