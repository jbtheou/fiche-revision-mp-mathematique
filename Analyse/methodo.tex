\documentclass[a4paper,12 pt,oneside]{report}     % Type de document
\usepackage[utf8x]{inputenc}			  % Utilisation du UTF8
\usepackage{textcomp}				  % Accents dans les titres
\usepackage [ french ] {babel}                    % Titres en français
\usepackage [T1] {fontenc} 			  % Correspondance clavier -> document
\usepackage[Lenny]{fncychap}                      % Beau Chapitre
\usepackage{dsfont}                    	  	  % Pour afficher N,Z,D,Q,R,C
\usepackage{fancyhdr}                             % Entete et pied de pages
\usepackage [outerbars] {changebar}               % Positionnement barre en marge externe
\usepackage{amsmath}				  % Utilisation de la librairie de Maths
%\usepackage{amsfont}				  % Utilisation des polices de Maths
\usepackage{cite}                                 % Citations de la bibliographie
\usepackage{openbib}                              % Gestion avancée de Bibtex
\usepackage{enumerate}				  % Permet d'utiliser la fonction énumerate
\usepackage{dsfont}				  % Utilisation des polices Dsfont
\usepackage{ae}					  % Rend le PDF plus lisible

\newtheorem{de}{Définition}
\newtheorem{theo}{Théorème}
\newtheorem{prop}{Propriété}

%$\begin{bmatrix}
%  \cos(x)-X & -\sin x \\
%  \sin x & \cos(x)-X \\
%\end{bmatrix}$

%opening
\title{Méthodologie}
\author{MP}
\begin{document}
\maketitle
\tableofcontents
\chapter{Intégrales généralisées}
\section{Convergence d'une intégrale}
\subsection{Propriétés}
Il existe plusieurs trois propriétés qui permettent de prouver la convergence, d'une intégrale complexe à l'aide d'une intégrale "simple".
\begin{itemize}
 \item[$\rightarrow$] La convergence d'une intégrale de fonction positive par majoration (Implication)
 \item[$\rightarrow$] L'intégration par domination (Implication)
 \item[$\rightarrow$] La convergence des intégrale de fonction positive par equivalence (Équivalence)
\end{itemize}
\subsection{Fonctions "classique"}
Il existe plusieurs "cas" standard, au quel on peut se rapporter pour démontrer la convergence d'une intégrale
\subsubsection{En $\infty$}
Nous retiendrons que dans l'étude en $+\infty$, les $\alpha$ "tendent plus vers l'infini" (utilisation du > dans les relations). \\
Nous avons la règle de Riemann :
\begin{prop}
Soit f fonction continue par morceaux de $\left[a,\infty\right[$ dans K.\\
Si il existe $\alpha > 1$ telque : 
$$t^{\alpha}f(t) \underset{t \rightarrow \infty}\rightarrow 0$$
Alors f est intégrable ( converge absolument) sur $\left[a,\infty\right[$
\end{prop}
Nous avons aussi l'intégrale de Bertrand : 
\begin{prop}
Soit a un réel strictement superieur à 1, et $(\alpha,\beta) \in \Re^2$.
$$\left(\int_a^{\infty} \dfrac{dx}{x^{\alpha}ln(x)^{\beta}} \mbox{ converge }\right) \Leftrightarrow ( \alpha > 1,~ ou~ \alpha=1,\beta>1)$$
\end{prop}
La propriété suivante n'est qu'un cas particulier de l'intégrale de Bertrand, ou $\beta = 0$ :
\begin{prop}
Soit $a \in \Re$, $a>0$ :
$$\left(\int_a^{\infty} \dfrac{dt}{t^{\alpha}}\right) \mbox{ converge } \Leftrightarrow (\alpha > 1)$$
\end{prop}
\subsubsection{En 0}
Nous retiendrons que l'ordre de l'étude en 0, les $\alpha$ "tend en quelque sorte plus vers 0" (utilisation du < dans les relations ).\\
Nous avons la règle de Riemann :
\begin{prop}
Soit f fonction continue par morceaux de $\left]0,a\right]$ dans K.\\
Si il existe $\alpha < 1$ telque : 
$$t^{\alpha}f(t) \underset{t \rightarrow 0^+}\rightarrow 0$$
Alors f est intégrable ( converge absolument) sur $\left]0,a\right]$
\end{prop}
Dans le cas de l'intégralde de Bertrand, nous retiendrons que le second cas, $\alpha =1,~ \beta > 1$, est identique en l'infini et en 0.\\
Nous avons aussi l'intégrale de Bertrand : 
\begin{prop}
Soit a un réel telque $a \in \left]0,1\right[$, et $(\alpha,\beta) \in \Re^2$.
$$\left(\int_0^a \dfrac{dx}{x^{\alpha}ln(x)^{\beta}} \mbox{ converge }\right) \Leftrightarrow ( \alpha < 1 ,~ ou~ \alpha=1,\beta>1)$$
\end{prop}
La propriété suivante n'est qu'un cas particulier de l'intégrale de Bertrand, ou $\beta = 0$ :
\begin{prop}
Soit $a \in \Re$, $a>0$ :
$$\left(\int_0^a \dfrac{dt}{t^{\alpha}}\right) \mbox{ converge } \Leftrightarrow (\alpha < 1)$$
\end{prop}
\section{Divergence d'une intégrale}
\subsection{Les règles de Riemann}
Les règles de Riemann nous donne les moyens de prouver la divergence d'une intégrale.
\subsubsection{En $\infty$}
Soit f fonction de $\left[a,\infty\right[$ dans $\Re$, continue par morceaux. Si :
$$t.f(t) \underset{t\rightarrow\infty}\rightarrow 0$$
Alors :
$$\int_a^{\infty} \mbox{ diverge }$$
\subsubsection{En 0}
Soit f fonction de $\left]0,a\right]$ dans $\Re$, continue par morceaux. Si :
$$t.f(t) \underset{t\rightarrow 0}\rightarrow 0$$
Alors :
$$\int_0^a \mbox{ diverge }$$
\chapter{Développement asymptotiques}
Il existe une méthodologie "classique" à utiliser dans le cas d'un développement asymptotique.
\section{Fonction du type $f^{\alpha}$, ou ln(f)}
Pour obtenir le développement asymtotique de fonction du type $f^{\alpha}$, ou ln(f), on range les termes de façon prépondérant décroissante, puis on met toujours le terme prépondérant en facteur, et enfin on effectue un développement limité avec le reste, qui tend vers 0.\\
\underline{Exemple : }
Considérons le fonction ln($x^2+x+1$). Cette fonction est rangé en considérant la prépondérance en l'infini. On obtient donc : 
$$ln(x^2+x+1) = ln((x^2)(1+\dfrac{1}{x}+\dfrac{1}{x^2})$$
$$ln(x^2+x+1) = ln(x^2) + ln(1+\dfrac{1}{x}+\dfrac{1}{x^2})$$
On effectue un développement limité de la forme ln(1+u), avec u qui tend vers 0.
\section{Développement asymptotique de $S_n$ ou de $R_n$ dans le cas d'une série}
La première question à se poser est de savoir si la série $\underset{n}\sum u_n$ converge.
\subsection{Si la série converge}
Si la série converge, alors :
$$\lim_{n\rightarrow \infty} S_n = S$$
On poursuit en écrivant l'égalité suivante : 
$$S_n = S - R_n$$
Pour continuer le développement asymptotique, il faut donc détérminer un équivalent à $R_n$
\subsection{Si la série diverge}
Alors on cherche directement un équivalent de $S_n$
\subsection{Méthode à suivre}
Pour obtenir un équivalent de $S_n$, dans le cas divergent, ou un équivalent de $R_n$ dans le cas convergent, on simplifie le problème en remplacent $u_k$ par un équivalent $w_k$ plus simple.\\
On obtient alors : 
$$S_n \underset{\infty}\sim \sum_{k=n_0}^n w_k \mbox{ cas non sommable }$$
$$R_n \underset{\infty}\sim \sum_{k=n+1}^{\infty} w_k \mbox{ cas sommable }$$
L'utilisation de ceci ramène le problème de recherche d'équivalent de la somme partielle ou du reste de la série $\underset{k}\sum w_k$, avec $(w_k)$ appartenant à une échelle de comparaison. : 
$$w_k = k^{\alpha}.ln(k)^{\beta}.e^{P(k)}$$
Avec P(k) un pseudo polynome.\\
Pour obtenir un équivalent, on distingue deux cas : 
\begin{itemize}
 \item[$\rightarrow$] Si $(w_k)$ est à variation lente ($P = 0$ ou deg(P)<1), alors on encadre par des intégrales
 \item[$\rightarrow$] Si $(w_k)$ est à variation rapide (deg(P) $\geq$ 1), on utilise une comparaisons asymptotique entre $w_{k+1}-w_k$ et $w_k$.
\end{itemize}
Les relations de comparaions utilisable dans le second cas sont : 
$$\gg;\ll;\sim$$
\chapter{Séries}
\section{Propriétés générales}
Pour montrer la convergence d'une série $\underset{n} \sum u_n$, il faut déjà vérifier que $(u_n) \underset{\infty}\rightarrow 0$. Ceci est une condition necessaire, mais non suffisante.\\
Nous avons trois propriétés générales qui implique la convergence à l'aide d'un terme générale plus simple :
\begin{itemize}
 \item[$\rightarrow$] La convergence d'une série de terme général positive par majoration (Implication)
 \item[$\rightarrow$] L'intégration par domination (Implication)
 \item[$\rightarrow$] La convergence d'une serie de terme générale par équivalence (Équivalence)
\end{itemize}
\section{Règle usuelle}
\subsection{Convergence des séries de Riemann}
Soit une séries de Riemman, de terme général :
$$u_n  = \dfrac{1}{n^{\alpha}}$$
Cette série converge si et seulement si :
$$(\alpha > 1)$$
\subsection{Règle de Riemann - Convergence}
Soit $(u_n)$ une suite à valeur complexe.\\
Si il existe $\beta > 1$ telque :
$$(n^{\beta}.u_n \underset{\infty}\rightarrow 0$$
Alors $(u_n)$ est sommable
\subsection{Règle de Riemann - Divergence}
Soit $(u_n)$ une suite à valeur réelle.\\
Si : 
$$n.u_n \underset{\infty}\rightarrow \infty $$
Alors la série de terme générale $u_n$ diverge.
\subsection{Série de Bertrand}
Soit une série de Bertrand, de terme général :
$$u_n = \dfrac{1}{n^{\alpha}.ln(n)^{\beta}}$$
Cette série converge si et seulement si : 
$$(\alpha > 1~ ou~ \alpha=1~ et~ \beta>1)$$
\subsection{Règle de d'Alembert}
Soit $(u_n)$ une suite de réels telques $\forall n \geq n_0$, $u_n>0$ et telque : 
$$\dfrac{u_{n+1}}{u_n} \underset{\infty}\rightarrow l$$
Alors :
\begin{itemize}
 \item{$\rightarrow$} Si l > 1, alors la série de terme général $u_n$ diverge grossièrement
 \item{$\rightarrow$} Si l < 1, alors la série converge
\end{itemize}
\end{document}
