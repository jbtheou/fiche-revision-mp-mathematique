% ************************
% * fichier de préambule *
% ************************
 
% ***** extensions *****
\def\renamesymbol#1#2{
  \expandafter\let\expandafter\newsym\expandafter=\csname#2\endcsname
  \expandafter\global\expandafter\let\csname#1#2\endcsname=\newsym
  \expandafter\global\expandafter\let\csname#2\endcsname=\origsym
}

\usepackage[utf8x]{inputenc}			  % Utilisation du UTF8
\usepackage{textcomp}				  % Accents dans les titres
\usepackage [ french ] {babel}                    % Titres en français
\usepackage [T1] {fontenc} 			  % Correspondance clavier -> document
\usepackage{wasysym}
\renamesymbol{wasysym}{iint}
\renamesymbol{wasysym}{iiint}
\usepackage[Lenny]{fncychap}                      % Beau Chapitre
\usepackage{dsfont}                    	  % Pour afficher N,Z,D,Q,R,C
\usepackage{fancyhdr}                             % Entete et pied de pages
\usepackage [outerbars] {changebar}               % Positionnement barre en marge externe
\usepackage{amsmath}				  % Utilisation de la librairie de Maths
\usepackage{amssymb}
%\usepackage{amsfont}				  % Utilisation des polices de Maths
\usepackage{enumerate}				  % Permet d'utiliser la fonction énumerate
\usepackage{dsfont}				  % Utilisation des polices Dsfont
%\usepackage{ae}					  % Rend le PDF plus lisible
\usepackage[pdftex]{graphicx}                 % dernière étant la langue principale
\usepackage{color}
%\usepackage{palatino}
\usepackage{paralist}
\usepackage{natbib}
\usepackage[margin=3cm]{geometry}
\usepackage{fancyhdr}
\usepackage[Lenny]{fncychap}
\usepackage{graphicx}
\usepackage{wrapfig}
\usepackage{url}
\usepackage{graphics}
\usepackage{blkarray} 


\definecolor{Dark}{gray}{.2}
\definecolor{Medium}{gray}{.6}
\definecolor{Light}{gray}{.8}
\newcommand*{\plogo}{\fbox{$\mathcal{PL}$}}


\newtheorem{de}{Définition}
\newtheorem{theo}{Théorème}
\newtheorem{prop}{Propriété}
\newtheorem{princ}{Principe}
\newtheorem{conv}{Convention}
\newtheorem{loi}{Loi}
\newtheorem{voc}{Vocabulaire}
\newtheorem{enon}{\'Enonc\'e}
\newtheorem{nota}{Nota}
\newtheorem{propfond}{Propriété fondamentale}
\newtheorem{lemme}{Lemme}
\newtheorem{corro}{Corollaire}
\newtheorem{corr}{Corollaire}
\newtheorem{coro}{Corollaire}

\newtheorem{hypo}{Hypothèses générales}
\newtheorem{gene}{Généralisation}

\newcommand{\mesbloc}[1]{\begin{array}{|c|} \hline #1 \\ \hline \end{array}}

\makeatletter
\newlength\BA@height
\def\BA@leftdel#1#2#3{%
  \setlength\BA@height{\dimen\z@}%
  \addtolength\BA@height{5pt}%
  \llap{%
    {#1}$\left#2\vrule height \BA@height width\z@ \right.$\kern-#3}}

\def\BA@rightdel#1#2#3{%
  \setlength\BA@height{\dimen\z@}%
  \addtolength\BA@height{5pt}%
  \rlap{%
    \kern-#3$\left.\vrule height \BA@height width\z@ \right#1${#2}}}%
\makeatother 

\newlength{\drop}


\newcommand*{\titleGMPHY}{\begingroup% Gentle Madness
\setlength{\drop}{0.1\textheight}
%\vspace*{\baselineskip}
\vfill
  \hbox{%
  \hspace*{0.2\textwidth}%
  \rule{1pt}{\textheight}
  \hspace*{0.05\textwidth}%
  \parbox[b]{0.75\textwidth}{
  \vbox{%
    %\vspace{\drop}
    {\noindent\Huge\bfseries Fiches de Révision\\[0.5\baselineskip]
               MP}\\[2\baselineskip]
    {\Large\itshape TOME II - Mathématique}\\[4\baselineskip]
    {\Large Jean-Baptiste Théou}\par
    \vspace{0.5\textheight}
    {\noindent Creactive Commons - Version 0.1}\\[\baselineskip]
    }% end of vbox
    }% end of parbox
  }% end of hbox

\null
\endgroup}

\newcommand*{\titleGMMATH}{\begingroup% Gentle Madness
\setlength{\drop}{0.1\textheight}
%\vspace*{\baselineskip}
\vfill
  \hbox{%
  \hspace*{0.2\textwidth}%
  \rule{1pt}{\textheight}
  \hspace*{0.05\textwidth}%
  \parbox[b]{0.75\textwidth}{
  \vbox{%
    %\vspace{\drop}
    {\noindent\Huge\bfseries Fiches de Révision\\[0.5\baselineskip]
               MPSI}\\[2\baselineskip]
    {\Large\itshape TOME II - Mathématiques}\\[4\baselineskip]
    {\Large Jean-Baptiste Théou}\par
    \vspace{0.5\textheight}
    {\noindent Creactive Commons}\\[\baselineskip]
    }% end of vbox
    }% end of parbox
  }% end of hbox

\null
\endgroup}
