\chapter{Séries}
\section{Propriétés générales}
Pour montrer la convergence d'une série $\underset{n} \sum u_n$, il faut déjà vérifier que $(u_n) \underset{\infty}\rightarrow 0$. Ceci est une condition necessaire, mais non suffisante.\\
Nous avons trois propriétés générales qui implique la convergence à l'aide d'un terme générale plus simple :
\begin{itemize}
 \item[$\rightarrow$] La convergence d'une série de terme général positive par majoration (Implication)
 \item[$\rightarrow$] L'intégration par domination (Implication)
 \item[$\rightarrow$] La convergence d'une serie de terme générale par équivalence (Équivalence)
\end{itemize}
\section{Règle usuelle}
\subsection{Convergence des séries de Riemann}
Soit une séries de Riemman, de terme général :
$$u_n  = \dfrac{1}{n^{\alpha}}$$
Cette série converge si et seulement si :
$$(\alpha > 1)$$
\subsection{Règle de Riemann - Convergence}
Soit $(u_n)$ une suite à valeur complexe.\\
Si il existe $\beta > 1$ telque :
$$(n^{\beta}.u_n \underset{\infty}\rightarrow 0$$
Alors $(u_n)$ est sommable
\subsection{Règle de Riemann - Divergence}
Soit $(u_n)$ une suite à valeur réelle.\\
Si : 
$$n.u_n \underset{\infty}\rightarrow \infty $$
Alors la série de terme générale $u_n$ diverge.
\subsection{Série de Bertrand}
Soit une série de Bertrand, de terme général :
$$u_n = \dfrac{1}{n^{\alpha}.ln(n)^{\beta}}$$
Cette série converge si et seulement si : 
$$(\alpha > 1~ ou~ \alpha=1~ et~ \beta>1)$$
\subsection{Règle de d'Alembert}
Soit $(u_n)$ une suite de réels telques $\forall n \geq n_0$, $u_n>0$ et telque : 
$$\dfrac{u_{n+1}}{u_n} \underset{\infty}\rightarrow l$$
Alors :
\begin{itemize}
 \item{$\rightarrow$} Si l > 1, alors la série de terme général $u_n$ diverge grossièrement
 \item{$\rightarrow$} Si l < 1, alors la série converge
\end{itemize}