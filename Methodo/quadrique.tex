\chapter{Points importants pour obtenir l'équation réduit d'un quadrique $(\Sigma)$ et l'identifier}
Nous avons les points suivants :
\begin{itemize}
 \item[$\rightarrow$] Réduire orthonormalement la forme quadratique q associé à $(\Sigma)$. Dans certains cas, pour identifier ($\Sigma$), la connaisance des valeurs propres suffit.
 \item[$\rightarrow$] Si l'équation réduite f(X,Y,Z) = 0 est homogène, donc de degrés 2, $(\Sigma)$ est une surface conique de sommet $\Omega$.
 \item[$\rightarrow$] Si l'équation réduite ne contient pas de Z (par exemple) du type f(X,Y)=0, ($\Sigma$) est une surface cylindrique de génératrice parallèle à $\Omega z$.
 \item[$\rightarrow$] S'il s'agit d'identifier ($\Sigma$), on peut encore simplifier l'équation réduite en transformant ($\Sigma$) par des affinités droites de base l'un des plans de coordonnées.
\end{itemize}

