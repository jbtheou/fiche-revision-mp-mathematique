\chapter{Développement asymptotiques}
Il existe une méthodologie "classique" à utiliser dans le cas d'un développement asymptotique.
\section{Fonction du type $f^{\alpha}$, ou ln(f)}
Pour obtenir le développement asymtotique de fonction du type $f^{\alpha}$, ou ln(f), on range les termes de façon prépondérant décroissante, puis on met toujours le terme prépondérant en facteur, et enfin on effectue un développement limité avec le reste, qui tend vers 0.\\
\underline{Exemple : }
Considérons le fonction ln($x^2+x+1$). Cette fonction est rangé en considérant la prépondérance en l'infini. On obtient donc : 
$$ln(x^2+x+1) = ln((x^2)(1+\dfrac{1}{x}+\dfrac{1}{x^2})$$
$$ln(x^2+x+1) = ln(x^2) + ln(1+\dfrac{1}{x}+\dfrac{1}{x^2})$$
On effectue un développement limité de la forme ln(1+u), avec u qui tend vers 0.
\section{Développement asymptotique de $S_n$ ou de $R_n$ dans le cas d'une série}
La première question à se poser est de savoir si la série $\underset{n}\sum u_n$ converge.
\subsection{Si la série converge}
Si la série converge, alors :
$$\lim_{n\rightarrow \infty} S_n = S$$
On poursuit en écrivant l'égalité suivante : 
$$S_n = S - R_n$$
Pour continuer le développement asymptotique, il faut donc détérminer un équivalent à $R_n$
\subsection{Si la série diverge}
Alors on cherche directement un équivalent de $S_n$
\subsection{Méthode à suivre}
Pour obtenir un équivalent de $S_n$, dans le cas divergent, ou un équivalent de $R_n$ dans le cas convergent, on simplifie le problème en remplacent $u_k$ par un équivalent $w_k$ plus simple.\\
On obtient alors : 
$$S_n \underset{\infty}\sim \sum_{k=n_0}^n w_k \mbox{ cas non sommable }$$
$$R_n \underset{\infty}\sim \sum_{k=n+1}^{\infty} w_k \mbox{ cas sommable }$$
L'utilisation de ceci ramène le problème de recherche d'équivalent de la somme partielle ou du reste de la série $\underset{k}\sum w_k$, avec $(w_k)$ appartenant à une échelle de comparaison. : 
$$w_k = k^{\alpha}.ln(k)^{\beta}.e^{P(k)}$$
Avec P(k) un pseudo polynome.\\
Pour obtenir un équivalent, on distingue deux cas : 
\begin{itemize}
 \item[$\rightarrow$] Si $(w_k)$ est à variation lente ($P = 0$ ou deg(P)<1), alors on encadre par des intégrales
 \item[$\rightarrow$] Si $(w_k)$ est à variation rapide (deg(P) $\geq$ 1), on utilise une comparaisons asymptotique entre $w_{k+1}-w_k$ et $w_k$.
\end{itemize}
Les relations de comparaions utilisable dans le second cas sont : 
$$\gg;\ll;\sim$$