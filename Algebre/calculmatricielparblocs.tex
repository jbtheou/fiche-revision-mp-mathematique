\chapter{Calcul matriciel par blocs}
\section{Produit matriciel par blocs}
Considérons deux matrices A et A' : 
\[A = \begin{blockarray}{cccc}
        & \overset{\beta_1}\leftrightarrow & \dots & \overset{\beta_p}\leftrightarrow  \\
        \begin{block}{c(ccc)}
        \alpha_1 \updownarrow & \mesbloc{A_{11}}  & \dots & \mesbloc{A_{1p}}  \\
        \vdots  & \vdots & \ddots & \vdots      \\
        \alpha_n \updownarrow   & \mesbloc{A_{n1}} & \dots  & \mesbloc{A_{np}}   \\
        \end{block}
        \end{blockarray}
\] 
\[A' = \begin{blockarray}{cccc}
        & \overset{\gamma_1}\leftrightarrow & \dots & \overset{\gamma_q}\leftrightarrow  \\
        \begin{block}{c(ccc)}
        \beta_1 \updownarrow & \mesbloc{A'_{11}}  & \dots & \mesbloc{A'_{1q}}  \\
        \vdots  & \vdots & \ddots & \vdots      \\
        \beta_p \updownarrow   & \mesbloc{A'_{p1}} & \dots  & \mesbloc{A'_{pq}}   \\
        \end{block}
        \end{blockarray}
\] 
Avec :
$$\begin{cases}
A_{ij} \in  \mathcal{M}_{\alpha_i,\beta_j}(K) \\  A'_{kl} \in  \mathcal{M}_{\beta_k,\gamma_l}(K)
\end{cases}$$
On défini de plus : 
$$\begin{cases}
\alpha = \alpha_1 + \dots + \alpha_n \\ \beta = \beta_1 + \dots + \beta_p \\ \gamma = \gamma_1 + \dots + \gamma_q
\end{cases}$$
On obtient, pour le produit de A' par A :
\[A.A' = \begin{blockarray}{ccc}
        \begin{block}{(ccc)}
         \mesbloc{C_{11}}  & \dots & \mesbloc{C_{1q}}  \\
         \vdots & \ddots & \vdots      \\
         \mesbloc{C_{n1}} & \dots  & \mesbloc{C_{nq}}   \\
        \end{block}
        \end{blockarray}
\] 
Avec : 
$$C_{ij} = \sum_{k=1}^p A_{ik}.A'_{kj}$$
L'ordre du produit réalisé dans la somme a une importance à priori, car le produit des matrices n'est pas commuatif, à priori.
\subsection{Cas particuliers}
\subsubsection{Matrices diagonales}
Soient A et A' deux matrices diagonales : 

\[A = \begin{blockarray}{cccc}
         \overset{\beta_1}\leftrightarrow & \dots & \overset{\beta_n}\leftrightarrow  \\
        \begin{block}{(ccc)c}
         \mesbloc{A_{11}}  &  & (0)  & ~~~~~ \\
                            &\ddots  & & ~~~~~     \\
         (0)                &  & \mesbloc{A_{nn}}  & ~~~~~\\
        \end{block}
        \end{blockarray}
\]

\[A' = \begin{blockarray}{cccc}
        \begin{block}{(ccc)c}
         \mesbloc{A'_{11}}  &  & (0) & \updownarrow \beta_1   \\
                            &\ddots  &      &  \\
         (0)                &  & \mesbloc{A'_{nn}} & \updownarrow\beta_n   \\
        \end{block}
        \end{blockarray}
\] 
On obtient, pour le produit : 
\[A.A' = \begin{blockarray}{ccc}
        \begin{block}{(ccc)}
         \mesbloc{A_{11}.A'_{11}}  &  & (0)    \\
                            &\ddots  &        \\
         (0)                &  & \mesbloc{A_{nn}.A'_{nn}}   \\
        \end{block}
        \end{blockarray}
\] 
\subsubsection{Matrices Triangulaires}
Soient A et A' deux matrices triangulaire : 
\[A = \begin{blockarray}{cccc}
         \overset{\beta_1}\leftrightarrow & \dots & \overset{\beta_n}\leftrightarrow  \\
        \begin{block}{(ccc)c}
         \mesbloc{A_{11}}  &  & (A)  & ~~~~~ \\
                            &\ddots  & & ~~~~~     \\
         (0)                &  & \mesbloc{A_{nn}}  & ~~~~~\\
        \end{block}
        \end{blockarray}
\]

\[A' = \begin{blockarray}{cccc}
        \begin{block}{(ccc)c}
         \mesbloc{A'_{11}}  &  & (A') & \updownarrow \beta_1   \\
                            &\ddots  &      &  \\
         (0)                &  & \mesbloc{A'_{nn}} & \updownarrow\beta_n   \\
        \end{block}
        \end{blockarray}
\] 
On obtient, pour le produit : 
\[A.A' = \begin{blockarray}{ccc}
        \begin{block}{(ccc)}
         \mesbloc{A_{11}.A'_{11}}  &  & (B)    \\
                            &\ddots  &        \\
         (0)                &  & \mesbloc{A_{nn}.A'_{nn}}   \\
        \end{block}
        \end{blockarray}
\] 
\section{Calcul de déterminants de matrices triangulaires par blocs}
\subsection{Propriétés}
\begin{prop}
Nous avons la propriété suivantes : 
\[det~ \begin{blockarray}{cc}
        \overset{n_1}\leftrightarrow & \overset{n_2}\leftrightarrow \\
	\begin{block}{(cc)}
         \mesbloc{A_{11}}  &  \mesbloc{A_{12}}    \\
         0   		   & \mesbloc{A_{22}}   \\
        \end{block}
        \end{blockarray} = det(A_{11}).det(A_{22})
\] 
Avec :
$$\begin{cases}
A_{11} \in  \mathcal{M}_{n_1}(K) \\  A_{22} \in  \mathcal{M}_{n_2}(K)
\end{cases}$$
\end{prop}
\subsection{Généralisation}
\begin{prop}
On peut généraliser la propriété de la façon suivante : 
\[det ~ \begin{blockarray}{ccc}
	\overset{n_1}\leftrightarrow & \dots & \overset{n_p}\leftrightarrow \\
        \begin{block}{(ccc)}
         \mesbloc{A_{11}}  &  & (B)    \\
                          &\ddots  &        \\
         (0)                &  & \mesbloc{A_{pp}}   \\
        \end{block}
        \end{blockarray} = det(A_{11})\dots det(A_{pp})
\] 
\end{prop}