\chapter{Adjoint d'un endomorphisme, Endomorphisme et matrice symétrique}
\section{Rappel}
Soit E un $\mathbb{R}$ espace vectoriel, B=($\overrightarrow{e_1},\dots,\overrightarrow{e_n}$) une base orthonormé de E. Soit x et x' deux vecteurs de E. Soit X et X' les matrices de x et x' dans B. On obtient que : 
$$<x|x'> = ^tX.X'$$
\section{Propriétés - Définition d'un adjoint}
\begin{prop}
Soit E un $\mathbb{R}$ espace vectoriel euclidien de $f \in \mathcal{L}(E)$. Alors :
$$\exists ! f^* \in \mathcal(E)~ tq~ \forall (\overrightarrow{x},\overrightarrow{y})\in E^2~ <f(\overrightarrow{x})|\overrightarrow{y}> = <\overrightarrow{x}|f^*(\overrightarrow{\overrightarrow{y}})>$$
\end{prop}
\begin{de}
$f^*$ est appelé l'adjoint de f, pour un produit scalaire défini.
\end{de}
\begin{prop}
Si E est un $\mathbb{R}$ espace vectoriel euclidien, et B une base orthonormée de E, alors : 
$$mat_B f^* = ^t(mat_B f)$$ 
\end{prop}
\begin{de}
Un endomorphisme f d'un $\mathbb{R}$ espace vectoriel euclidien E est dit auto-adjoint ou symétrique si :
$$f^* = f$$
\end{de}
\begin{prop}
Soit $f \in \mathcal{L}(E)$, avec E un $\mathbb{R}$ espace vectoriel euclidien.
\begin{itemize}
 \item[$\rightarrow$] Si f est symétrique, alors quelque soit la base orthonormée B de E, la matrice de f dans cette base est symétrique.
 \item[$\rightarrow$] S'il existe une base orthonormée B de E telque la matrice de f soit symétrique, alors f est symétrique
\end{itemize}
\end{prop}
\section{Propriétés élémentaires}
Dans tout ce paragraphe, E désigne un $\mathbb{R}$ espace vectoriel euclidien. f et g désigne des endomorphisme de E.
\begin{prop}
Nous avons les propriétés suivantes :
\begin{itemize}
 \item[$\rightarrow$] $(f+g)^* = f^* + g^*$
 \item[$\rightarrow$] $\forall \lambda \in \mathbb{R}$ $(\lambda.f)^* = \lambda.f^*$
 \item[$\rightarrow$] $(fog)^* = g^*of^*$
 \item[$\rightarrow$] $(f^*)^* = f$
\end{itemize}
\end{prop}
\begin{prop}
Nous avons les propriétés suivantes : 
$$Ker(f^*) = (Im(f))^{\bot}$$
$$Im(f^*) = (Ker(f))^{\bot}$$
\end{prop}
\begin{prop}
Si $\lambda \in Sp(f)$ et $\lambda' \in Sp(f^*)$. Si $\lambda \neq \lambda'$, alors : \\
Ker($f-\lambda.Id$) et Ker($f^* - \lambda'.Id$) sont orthogonaux.
\end{prop}
\begin{prop}
Soit F un sous espace vectoriel de E.
$$(\mbox{ F est stable par f }) \Leftrightarrow ( F^{\bot}\mbox{ est un sous espace vectoriel stable par f })$$
\end{prop}
\subsection{Cas ou f et g sont symétrique}
\begin{coro}
L'ensemble des endomorphismes symétriques de E est un sous espace vectoriel de $\mathcal{L}(E)$.
\end{coro}
\begin{coro}
Nous avons les égalités suivantes : 
$$Ker(f) = (Im(f))^{\bot}$$
$$Im(f) = (Ker(f))^{\bot}$$
\end{coro}
\begin{coro}
Si $\lambda$ et $\lambda'$ sont deux valeurs propres distinct d'un endomorphisme f symétrique, alors : \\
Ker($f-\lambda.Id$) et Ker($f - \lambda'.Id$) sont orthogonaux.
\end{coro}
\begin{coro}
Si f est symétrie et si F est un sous espace vectoriel de E : 
$$(\mbox{ F est stable par f }) \Leftrightarrow ( \mbox{ F est un sous espace vectoriel stable par f })$$
\end{coro}
\section{Théorème d'orthogonalisation}
\begin{theo}
Soit E un $\mathbb{R}$ espace vectoriel euclidien de dimension finie n. Soit f un endomorphisme symétrique de E. Alors : 
\begin{itemize}
 \item[$\rightarrow$] $P_f$ est scindé sur $\mathbb{R}$, c'est à dire que le spectre complexe de f est égale au spectre réel de f.
 \item[$\rightarrow$] Les sous espaces vectoriels propre de f sont deux à deux orthogonaux et supplémentaires. En particulier, f est diagonalisable.
 \item[$\rightarrow$] Il existe une base orthonormée de diagonalisation de f, c'est à dire une base orthonormée de E formée de vecteurs propres de f.
 \item[$\rightarrow$] Réciproquement, si f est un endomorphisme de E orthonormalement diagonalisable, c'est à dire si f admet une base orthonormée de diagonalisation, alors f est symétrique.
\end{itemize}
\end{theo}
\subsection{Corollaire matricielle}
Si $A \in \mathcal{M}_n(\mathbb{R})$ est symétrique, alors A est orthonormalement diagonalisable, c'est à dire qu'il existe $P \in O_n(\mathbb{R})$ telque $P^{-1}.A.P$ soit diagonale. Dans ce cas $P^{-1} = P^{\bot}$
\section{Caractéristation par l'adjoint de certains endomorphismes classiques d'un espace euclidien}
Dans tout ce paragraphe, E désigne un $\mathbb{R}$ espace vectoriel euclidien, $f \in \mathcal{L}(E)$.
\begin{prop}
f est un projecteur orthogonal si et seulement si : 
$$\begin{cases}
   fof = f \\
   f^* = f \\
  \end{cases}
$$
\end{prop}
\begin{prop}
Nous avons la propriété suivante : 
$$(f \in O(E) ) \Leftrightarrow (fof^* = Id_E)$$
\end{prop}
\begin{prop}
Nous avons la propriété suivante : 
$$\mbox{ f est symétrique orthogonale } \Leftrightarrow \begin{cases}
                                                           f^2 = Id \\
							   f^* = f \\
                                                          \end{cases}
$$
\end{prop}
