\chapter{Espace vectoriel normé - Deuxième partie}
\section{Interieur d'un ensemble, ensemble ouvert}
\begin{de}
Soit (E,d) un espace métrique. En général, E est un K espace vectoriel munie d'une norme $\parallel~\parallel$, et d est la distance déduite de cette norme.\\
Soit A un ensemble non vide de E. On dit que $a \in E$ est intérieur à A, si il existe r>0 telque :
$$B(a,r) c A$$
\end{de}
\begin{de}
On appelle interieur de A, et on le note $\overset{o}A$, l'ensemble des points interieurs à a.
\end{de}
\begin{de}
A est dit ouvert si tout point de A est interieur à A. C'est à dire si :
$$A c \overset{o}A$$
Or, par définition, nous avons : 
$$\overset{o}A c A$$
Donc, nous avons l'équivalence suivante : 
$$(\mbox{ A est un ouvert }) \Leftrightarrow (A = \overset{o}A)$$
\end{de}
\begin{prop}
Nous avons les propriétés suivantes : 
\begin{itemize}
 \item[$\rightarrow$] $\emptyset$ est un ouvert par convention.
 \item[$\rightarrow$] E est un ouvert
 \item[$\rightarrow$] Une intersection finie d'ouvert est un ouvert.
 \item[$\rightarrow$] Une réunion quelconque (finie ou infinie) d'ouvert est un ouvert.
\end{itemize}
\end{prop}
\begin{prop}
$\overset{o}A$ est le plus grand (Au sens de l'inclusion) ouvert inclu dans A.
\end{prop}
\subsubsection{Exemple}
Nous avons un exemple classique : 
\begin{itemize}
 \item[$\rightarrow$] Les boules ouvertes d'une espace métrique sont des ouverts
\end{itemize}

\section{Adérence d'un ensemble, ensemble fermé}
Considérons toujours A, un sous ensemble non vide d'un espace métrique (E,d).
\begin{de}
On dit que $x \in E$ est adhérent à A s'il existe une suite ($a_n$) d'élements de A telque : 
$$a_n \underset{n \rightarrow \infty}\rightarrow x$$
x peut être ou ne pas être un élement de A.\\
Si d est une distance déduite d'une norme $\parallel~\parallel$, on peut écrire cette condition sous la forme : 
$$\parallel a_n - x \parallel \underset{n\rightarrow +\infty}\rightarrow 0$$
Par définition, tout point de A est adhérent à A
\end{de}
\begin{prop}
Nous avons la propriété suivante : 
$$( x \mbox{ est adhérent à A }) \Leftrightarrow (\forall \varepsilon > 0,~ B(x,\varepsilon) \cap A \neq \emptyset$$
On dit que x est adhérent à A si pour tout $\varepsilon >0$, B($x,\varepsilon$) rencontre A.
\end{prop}
\begin{de}
L'ensemble des points adhérent à A, s'appelle l'adhérence à A, et est noté $\overline{A}$. Par définition, nous avons donc : 
$$\overset{o}A c A c \overline{A}$$
\end{de}
\begin{de}
Un sous ensemble A d'un ensemble normé (E,d) est dit fermé s'il est égale à son adhérence, c'est à dire si A=$\overline{A}$. Et comme par définition nous avons l'une des inclusions, on obtient que A est fermé si et seulement si : 
$$\overline{A} c A$$
\end{de}
\begin{de}
On appelle complémentaire de A, et on le note $C_A$, l'ensemble défini par : 
$$C_A = E - A$$
\end{de}
\begin{prop}
Nous avons la propriété suivante :
\begin{center}
 A est fermé $\Leftrightarrow$ $C_A$ est ouvert
\end{center}
\end{prop}
\begin{prop}
Caractérisation séquentielle : \\
A est fermé si et seulement si toute suite convergente d'élement de A à sa limite dans A.
\end{prop}
\begin{prop}
Nous avons les propriétés suivantes : 
\begin{itemize}
 \item[$\rightarrow$] $\emptyset$ est un fermé
 \item[$\rightarrow$] E est un fermé
 \item[$\rightarrow$] La réunion d'un nombre fini de fermés est un fermé
 \item[$\rightarrow$] Une intersection quelconque de fermés est un fermé
\end{itemize}
\end{prop}
\begin{prop}
Nous avons les propriétes suivantes :
\begin{itemize}
 \item[$\rightarrow$] $\overline{A}$ est le plus petit fermé (au sens de l'inclusion) contenant A.
 \item[$\rightarrow$] Dans un espace vectorielle normé, la boule fermé $\overline{B}(x_0,r)$ est l'adhérence de B($x_0,r$). Celà n'est pas nécessairement vrai dans un espace métrique.
\end{itemize}
\end{prop}
\subsubsection{Exemples}
Il y a un exemple classique de fermé : 
\begin{itemize}
 \item[$\rightarrow$] Les boules fermés d'un espace métrique sont des fermés.
\end{itemize}
\section{Frontière}
\begin{de}
Soit A une partie non vide d'un espace métrique (E,d). On appelle frontière de A, et on le note parfois $\partial A$, l'ensemble défini par : 
$$\partial A = \overline{A} \cap \overline{C_A}$$
\end{de}
\begin{prop}
Nous avons la propriété suivante :
$$\partial A = \overline{A} - \overset{o}A$$
\end{prop}
\section{Diamètre d'une partie bornée}
\begin{de}
Une partie A d'un espace métrique (E,d) est dit borné si : 
$$\exists M \in \mathbb{R}_+,~\exists x_0 \in E,~tq~ \forall x \in A d(x_0,x) \leq M$$
Cette définition est équivalente à : 
$$\exists M \in \mathbb{R}_+,~\exists x_0 \in E,~tq~ A c \overline{B}(x_0,M)$$
\end{de}
\begin{prop}
Si cette condition est remplie, alors : 
$$\forall x_1 \in E,~ \exists M_1 \in \mathbb{R}_+,~ tq~\forall x \in A,~ d(x_1,x) \leq M_1$$
\end{prop}
\section{Ensembles compacts}
\section{Définitions et propriétés}
\subsection{Définition de Bolzano-Weierstrass}
\begin{de}
Un sous ensemble C d'un espace vectoriel normée E (ou un espace métrique (E,d), avec d la distance déduite de la norme), munie de la norme $\parallel~\parallel$, est dit compact si de toute suite $(\gamma_n)$ à valeur dans C, on peut extraire une sous-suite $(\gamma_{\phi(n)})$, avec $\phi$ une application strictement croissante de N dans N, qui converge vers un éléments de C.
\end{de}
\subsection{Théorème}
\begin{theo}
Tout segment [a,b] inclu dans $\Re$ est compact, c'est à dire que de toute suite bornée on peut extraire une suite qui converge.
\end{theo}
\begin{theo}
Nous avons les propriétes suivantes :
\begin{itemize}
 \item[$\rightarrow$] Tout compact est fermé et borné
 \item[$\rightarrow$] Tout fermé inclus dans un compact est compact
 \item[$\rightarrow$][$a_1,b_1$]$\times...\times$[$a_n,b_n$] est un compact de ($\mathbb{R}^n,\parallel~\parallel_{\infty}$), ou même de ($\mathbb{R}^n,\parallel~\parallel$) car toutes les normes sont équivalentes en dimension finie.
\end{itemize}
\end{theo}
\begin{theo}
Dans un espace vectoriel de dimension finie, tout ensemble, non vide, fermé et borné est un compact.
\end{theo}
\begin{prop}
Si $(x_n)$ est une suite d'élements d'un espace métrique convergeant vers l, alors : 
$$K = \left\lbrace x_n, n \in \mathbb{N} \right\rbrace \cup \left\lbrace l\right\rbrace  $$
est un compact.
\end{prop}
\begin{coro}
Soit f une application de E dans E', avec E et E' des espaces vectoriels normés ou des espaces métriques, définie sur une partie non vide A c E.\\
f est continue sur A si et seulement si f est continue sur tout compact inclu dans A.
\end{coro}
\begin{theo}
Soit f une fonction de E dans E', avec E et E' deux K espaces vectoriels (ou espace métrique) munie respectivement des normes $\parallel~\parallel$ et $\parallel~\parallel'$, définie et continue sur un compact C de E. Alors :
\begin{itemize}
 \item[$\rightarrow$] f(C) est un compact de E'
 \item[$\rightarrow$] f est bornée sur C, c'est à dire : $$\exists M \in R^+~ tq~ \forall \gamma \in C,~ \parallel f(\gamma)\parallel' \leq M$$
 \item[$\rightarrow$] f est uniformement continue sur C
 \item[$\rightarrow$] Si E'=$\Re$, alors f atteint ces bornes.
\end{itemize}
\end{theo}