\chapter{Formes quadratiques}
\section{Formes quadratiques sur $\mathbb{R}^n$}
\begin{de}
On appelle forme quadratique sur $\mathbb{R}^n$ toute application polynomiale homogène du second degrès de $\mathbb{R}^n$ dans $\mathbb{R}$, donc une application défini par : 
$$\mathbb{R}^n \overset{q}\rightarrow \mathbb{R}$$
$$X \mapsto q(X)$$
Si :
$$X = \begin{pmatrix}
       x_1\\
 	\vdots \\
	x_n
      \end{pmatrix}
$$
Alors : 
$$q(X) = \sum_{i=1}^n a_{ii}.x_i^2 + 2.\sum_{1 \leq i < j \leq n} a_{ij}.x_j.x_i$$
Avec : 
$$\forall i > j,~ a_{ij} = a_{ji}$$
\end{de}
\subsection{Forme bilinéaire symétrique associé à une forme quadratique}
\begin{prop}
Si q est une forme quadratique sur $\mathbb{R}^n$, il existe une unique forme bilinéaire symétrique $\varphi$ sur $\mathbb{R}^n$ telle que : 
$$\forall X \in \mathbb{R}~ q(X) = \varphi(X,X)$$
$\varphi$ est appelé forme polaire de q.\\
On montre que l'unicité est du à l'identité de polarisation, qui permet d'exprimer $\varphi$ explicitement en fonction de q.
\end{prop}
\subsection{Expressions matricielles de q et de $\varphi$}
\begin{prop}
Soient q et $\varphi$ une forme quadratique et sa forme polaire associé sur $\mathbb{R}^n$, alors : 
$$\forall X \in \mathbb{R}^n~ q(X) = ^tX.A.X$$
$$\forall(X,X') \in \mathbb{R}^n~ \varphi(X,X') = ^tX.A.X'$$
Avec A matrice symétrique.
\end{prop}
\begin{prop}
Il existe une unique matrice $A \in \mathcal{M}_n(\mathbb{R})$ symétrique telque : 
$$\forall X \in \mathbb{R}^n~ q(X) = ^tX.A.X$$
A est appelé la matrice de la forme quadratique q dans la base canonique, et se note :
$$A = mat_{can}(q)$$
De même, A est l'unique matrice symétrique de $\mathcal{M}_n(\mathbb{R})$ telque : 
$$\forall(X,X') \in (\mathbb{R}^n)^2~ \varphi(X,X') = t^X.A.X'$$
On obtient :
$$A = mat_{can}(q) = mat_{can}(\varphi) = (a_{ij})$$
Avec : 
$$\forall (i,j) \in [|1,n|]^2~ a_{ij} = \varphi(\overrightarrow{e_i},\overrightarrow{e_j})$$
Avec ($\overrightarrow{e_1},...,\overrightarrow{e_n}$) la base canonique de $\mathbb{R}^n$.
\end{prop}
\subsection{Endomorphisme symétrique de $\mathbb{R}^n$ associé à une forme quadratique q}
$\mathbb{R}^n$ est munie du produit scalaire canonique. q est une forme quadratique sur $\mathbb{R}^n$, de forme polaire $\varphi$, de matrice A sur la base canonique.
\begin{prop}
Il existe un unique endomorphisme f de $\mathbb{R}^n$ symétrique telque : 
$$\forall X \in \mathbb{R}^n~ q(X) = <f(X)| X >$$
Cet endomorphisme vérifie aussi : 
$$\forall (X,X') \in (\mathbb{R}^n)^2~ <f(X),X'> = <X|f(X')> = \varphi(X,X')$$
f est appelé l'endomorphisme symétrique associé à q dans $(\mathbb{R}^n,<|>)$. De plus, nous avons : 
$$mat_{cam}(f)=mat_{cam}(q)$$
\end{prop}
\subsection{Théorème de réduction orthonormale d'une forme quadratique de $\mathbb{R}^n$}
\begin{theo}
$\mathbb{R}^n$ étant munie de son produit scalaire canonique euclidien, et q étant une forme quadratique sur $\mathbb{R}^n$, il existe B=($\overrightarrow{u_1},...,\overrightarrow{u_n}$) base orthonormée de $\mathbb{R}^n$ telque si :
$$X = x_1.\overrightarrow{u_1} + ... + x_n.\overrightarrow{u_n}$$
alors : 
$$q(X) = \lambda_1.x_1'^2 + ... + \lambda_n.x_n'^2$$
ou les $\lambda_i$ sont des valeurs propres associés à f, l'endomorphisme symétrique canoniquement associé à q. Une telle base B s'obtient en orthonormalisent f. En résumé, par un changement de base, on fait "disparaitre" les termes rectangles ($x_i.x_j$ ...)
\end{theo}
\subsection{Version matricielle du théorème de réduction orthonormale}
Soit q une forme quadratique sur $\mathbb{R}^n$, et $\varphi$ sa forme polaire. Soit A la matrice de q dans la base canonique, et f l'endomorphisme symétrique canoniquement associé à q dans $\mathbb{R}^n$ euclidien canonique.\\
Si $X \in \mathbb{R}^n$ : 
$$q(X) = ^tX.A.X = <f(X) | X>$$
\subsubsection{Changement de base}
Soit B une nouvelle base de $\mathbb{R}^n$, B=($\overrightarrow{u_1},...,\overrightarrow{u_n}$) et P la matrice de passage entre la base canonique et B. Soit $X' = mat_B(X)$. On sait que : 
$$X = PX'$$
Donc : 
$$q(X) = ^tX.(^tP.A.P).X'$$
Posons A' = $^tP.A.P$. A' est symétrique et c'est l'unique matrice symétrique vérifiant l'expression précédent $\forall X \in \mathbb{R}^n$. Par définition : 
$$A' = mat_B(q)$$
\begin{prop}
Si A=($a'_{ij}$), alors :
$$\forall (i,j) \in [|1,n|]^2~a'_{ij} = \varphi(\overrightarrow{u_i},\overrightarrow{u_j}) $$
De plus, si B est une base orthonormée, alors : 
$$A- = ^tP.A.P = P^{-1}.A.P$$
\end{prop}

\section{Généralisation : Formes quadratiques sur un espace vectoriel E}
\begin{de}
Soit E un $\mathbb{R}$ espace vectoriel quelconque, non nécessairement de dimension finie. On appelle forme quadratique sur E une application q : E $\rightarrow$ $\mathbb{R}$ tel qu'il existe $\varphi$ application de $E\times E \rightarrow \mathbb{R}$ bilinéaire symétrique vérifiant : 
$$\forall \overrightarrow{x} \in E, q(\overrightarrow{x}) = \varphi(\overrightarrow{x},\overrightarrow{x})$$
\end{de}

\begin{prop}
Avec les notations précédentes, $\varphi$ est alors unique et appelé forme polaire q.
\end{prop}
\begin{prop}
En conservant les notations précédentes : 
$$\forall (\overrightarrow{x},\overrightarrow{y}) \in E^2~ \varphi(\overrightarrow{x},\overrightarrow{y}) = \dfrac{1}{4}[q(\overrightarrow{x} + \overrightarrow{y}) - q(\overrightarrow{x}-\overrightarrow{y})]$$ 
\end{prop}
\begin{prop}
Si q est une forme quadratique sur E, $\forall \lambda \in \mathbb{R}$ :
$$\forall \overrightarrow{x} \in E~ q(\lambda.\overrightarrow{x}) = \lambda^2.q(\overrightarrow{x})$$ 
\end{prop}
\subsection{Exemples}
Les formes quadratiques défini à la section précédente sur $\mathbb{R}^n$ sont des formes quadratiques au sens de la nouvelle définition générale. Réciproquement, toute forme quadratique sur $E = \mathbb{R}^n$ au sens de la nouvelle définition est une application polynomiale homogène du $2^{nd}$ degrès. En résumé, les formes quadratiques définies au paragraphe 1 sur $\mathbb{R}^n$ sont des cas particuliers des formes quadratiques défini globalement.
\subsection{Expression dans une base, matrice d'une forme quadratique dans une base}
\begin{prop}
Soit q, application de E dans $\mathbb{R}$, une forme quadratique sur un $\mathbb{R}$ espace vectoriel E de dimension n. $\varphi$ est sa forme polaire. Soit B=($\overrightarrow{u_1},...,\overrightarrow{u_n}$) une base de E. Si : 
$$\begin{cases}
   \overrightarrow{x} = x_1.\overrightarrow{u_1} + ... + x_n.\overrightarrow{u_n} \\
   \overrightarrow{y} = y_1.\overrightarrow{u_1} + ... + y_n.\overrightarrow{u_n} \\
  \end{cases}
$$
Alors : 
$$
\begin{cases}
 q(\overrightarrow{x}) = \sum_{i=1}^n a_{ii}.x_i^2 + 2.\sum_{1 \leq i < j \leq n} a_{ij}.x_i.x_j \\
 \varphi(\overrightarrow{x},\overrightarrow{y}) = \sum_{(i,j) \in [|1,n|]^2} a_{ij}.x_i.x_j \\
\end{cases}
$$
Réciproquement, si q est une application E dans $\mathbb{R}$, alors : 
$$\forall \overrightarrow{x} \in E~ q(\overrightarrow{x}) \sum_{i=1}^n a_{ii}.x_i^2 + 2.\sum_{1\leq i<j\leq n} a_{ij}.x_i.x_j$$
alors q est une forme quadratique et sa forme polaire $\varphi$ est définie par : 
$$\varphi : E\times E \rightarrow \mathbb{R}$$
$$(\overrightarrow{x},\overrightarrow{y}) \mapsto \sum_{(i,j) \in [1,n]^2} a_{ij}.x_i.x_j$$
\end{prop}
\begin{de}
Avec les notations précédentes, la matrice A=($a_{ij}$) telque :
$$\forall (i,j) \in [1,n]~ a_{ij}= \varphi(\overrightarrow{u_i},\overrightarrow{u_j})$$
A est appelé matrice de q ou de $\varphi$ dans la base B.
\end{de}
\begin{prop}
Avec les notations et définitions précédentes, A est l'unique matrice symétrique de $\mathcal{M}_n(\mathbb{R})$ telque :
$$\forall \overrightarrow{x} = \sum_{i=1}^n x_i.\overrightarrow{u_i} \in E, q(\overrightarrow{x}) = ^tX.A.X$$
Avec X la matrice de $\overrightarrow{x}$ dans B.
\end{prop}
\subsection{Changement de Base}
Soit E un $\mathbb{R}$ espace vectoriel de dimension finie n, B et B' deux bases de E. Soit P la matrice de passage entre B et B'. q est une forme quadratique sur E. A est la matrice de q dans B, A' la matrice de q dans B' : 
$$A' = t^P.A.P$$
\begin{de}
Si A et A' deux matrices de $\mathcal{M}_n(\mathbb{R})$. On dit que A et A' sont congruente s'il existe $P \in Gl_n(\mathbb{R})$ telque : 
$$A' = ^tP.A.P$$
On défini ainsi une relation d'équivalence sur E. Deux matrices congruentes sont en particulier équivalentes. Donc deux matrices congruentes ont même rang.
\end{de}
\begin{de}
Soit q une forme quadratique sur un $\mathbb{R}$ espace vectoriel de dimension finie. On appelle rang de q le rang de sa matrice sur une base B de E. Cette définition ne dépend pas de la base B choisie.
\end{de}
\begin{de}
Si E est un $\mathbb{R}$ espace vectoriel de dimension finie n et q une forme quadratique sur E. On dit que q est non dégénéré si rang(q)=n, c'est à dire si la matrice de q sur une base est inversible.
\end{de}
\subsection{Endomorphisme symétriques associés à une forme quadratique dans un espace vectoriel euclidien}
\begin{prop}
Soit q une forme quadratique sur un $\mathbb{R}$ espace vectoriel euclidien E, alors il existe un unique endomorphisme f$\in \mathcal{L}(E)$, symétrique ( pour le produit scalaire de E) telque : 
$$\forall \overrightarrow{x} \in E~ q(\overrightarrow{x}) = <f(\overrightarrow{x})|\overrightarrow{x}>$$
En outre, quelque soit la base B orthonormée de E : 
$$mat_B(f) = mat_B(q)$$
De plus, si $\varphi$ est la forme polaire de q :
$$\forall (\overrightarrow{x},\overrightarrow{y})\in E^2,~ \varphi(\overrightarrow{x},\overrightarrow{y}) = <f(\overrightarrow{x}) | \overrightarrow{x}> = <\overrightarrow{x} | f(\overrightarrow{x})>$$
\end{prop}
\subsection{Réduction orthonormale d'une forme quadratique dans un espace vectoriel euclidien}
\begin{theo}
Soit E un $\mathbb{R}$ espace vectoriel euclidien de dimension n, q une forme quadratique sur E, alors il existe une base B orthonormée de E et des réels $\lambda_1,...,\lambda_n$ telque : 
$$\forall \overrightarrow{x} = x_1.\overrightarrow{u_1} + ... + x_n.\overrightarrow{u_n} \in E~ q(\overrightarrow{x}) = \sum_{i=1}^n \lambda_i.x_i^2$$
Une telle base B s'obtient en orthonormalisent l'endomorphisme symétrique f associé à q.\\
Avec les notations précédentes, si $\overrightarrow{y} = y_1.\overrightarrow{u_1} + ... + y_n.\overrightarrow{u_n}$ et si $\varphi$ est la forme polaire de q : 
$$\varphi(\overrightarrow{x},\overrightarrow{y}) = \sum_{i=1}^n \lambda_i.x_i.y_i$$ 
\end{theo}
\section{Application à la réduction de coniques}
\begin{de}
Soit E un $\mathbb{R}$ espace affine de dimension 2. Soit $\mathcal{R}$ un repère de E. On appelle conique au sens large un sous ensemble $(\Gamma)$ de E admettent dans $\mathcal{R}$ une équation du type : 
$$P(x,y) = 0$$
Avec P polynome du second degrès : 
$$P(x,y) = a.x^2 + 2.b.x.y + c.y^2 + u.x + v.y + h$$
Avec $(a,b,c) \neq (0,0,0)$. Soit $\Omega = (O,\overrightarrow{i},\overrightarrow{j})$. On défini une forme quadratique q par : 
$$\overrightarrow{E} \rightarrow \mathcal{R}$$
$$x.\overrightarrow{i} + y.\overrightarrow{j} \mapsto a.x^2 + 2.b.x.y + c.y^2$$
Avec $\overrightarrow{E}$ l'espace vectoriel associé à E. Nous avons donc : 
$$mat_{(\overrightarrow{i},\overrightarrow{j})}(q) = \begin{pmatrix}
                                                      a & b \\
						      b & c \\
                                                     \end{pmatrix}
$$
On défini de même une forme linéaire l par : 
$$\overrightarrow{E} \rightarrow \mathcal{R}$$
$$x.\overrightarrow{i} + y.\overrightarrow{j} \mapsto u.x + v.y$$
Avec ces notations, l'équation de $(\Gamma)$ s'écrit : 
$$M \in (\Gamma) \Leftrightarrow q(\overrightarrow{OM}) + l(\overrightarrow{OM}) + h = 0$$
\end{de}
\begin{prop}
Si $\mathcal{R}'$ est un autre repère de (E) si ($\Gamma$) a pour équation P(x,y)=0, avec P polynome du $2^{nd}$ degrès. dans le repère $\mathcal{R}$, alors ($\Gamma$) a pour équation :
$$Q(x,y) = 0$$
avec Q polynome du $2^{nd}$ degrès. Autrement dit, la définition ci-dessus ne dépend pas du repère choisi.
\end{prop}
\subsection{Cas général}
On suppose ici E euclidien et $\mathcal{R}$ un repère orthonormée. En réduisant d'abord orthonormalement q, ce qui revient à un changement de repère par rotation, puis en effectuant éventuellement un deuxième changement de repère par translation, on montre qu'il existe un repère orthonormée $\mathcal{R}_1$ dans lequel $(\Gamma)$ à une équation du type (a,b>0):
\begin{itemize}
 \item[$\rightarrow$] Coniques non dégénéré :
\begin{itemize}
 \item[$\rightarrow$]$\dfrac{x_1^2}{a^2} + \dfrac{y_1^2}{b^2} = 1$. C'est une ellipse. q a deux valeurs propres de même signe
 \item[$\rightarrow$]$\dfrac{x_1^2}{a^2} - \dfrac{y_1^2}{b^2} = 1$. C'est une hyperbole. q a deux valeurs propres de signe contraire
 \item[$\rightarrow$]$x_1^2 = 2.p.y_1$. C'est une parabole. q a 1 valeur propre nulle.
\end{itemize}
 \item[$\rightarrow$] Coniques dégénéré :
\begin{itemize}
 \item[$\rightarrow$] $\dfrac{x_1^2}{a^2} + \dfrac{y_1^2}{b^2} = 0$. ($\Gamma$) est réduit au point (0,0).
 \item[$\rightarrow$] $\dfrac{x_1^2}{a^2} + \dfrac{y_1^2}{b^2} = \alpha < 0$. ($\Gamma$) est égale au vide. 
 \item[$\rightarrow$] $\dfrac{x_1^2}{a^2} - \dfrac{y_1^2}{b^2} = 0$. ($\Gamma$) est la réunion de deux droites sécantes.
\end{itemize}
\end{itemize}
Pour déterminer la conique associé à $(\Gamma)$, on peut utiliser la méthode suivantes : Soit A la matrice de q ou de f dans une base orthonormée de dimension 2. Soit $\lambda_1$ et $\lambda_2$ les valeurs propres associé à A. On montre que :
\begin{itemize}
 \item[$\rightarrow$] $\lambda_1$ et $\lambda_2$ non nulle et de même signe $\Leftrightarrow$ det(A) > 0
 \item[$\rightarrow$] $\lambda_1$ et $\lambda_2$ non nulle et de signe contraire $\Leftrightarrow$ det(A) < 0
\end{itemize}
De plus, si $(\Gamma)$ est une conique dégénéré, les sous espaces propres vectoriels de f fournissent les dimensions vectorielles des axes de la conique.\\
Dans le cas d'un hyperbole, on obtient les droites parallèles au asymptotes en annulant q.
\section{Catalogue des quadriques dans un $\mathbb{R}$ espace affine E (euclidien) de dimension 3}
\begin{de}
On appelle quadrique de (E) un ensemble ($\Sigma$) telque qu'il existe un repère $\mathbb{R}$ de E dans lequel ($\Sigma$) a pour équation P(x,y,z) = 0, avec P un polynome à coefficiant réele du $2^{nd}$ degrès.
\end{de}
\begin{prop}
Dans ce cas, $\forall \mathcal{R}'$ repère de (E), ($\Sigma$) admet également une équation polynomiale de degrès 2 dans $\mathcal{R}'$. 
\end{prop}
\subsection{Catalogue des quadriques}
On suppose E euclidien, $\mathcal{R}$ un repère orthonormée. Pour simplifier l'équation de $(\Sigma)$, on commence par réduire orthonormalement la forme quadratique q, c'est à dire diagonalisé orthonormalement l'endomorphisme symétrique f qui va de $\overrightarrow{E}$ dans $\overrightarrow{E}$ associé à q. On sait que l'on peut eliminer les termes rectangles. On va classer les quadriques possibles suivant, essentiellement, le nombre de valeur propre non nulle. 
\subsubsection{Les trois valeurs propres sont non nulle}
En développent les expressions, on montre que l'on obtient qu'il $\exists \Omega \in E~ /~ Si~ M\underset{\mathcal{R''}}\equiv (X,Y,Z)$, alors : 
$$M \in (\Sigma) \Leftrightarrow \lambda_1.X^2 + \lambda_2.Y^2 + \lambda_3.Z^2 = h'$$
On montre que ceci équivaut à : 
$$\begin{cases}
   \varepsilon_1.\dfrac{X^2}{a^2} + \varepsilon_2.\dfrac{Y^2}{b^2} + \varepsilon_3.\dfrac{Z^2}{c^2} = 1~ si~ h'\neq 0. \\
   \varepsilon_1.\dfrac{X^2}{a^2} + \varepsilon_2.\dfrac{Y^2}{b^2} + \varepsilon_3.\dfrac{Z^2}{c^2} = 0~ si~ h'= 0. \\
  \end{cases}
$$
Avec $\varepsilon_1 = \varepsilon_2 = \varepsilon_3 = \pm 1$. Quitte à permutter les vecteurs $\overrightarrow{u_1},\overrightarrow{u_2},\overrightarrow{u_3}$ de la base B', nous avons les possibilités suivantes :
$$(\varepsilon_1,\varepsilon_2,\varepsilon_3) = (+1,+1,+1)~ (1) $$
$$(\varepsilon_1,\varepsilon_2,\varepsilon_3) = (+1,+1,-1)~ (2) $$
$$(\varepsilon_1,\varepsilon_2,\varepsilon_3) = (+1,-1,-1)~ (3) $$
$$(\varepsilon_1,\varepsilon_2,\varepsilon_3) = (-1,-1,-1)~ (4) $$
Dans tous les cas suivants, on montre que l'on obtient les cas suivantes à partir de cas plus simple, au moyen d'affinité. 
\subparagraph{Cas (1)}
$\Sigma$ est défini par : 
$$\dfrac{X^2}{a^2} + \dfrac{Y^2}{b^2} + \dfrac{Z^2}{c^2} = 1$$
La quadrique se déduit de la sphère unité par au plus 3 affinité droites. On obtient un ellipsoïde (allongé ou aplati). $\Omega$ est un center de symétrie. Les axes de coordonnée dans $\mathcal{R}''$ en sont les axes de symétries.
\subparagraph{Cas (1')}
$(\Sigma)$ est défini par : 
$$\dfrac{X^2}{a^2} + \dfrac{Y^2}{b^2} + \dfrac{Z^2}{c^2} = 0$$
C'est une quadrique dégénéré associé à 1 point :
$$(\Sigma) = \left\lbrace \Omega \right\rbrace $$
\subparagraph{Cas (2)}
$(\Sigma)$ est défini par : 
$$\dfrac{X^2}{a^2} + \dfrac{Y^2}{b^2} - \dfrac{Z^2}{c^2} = 1$$
$(\Sigma)$ est un hyperboloide elliptique à une nappe
\subparagraph{Cas (2')}
$(\Sigma)$ est défini par : 
$$\dfrac{X^2}{a^2} + \dfrac{Y^2}{b^2} - \dfrac{Z^2}{c^2} = 0$$
$(\Sigma)$ est le cône asymptote de l'hyperbolide elliptique précédent.
\subparagraph{Cas (3)}
$(\Sigma)$ est défini par : 
$$\dfrac{X^2}{a^2} - \dfrac{Y^2}{b^2} - \dfrac{Z^2}{c^2} = 1$$
$(\Sigma)$ est un hyperbolide elliptique de révolution à deux nappes
\subparagraph{Cas (3')}
$(\Sigma)$ est défini par : 
$$\dfrac{X^2}{a^2} - \dfrac{Y^2}{b^2} - \dfrac{Z^2}{c^2} = 0$$
$(\Sigma)$ est le cone asymptote du cas précédent.
\subparagraph{Cas (4)}
$(\Sigma)$ est défini par : 
$$-\dfrac{X^2}{a^2} - \dfrac{Y^2}{b^2} - \dfrac{Z^2}{c^2} = 1$$
$(\Sigma)= \emptyset$
\subparagraph{Cas (4')}
$(\Sigma)$ est défini par : 
$$-\dfrac{X^2}{a^2} - \dfrac{Y^2}{b^2} - \dfrac{Z^2}{c^2} = 0$$
$(\Sigma) = \left\lbrace \Omega \right\rbrace $
\subsubsection{Deux valeurs propres distinct, la troisième nulle}
Quitte à réordonnée les vecteurs propres, $\overrightarrow{u_1},\overrightarrow{u_2},\overrightarrow{u_3}$ de la vase B' on peut supposé :
$$\lambda_1 \neq 0,~ \lambda_2 \neq 0,~ \lambda_3=0$$
De la même façon que précédement : 
$$\begin{cases}
   \varepsilon_1.\dfrac{X^2}{a^2} + \varepsilon_2.\dfrac{Y^2}{b^2} = Z ~ si~ \omega_0'\neq 0. \\
   \varepsilon_1.\dfrac{X^2}{a^2} + \varepsilon_2.\dfrac{Y^2}{b^2} = 1~ si~ \omega_0' = 0,~ h'\neq 0. \\
   \varepsilon_1.\dfrac{X^2}{a^2} + \varepsilon_2.\dfrac{Y^2}{b^2} = 0~ si~ \omega_0' = 0,~ h'= 0. \\
  \end{cases}
$$
Quitte à permutter $\overrightarrow{u_1},\overrightarrow{u_2}$, on peut avoir les cas suivants :
$$(\varepsilon_1,\varepsilon_2) = (+1,+1)~ (1)$$
$$(\varepsilon_1,\varepsilon_2) = (-1,-1)~ (2)$$
$$(\varepsilon_1,\varepsilon_2) = (+1,-1)~ (3)$$
\subparagraph{Cas (1)}
$(\Sigma)$ est défini par : 
$$\dfrac{X^2}{a^2} + \dfrac{Y^2}{b^2} = Z$$
$(\Sigma)$ est un paraboloide elliptique
\subparagraph{Cas (2)}
$(\Sigma)$ est défini par : 
$$-\dfrac{X^2}{a^2} - \dfrac{Y^2}{b^2} = Z$$
On peut se ramener au cas précédent en changent de repère.
\subparagraph{Cas (3)}
$(\Sigma)$ est défini par : 
$$\dfrac{X^2}{a^2} - \dfrac{Y^2}{b^2} = Z$$
$(\Sigma)$ est un paraboloide hyperbolique.\\
Dans les cas suivants, l'équation est du type : 
$$f(X,Y)=0$$
On montre que ces surfaces sont des cylindres.
\subparagraph{Cas (4)}
$(\Sigma)$ est défini par : 
$$\dfrac{X^2}{a^2} + \dfrac{Y^2}{b^2} = 0$$
$(\Sigma)$ est l'intersection des plans X=0 et Y=0. C'est à dire :
$$(\Sigma) = (\Omega.Z) = \Omega + Vect(\overrightarrow{u_3})$$
C'est une quadrique dégénéré.
\subparagraph{Cas (5)}
$(\Sigma)$ est défini par : 
$$\dfrac{X^2}{a^2} - \dfrac{Y^2}{b^2} = 0$$
$(\Sigma)$ est la réunion de deux plans parallèle à Oz.
\subparagraph{Cas (6)}
$(\Sigma)$ est défini par : 
$$-\dfrac{X^2}{a^2} - \dfrac{Y^2}{b^2} = 1$$
$(\Sigma) = \emptyset$
\subparagraph{Cas (7)}
$(\Sigma)$ est défini par : 
$$\dfrac{X^2}{a^2} - \dfrac{Y^2}{b^2} = 1$$
$(\Sigma)$ est un cylindre hyperbolique de génératrice parallèle à $\Omega z$.
\subparagraph{Cas (8)}
$(\Sigma)$ est défini par : 
$$\dfrac{X^2}{a^2} + \dfrac{Y^2}{b^2} = 1$$
$(\Sigma)$ est un cylindre hyperbolique de génératrice parallèle à $\Omega z$.
\subsubsection{Une seule valeur propre non nulle}
Quitte à réordonnée les vecteurs propres, $\overrightarrow{u_1},\overrightarrow{u_2},\overrightarrow{u_3}$ de la vase B' on peut supposé :
$$\lambda_1 \neq 0,~ \lambda_2 = 0,~ \lambda_3=0$$
\subparagraph{Cas 1}
Si $(v',w') \neq (0,0)$. Dans un certain repère, on montre que l'équation de $(\Sigma)$ est donnée par : 
$$X^2 - 2.p.Y = 0$$
$(\Sigma)$ est un cylindre parabolique de génératice paralèlle ($\Omega z$)
\subparagraph{Cas 2}
Si $(v',w') = (0,0)$, on obtient une équation du type : 
$$X^2 = h''$$
Nous avons les cas suivants :
\begin{itemize}
 \item[$\rightarrow$] h'' < 0 $\Rightarrow$ ($\Sigma$) = 0
 \item[$\rightarrow$] h'' = 0 $\Rightarrow$ ($\Sigma$) est le plan X=0
 \item[$\rightarrow$] h'' > 0 $\Rightarrow$ ($\Sigma$) est la réunion de deux plans
\end{itemize}
