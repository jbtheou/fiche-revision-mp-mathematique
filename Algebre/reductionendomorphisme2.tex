\chapter{Réduction des endomorphismes et des matrices - Deuxième Partie}
\section{Polynomes d'endomorphisme ou de matrice}
Dans ce chapitre, toutes les relations vu sont transposable aux matrices
\begin{de}
Soit f une application linéaire de E dans E, avec E un K espace vectoriel.\\
Soit P $\in K[X]$ défini par : 
$$P = a_0 + a_1.X+\dots+a_p.X^p$$
Avec $\forall i~ a_i \in K$. On défini :
$$P(f) = a_0.Id + a_1.f +\dots+a_p.f^p$$
Avec : 
\begin{itemize}
 \item[$\rightarrow$] $fofo\dots of = f^k$
 \item[$\rightarrow$] $f^0 = Id$
\end{itemize}
De meme, si A $\in \mathcal{M}_n(K)$ : 
$$P(A) = a_0.I_n + a_1.A + a_p.A^p$$
\end{de}
\begin{prop}
Soit B est une base de E, avec E un K espace vectoriel de dimension fini, f$\in \mathcal{L}(E)$ et $A = mat_B(f)$. On obtient : 
$$P(A) = mat_B(P(f))$$
\end{prop}
\begin{prop}
Soit $P_1$ et $P_2 \in K[X]$, $\lambda \in K$, $f\in \mathcal{L}(E)$.\\
Nous avons les résultats suivants :
$$(P_1+P_2)(f) = P_1(f)+P_2(f)$$
$$(\lambda P_1)(f) = \lambda P_1(f)$$
$$(P_1.P_2)(f) = P_1(f) o P_2(f)$$
De ce dernier résultat, on obtient que : 
$$P_1(f)oP_2(f) = P_2(f)oP_1(f)$$
\end{prop}
\begin{de}
Soit f un endomorphisme de E, avec E un K espace vectoriel et :
$$K[f] = \left\lbrace P(f), P\in K[X] \right\rbrace $$
Soit $A \in \mathcal{M}_n(K)$.\\
On note : 
$$K[A] = \left\lbrace P(A), P \in K[X] \right\rbrace $$
\end{de}
\begin{prop}
Les espaces défini ci dessus sont des sous algèbres commuative de respectivement $(\mathcal{L}(E),+,\lambda.,o)$ et $(\mathcal{M}_n(K),+,\lambda.,x)$
\end{prop}
\section{Idéaux de K[X]}
\begin{de}
On appelle idéale de K[X] toute partie non vide $\mathcal{I}$ de K[X] tq : 
\begin{itemize}
 \item[$\rightarrow$] $\mathcal{I}$ est stable par +
 \item[$\rightarrow$] $\forall P \in \mathcal{I}$ et $\forall Q \in K[X]$, PQ $\in \mathcal{I}$
\end{itemize}
De cette définition, on obtient que : 
$$P \in \mathcal{I} \Rightarrow -P \in \mathcal{I}$$
$$0 \in \mathcal{I}$$
Avec ici 0, le polynome constant nul.
\end{de}
\subsection{Exemple}
Nous avons les ensembles suivants, qui sont des idéaux triviaux : 
$$\mathcal{I} = \left\lbrace 0 \right\rbrace $$
Celui ci constitue l'idéal nul.
$$K[X] = \mathcal{I}$$
\begin{de}
Soit P un polynome de K[X]. On défini l'idéal engendré par P, noté [P], par : 
$$[P] = \left\lbrace PQ, Q \in K[X]\right\rbrace $$
C'est donc l'ensemble constitué des multiples de P.
\end{de}
\subsection{Définitions et théorème}
\begin{de}
Un idéal engendrée par un seul polynome, du type [P], est appelé idéal principale
\end{de}
\begin{theo}
Tout idéal de K[X] est principale. On dit donc que l'anneau K[X] est principale. 
\end{theo}
\begin{de}
Soit $\mathcal{I}$, un idéal de K[X], donc un idéal idéal.\\
On appelle générateurs de $\mathcal{I}$ les polynomes $\omega$ telque : 
$$\mathcal{I} = [\omega]$$
\end{de}
\begin{prop}
Les générateurs $\omega$ se déduisent les uns des autres par multiplication par une constante non nulle $\lambda \in K^*$.\\
De plus, si $\mathcal{I} \neq \left\lbrace 0 \right\rbrace $, alors les générateurs ont tous le même degrés : 
$$deg(\omega) = min\left\lbrace deg(P), P \in \mathcal{I}-\left\lbrace  0\right\rbrace  \right\rbrace  $$
\end{prop}
\begin{prop}
De la propriété précédente, on déduit que si : 
\begin{itemize}
 \item[$\rightarrow$] $\omega \in \mathcal{I}$
 \item[$\rightarrow$] deg($\omega) min\left\lbrace deg(P), P \in \mathcal{I}-\left\lbrace  0 \right\rbrace\right\rbrace $
Alors : 
$$\mathcal{I} = [\omega]$$ 
\end{itemize}
\end{prop}
\begin{de}
L'unique générateur unitaire d'un idéal $\mathcal{I}$ non nul est appelé polynome minmale de l'idéal $\mathcal{I}$
\end{de}
\subsection{Application au pgcd de deux polynomes, et à l'algorithme d'Euclide}
Soit $P_1$ et $P_2$ deux polynomes de K[X] non tous les deux nuls.\\
On sait que [$P_1,P_2$] est un idéal de K[X], donc un idéal principale.\\
On obtient donc qu'il existe un unique $\omega \in K[X] - \left\lbrace 0 \right\rbrace $ telque : 
$$[P_1,P_2] = [\omega]$$
On montre que $\omega$ est le pgcd de $P_1$ et $P_2$
\subsubsection{Algorithme d'Euclide}
Cet algoritme se base sur la propriété suivante : 
$$\forall Q \in K[X] ~ [P_1,P_2] = [P_1,P_2 + Q.P_1]$$
\subsection{Polynome annulateur d'un endomorphisme ou d'une matrice, Polynome minimal d'un endomorphisme ou d'une matrice}
\begin{de}
Soit $f \in \mathcal{L}(E)$, avec E un K espace vectoriel.\\
On dit que $P \in K[X]$ annule f, ou que P est un polynome annulateur de f, ou encore que f annule P si : 
$$P(f) = \tilde{0}$$
De meme, si $A \in \mathcal{M}_n(K)$, on dit que P annule A si :
$$P(A) = \begin{pmatrix}
          0 & \dots & 0 \\
          \vdots & \ddots & \vdots \\
          0 & \dots & 0
         \end{pmatrix}
$$
\end{de}
\begin{prop}
L'ensemble $A_{nn}(f)$ (Notation non standard), des polynomes annulateur de f, défini par :
$$A_{nn}(f) = \left\lbrace P \in K[X]~ tq~ P(f) = \tilde{0}\right\rbrace $$
Cet ensemble est un idéal de K[X].
\end{prop}
\begin{de}
On appelle polynome minimale de f, noté $\omega_f$, l'unique polynome unitaire telque que : 
$$A_{nn}(f) = [\omega]$$
On défini de même le polynome minimal d'une matrice
\end{de}
\begin{prop}
Soit $f \in \mathcal{L}(E)$, avec E un K espace vectoriel de dimension finie, B une base de E et A = $mat_{B}(f)$.\\
On obtient dans ce cas que : 
$$\omega_A = \omega_f$$
\end{prop}
\begin{prop}
Soit E une K espace vectoriel de dimension n.\\
On obtient que, $\forall f \in \mathcal{L}(E)$ : 
$$A_{nn} \neq \left\lbrace O\right\rbrace $$
Donc que :
$$\omega_f \neq 0$$
\end{prop}
\begin{prop}
Soit $f \in \mathcal{L}(E)$, avec E un K espace vectoriel.\\
Si $\omega_f \neq 0$ et d = deg($\omega_f$), alors (Id,f,...,$f^{d_1}$) est une base de K[f], en particulier : 
$$dim K[f] = deg(\omega_f)$$
\end{prop}
\subsection{Théorème de Cayley-Hamilton}
\begin{theo}
Soit $f \in \mathcal{L}(E)$, avec E un K espace vectoriel de dimension fini. On obtient alors que :
$$P_f(f) = \tilde{0}$$
Nous avons les équivalences suivantes : 
$$(P_f(f) = \tilde{0}) \Leftrightarrow (P_f \in A_{nn}(f) )\Leftrightarrow (\omega_f | P_f)$$
\end{theo}
\begin{corr}
 Si E est une K espace vectoriel de dimension n, et $f \in \mathcal{L}(E)$, alors deg $w_f \leq n$
\end{corr}
\subsection{Relation entre valeurs propres et racines des polynomes annulateurs}
\begin{prop}
Soit $f \in \mathcal{L}(E)$, avec E un K espace vectoriel ( E peut être un espace de dimension infini).\\
Si f annule $P \in K[X]$, alors :
$$S_p(f) c Z(P)$$
Avec Z(P) l'ensemble des zéros de P, c'est à dire l'ensemble des racines de P.
\end{prop}
\begin{lemme}
Si $f(\overrightarrow{x}) = \lambda.\overrightarrow{x}$, alors :
$$\forall P \in K[X]~ P(f)(\overrightarrow{x}) = P(\lambda)(\overrightarrow{x})$$
\end{lemme}
\begin{prop}
Si $f \in \mathcal{L}(E)$, avec E un K espace vectoriel de dimension finies, alors :
$$S_p(f) = Z(w_f)$$
Cependant, ceci ne nous donne bien évidement aucune information sur la multiplicité des racines.
\end{prop}
\begin{prop}
Soit f et g deux endomorphisme de E dans E. Si : 
$$fog = gof$$
C'est à dire, si les deux endomorphimes communent, alors Ker(g) et Im(g) sont stable par f
\end{prop}
\section{Lemme des noyaux}
\begin{prop}
Soit $f \in \mathcal{L}(E)$, avec E un K espace vectoriel. Soit $P_1$ et $P_2$ deux polynomes de $K[X]$, premiers entre eux. On obtient alors : 
$$Ker(P_1.P_2)(f) = Ker(P_1)(f) \oplus Ker(P_2)(f)$$
\end{prop}
\begin{gene}
Soit $f \in \mathcal{L}(E)$, avec E un K espace vectoriel. Soit $P_1,\dots,P_k$ des polynomes de K[X] deux à deux premiers entre eux. Soit P = $P_1\dots P_k$, on obtient alors : 
$$Ker(P)(f) = Ker(P_1)(f) \oplus \dots \oplus Ker(P_k)(f)$$
\end{gene}
\subsection{Application fondamentale de Lemme des noyaux}
Soit $f \in \mathcal{L}(E)$, avec E un K espace de dimension n telque $P_f$ soit scindé sur K[X]. On peut donc écrire $P_f$ sous la forme : 
$$P_f(X) = (-1)^n(X-\lambda_1)^{n_1}\dots(X-\lambda_p)^{n_p}$$
En utilisant conjointement le lemme des noyaux et le théorème de Cayley-Hamilton, on obtient que : 
$$P_f(f) = (-1)^n.(f-\lambda_1.Id)^{n_1}o\dots o (f-\lambda_p.Id)^{n_p} = \tilde{0}$$
Grâce au lemme des noyaux, on obtient que :
$$E = E_1 \oplus \dots \oplus E_p$$
Avec :
$$E_k = Ker(f-\lambda_k.Id)^{n_k}$$
On peut obtenir à partir de tout ceci une matrice diagonale par bloc, de la forme : 
\[mat_{B}(f) = \begin{blockarray}{cccccccccc}
        | \leftarrow & -\overset{n_1}-- &\rightarrow | & | \leftarrow & -\overset{n_2}-- &\rightarrow|& \dots  & |\leftarrow & -\overset{n_p}-- & \rightarrow| \\
        \begin{block}{(cccccccccc)}
           \lambda_1 &  &  & & & & & & & (0) \\
   	   & \ddots & & & & & & & \\
  	   &  & \lambda_1 & &  & & & & &\\
  	   &  &  & \lambda_2 & & & & & & \\
  	   &  &  &  & \ddots & & & & &\\
 	   &  &  &  & & \lambda_2 & & & & \\
 	   &  &  &  & &  & \ddots & & &\\
 	   &  &  &  & &  &  & \lambda_p & &\\
 	   &  &  &  & &  & & & \ddots &\\
  	  (0) &  &  &  & &  &  & & & \lambda_p\\
        \end{block}
        \end{blockarray}
\]
\section{Endomorphismes et matrices nilpotants}
\begin{de}
Soit $f \in \mathcal{L}(E)$.\\
f est dit nilpotant si il existe $p \in N^*$ telque : 
$$f^p = \tilde{0}$$
La définition est analogue pour les matrices
\end{de}
\begin{prop}
Soit E un C espace vectoriel de dimension n et $f \in \mathcal{L}(E)$, alors les conditions suivantes sont équivalentes :
\begin{itemize}
 \item[$\rightarrow$] f est nilpotant
 \item[$\rightarrow$] $S_p(f) = \left\lbrace 0 \right\rbrace $
 \item[$\rightarrow$] $\exists$ B base de E telque $mat_B(f)$ soit une matrice stricement triangulaire, c'est à dire triangulaire avec tous ces termes diagonaux nuls.
\end{itemize}
\end{prop}

\begin{de}
On appelle indice de nilpotence de f le plus petit entier $\nu$ telque : 
$$f^{\nu} = \tilde{0}$$
Si E est un C espace vectoriel de dimension n, on obtient que : 
$$\nu \geq n$$
Plus précisement, on obtient que : 
$$w_f = X^{\nu}$$
\end{de}
\section{Nouveaux critères de trigonabilité}
\begin{theo}
Soit $f \in \mathcal{L}(E)$, avec E un K espace vectoriel de dimension n. Les conditions suivantes sont équivalentes : 
\begin{itemize}
 \item[$\rightarrow$] f est trigonalisable
 \item[$\rightarrow$] f annule un polynome scindé sur K
 \item[$\rightarrow$] $w_f$ est scindé sur K
\end{itemize}
\end{theo}
\subsection{Réduction de Dunford}
Si :
$$w_f = (X - \lambda_1)^{m_1}...(X - \lambda_p)^{m_p}$$
avec les $\lambda_k$ deux à deux distinct, alors on obtient que, d'après le lemme de noyaux : 
$$E = E_1 \oplus ... \oplus E_P$$
avec : 
$$E_k = Ker(f + \lambda_k.id)^{m_k}$$
On peut donc choisir une base de chaqu'un des $E_k$, notée $B_k$, telque :
$$mat_{B_k}f_{\parallel E_k} = \begin{pmatrix}
          \lambda_k &  & (a_{ij}) \\
	   & \ddots & \\
          (0) &  & \lambda_k
         \end{pmatrix}$$
C'est donc une matrice triangulaire superieur. De plus, on a : 
$$B = B_1 \vee \dots \vee B_P$$
qui est une base de E. On obtient donc que : 
\[mat_B f = \begin{blockarray}{cccc}
        \overset{n'_1}\leftrightarrow & \dots & \overset{n'_p}\leftrightarrow  \\
        \begin{block}{(ccc)c}
        \mesbloc{A_{1}}  &  & (0) & \updownarrow n'_1   \\
          & \ddots &  &       \\
        (0) &   & \mesbloc{A_{p}} & \updownarrow n'_p      \\
        \end{block}
        \end{blockarray}
\] 
Avec $A_k = mat_{B_k}f_{\parallel E_k}$. On montre de plus qu'en réalité : 
\begin{itemize}
 \item[$\rightarrow$] $n'_k$ est en réalité egale à $mult_{Pf}(\lambda_k)$
 \item[$\rightarrow$] $E_k$ = Ker(f-$\lambda$.id$)^{n_k}$
\end{itemize}
\section{Nouveau critère de diagonabilité}
\begin{theo}
Soit $f \in \mathcal{L}(E)$, avec E un K espace vectoriel de dimension finies. Les propositions suivantes sont équivalente : 
\begin{itemize}
 \item[$\rightarrow$] f est diagonalisable
 \item[$\rightarrow$] f annule un polynome scindé sur K à racine simple
 \item[$\rightarrow$] $w_f$ est scindé sur K à racine simple
\end{itemize}
De plus, nous savons que : 
$$Z(w_f) = S_p(f)$$
La première proposition s'écrit donc : 
$$w_f(X) = (X-\lambda_1)...(X-\lambda_p)$$
ou 
$$S_p(f) = \left\lbrace \lambda_1,...,\lambda_p\right\rbrace $$
\end{theo}

