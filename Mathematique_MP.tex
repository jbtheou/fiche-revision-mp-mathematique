% % % % % % % % % % Tout ce qui est mis derrière un « % » n'est pas vu par LaTeX
% On appelle cela des « commentaires ».  Les commentaires permettent de
% commenter son document - comme ce que je suis en train de faire
% actuellement - et de cacher du code - cf. la ligne \pagestyle.

\documentclass[a4paper, titlepage, draft,twoside]{book}

% ************************
% * fichier de préambule *
% ************************
 
% ***** extensions *****
\def\renamesymbol#1#2{
  \expandafter\let\expandafter\newsym\expandafter=\csname#2\endcsname
  \expandafter\global\expandafter\let\csname#1#2\endcsname=\newsym
  \expandafter\global\expandafter\let\csname#2\endcsname=\origsym
}

\usepackage[utf8x]{inputenc}			  % Utilisation du UTF8
\usepackage{textcomp}				  % Accents dans les titres
\usepackage [ french ] {babel}                    % Titres en français
\usepackage [T1] {fontenc} 			  % Correspondance clavier -> document
\usepackage{wasysym}
\renamesymbol{wasysym}{iint}
\renamesymbol{wasysym}{iiint}
\usepackage[Lenny]{fncychap}                      % Beau Chapitre
\usepackage{dsfont}                    	  % Pour afficher N,Z,D,Q,R,C
\usepackage{fancyhdr}                             % Entete et pied de pages
\usepackage [outerbars] {changebar}               % Positionnement barre en marge externe
\usepackage{amsmath}				  % Utilisation de la librairie de Maths
\usepackage{amssymb}
%\usepackage{amsfont}				  % Utilisation des polices de Maths
\usepackage{enumerate}				  % Permet d'utiliser la fonction énumerate
\usepackage{dsfont}				  % Utilisation des polices Dsfont
%\usepackage{ae}					  % Rend le PDF plus lisible
\usepackage[pdftex]{graphicx}                 % dernière étant la langue principale
\usepackage{color}
%\usepackage{palatino}
\usepackage{paralist}
\usepackage{natbib}
\usepackage[margin=3cm]{geometry}
\usepackage{fancyhdr}
\usepackage[Lenny]{fncychap}
\usepackage{graphicx}
\usepackage{wrapfig}
\usepackage{url}
\usepackage{graphics}
\usepackage{blkarray} 


\definecolor{Dark}{gray}{.2}
\definecolor{Medium}{gray}{.6}
\definecolor{Light}{gray}{.8}
\newcommand*{\plogo}{\fbox{$\mathcal{PL}$}}


\newtheorem{de}{Définition}
\newtheorem{theo}{Théorème}
\newtheorem{prop}{Propriété}
\newtheorem{princ}{Principe}
\newtheorem{conv}{Convention}
\newtheorem{loi}{Loi}
\newtheorem{voc}{Vocabulaire}
\newtheorem{enon}{\'Enonc\'e}
\newtheorem{nota}{Nota}
\newtheorem{propfond}{Propriété fondamentale}
\newtheorem{lemme}{Lemme}
\newtheorem{corro}{Corollaire}
\newtheorem{corr}{Corollaire}
\newtheorem{coro}{Corollaire}

\newtheorem{hypo}{Hypothèses générales}
\newtheorem{gene}{Généralisation}

\newcommand{\mesbloc}[1]{\begin{array}{|c|} \hline #1 \\ \hline \end{array}}

\makeatletter
\newlength\BA@height
\def\BA@leftdel#1#2#3{%
  \setlength\BA@height{\dimen\z@}%
  \addtolength\BA@height{5pt}%
  \llap{%
    {#1}$\left#2\vrule height \BA@height width\z@ \right.$\kern-#3}}

\def\BA@rightdel#1#2#3{%
  \setlength\BA@height{\dimen\z@}%
  \addtolength\BA@height{5pt}%
  \rlap{%
    \kern-#3$\left.\vrule height \BA@height width\z@ \right#1${#2}}}%
\makeatother 

\newlength{\drop}


\newcommand*{\titleGMPHY}{\begingroup% Gentle Madness
\setlength{\drop}{0.1\textheight}
%\vspace*{\baselineskip}
\vfill
  \hbox{%
  \hspace*{0.2\textwidth}%
  \rule{1pt}{\textheight}
  \hspace*{0.05\textwidth}%
  \parbox[b]{0.75\textwidth}{
  \vbox{%
    %\vspace{\drop}
    {\noindent\Huge\bfseries Fiches de Révision\\[0.5\baselineskip]
               MP}\\[2\baselineskip]
    {\Large\itshape TOME II - Mathématique}\\[4\baselineskip]
    {\Large Jean-Baptiste Théou}\par
    \vspace{0.5\textheight}
    {\noindent Creactive Commons - Version 0.1}\\[\baselineskip]
    }% end of vbox
    }% end of parbox
  }% end of hbox

\null
\endgroup}

\newcommand*{\titleGMMATH}{\begingroup% Gentle Madness
\setlength{\drop}{0.1\textheight}
%\vspace*{\baselineskip}
\vfill
  \hbox{%
  \hspace*{0.2\textwidth}%
  \rule{1pt}{\textheight}
  \hspace*{0.05\textwidth}%
  \parbox[b]{0.75\textwidth}{
  \vbox{%
    %\vspace{\drop}
    {\noindent\Huge\bfseries Fiches de Révision\\[0.5\baselineskip]
               MPSI}\\[2\baselineskip]
    {\Large\itshape TOME II - Mathématiques}\\[4\baselineskip]
    {\Large Jean-Baptiste Théou}\par
    \vspace{0.5\textheight}
    {\noindent Creactive Commons}\\[\baselineskip]
    }% end of vbox
    }% end of parbox
  }% end of hbox

\null
\endgroup}


\begin{document}

\pagestyle{empty}
\titleGMPHY
\clearpage
\frontmatter                  % Prologue.
\chapter{Licence}
J'ai décidé d'éditer cet ouvrage sous la licence Créative Commons suivante : CC-by-nc-sa. Pour plus d'information :\\
http://creativecommons.org/licenses/by-nc-sa/2.0/fr/.\\
Ce type de licence vous offre une grande liberté, tout en permettant de protéger mon travail contre une utilisation commercial à mon insu par exemple.\\
Pour plus d'information sur vos droits, consultez le site de Créative Commons
\chapter{Avant-propos}
Il y a un plus d'un an, au milieu de ma SUP MP, j'ai décidé de faire mes fiches de révision à l'aide de Latex, un "traitement de texte" très puissant. Il en résulte les fiches qui suivent. Je pense que travailler sur des fiches de révision, totalement séparé de notre cours, est un énorme plus, et réduit grandement la quantité de travail pour apprendre son cours, ce qui laisse plus de temps pour les exercices. Mon experience en tout cas va dans ce sens, j'ai notablement progressé à l'aide de ces fiches.\\
J'ai décidé de les rassembler sous forme d'un "livre", ou plutôt sous forme d'un recueil. Ce livre à pour principal interet pour moi d'être transportable en cours. C'est cet interet qui m'a poussé à faire ce livre.\\
Dans la philosophie de mes fiches de révision, ce livre est disponible gratuitement et librement sur mon blog. Il est édité sous License Créative Commons. Vous pouvez librement adapter ce libre à vos besoins, les sources Latex sont disponibles sur mon blog. Je pense que pour être en accord avec la philosophie de ces fiches, il serai bien que si vous effectuez des modifications de mon ouvrage, vous rendiez ces modifications disponible à tous. Je laisserai volontiers une place pour vos modifications sur mon blog. Je pense sincèrement que ce serai vraiment profitable au plus grand nombre, et dans la logique de mon travail.\\
J'ai hiérarchisé mon ouvrage de façon chronologique. Les parties sont rangées dans l'ordre "d'apparition" en MP, tout en conservant une certaine logique dans les parties. J'ai mis en Annexe des petites fiches de méthodologie, qui peuvent s'avérer utiles.\\
Je vous souhaite une bonne lecture, et surtout une bonne réussite.\\
Jean-Baptiste Théou
\chapter{Remerciements}
Je tient à remercier Georges Marin, Professeur de Physique-Chimie en MP au Lycée Lesage et François Brunou, Professeur de Mathématiques en MP au Lycée Lesage.\\
Sans eux, ce livre ne pourrai exister.\\
\mainmatter                   % On passe aux choses serieuses
\part{Révisions}
\chapter{Rappels et Compléments}
\section{Relations de comparaison}
\subsection{Relations d'équivalence}
Soit R une relation.
\begin{enon}
R est une relation d'equivalence sur E si et seulement si :
\begin{enumerate}[I) ]
 \item R est réflexive : $\forall x \in E,~ xRx$
 \item R est symétrique : $\forall (x,y) \in E^2~ tq~ xRy,~ on~ as~ :~ yRx$
 \item R est transitive : $\forall (x,y,z) \in E^3~ tq~ xRy~ et~ yRz,~ on~ as~ :~ xRz$
\end{enumerate}
\end{enon}
\subsection{Fonction module}
\begin{enon}
La fonction module, fonction de $\left[a,b\right]$ dans $\Re$, est une fonction continue
\end{enon}
\subsection{Voisinage fondamental}
\begin{de}
On défini un voisinage fondamental de $x_0 \in \bar{\Re}$ par : 
\begin{itemize}
 \item[$\rightarrow$] Si $x_0 \in \Re$ : V = ]$x_0-r,x_0+r$[
 \item[$\rightarrow$] Si $x_0 = + \infty$ : V= ]a,$\infty$[
 \item[$\rightarrow$] Si $x_0 = - \infty$ : V= ]$-\infty$,a[
\end{itemize}
\end{de}
\subsection{Négligabilité}
\begin{enon}
Soient u et v deux fonctions de $\Re$ dans K, définies sur un même voisinage de 0.\\
Par exemple, définies sur ]-r,r[, avec r > 0.\\
On dit que u est négligable devant v en 0 si et seulement si, $\exists r' \in ]0,r[$ et h, une fonction définie par : 
$$h : \left] -r',r' \right[  \rightarrow K$$
avec $\lim\limits_{0} h = 0$ telque : 
$$\forall x \in \left]-r',r'\right[~ u(x) = v(x).h(x)$$
On le note $$u \underset{0}= o(v)$$
\end{enon}
\begin{prop}
Soit u fonction de V dans K, avec V voisinage fondamental de $x_0 \in \bar{\Re}$.\\
Si $\lambda$ est une constante de $K^*$, indépendante de la variable x, alors :
$$o(\lambda.u) \underset{x_0}= o(u) $$
\end{prop}
\begin{prop}
Soient $o_1(u),o_2(u),...,o_p(u)$ fonctions négligable devant u.\\
Si p ne dépend pas de la variable x :
$$o_1(u)+o_2(u)+...+o_p(u) = o(u)$$
\end{prop}
\subsubsection{Notation d'Hardy et de Landau}
\begin{enon}
Soient u et v deux fonctions de $\Re$ dans K, définies sur un voisinage $\left]-r,r\right[$, avec r>0, en 0.
$$u \underset{0}\ll v \Leftrightarrow u \underset{0}= o(v)$$
La première notation est la notation d'Hardy. La seconde est celle de Landau.
\end{enon}
\subsection{Équivalence}
\begin{enon}
Soient u et v deux fonctions de $\Re$ dans K, définies sur un voisinage $\left]-r,r \right[ $, avec r>0, de 0.
On dit que u est équivalent à v en 0 si et seulement si, $\exists r' \in ]0,r[$ et h, une fonction définie par : 
$$h : \left] -r',r' \right[  \rightarrow K$$
avec $\lim\limits_{0} h = 1$ telque : 
$$\forall x \in \left]-r',r'\right[~ u(x) = v(x).h(x)$$
On le note :
$$u \underset{0}\sim v$$
\end{enon}
\begin{prop}
Soient u et v deux fonctions équivalente en $x_0$.\\
Nous avons, si $\alpha$ est indépendant de la variable :
$$u \underset{x_0}\sim v \Rightarrow u^{\alpha} \underset{x_0}\sim v^{\alpha}$$
\end{prop}
\begin{prop}
Avec les conditions précédentes, nous avons :
$$u \underset{x_0}\sim v \Leftrightarrow u \underset{x_0}= v + o(v)$$
$$u \underset{x_0}\sim v \Leftrightarrow u-v \underset{x_0}\ll v$$
Par symétrie, on peut inverser ces relations.
\end{prop}
\begin{prop}
Soient u et v deux applications de V dans K, définies sur un voisinage fondamental V de $x_0\in \bar{\Re}$.\\
Si $u \underset{x_0}\sim v$, et $u(x) \underset{x_0}\rightarrow l \in C~ ou~ l \in \bar{\Re}$, alors :
$$v(x) \underset{x_0}\rightarrow l$$
\end{prop}
\begin{prop}
Avec les conditions précédentes :
Si $u(x) \underset{x_0}\sim u_1(x)$ et $v(x) \underset{x_0}\sim v_1(x)$, alors : 
$$u(x).v(x) \underset{x_0}\sim u_1(x).v_1(x)$$
$$\dfrac{u}{v} \underset{x_0}\sim \dfrac{u_1}{v_1}$$
\end{prop}
\begin{prop}
Si $u(x) \underset{x_0}\sim v(x)$ et si u et v restent > 0 au voisinage de $x_0$ et si $u(x)$ et $v(x)$ tendent vers l$\in \bar{\Re}_+ -\left\lbrace1\right\rbrace $ en $x_0$, alors :
$$ln(u) \underset{x_0}\sim ln(v)$$
\end{prop}
\subsection{Négligabilité et équivalence}
\begin{prop}
Soient u et v deux fonctions de V dans K, avec V un voisinage fondamental de $x_0 \in \bar{\Re}$.\\
Alors :
$$u \underset{x_0}\sim v \Rightarrow o(u) \underset{x_0}= o(v)$$
\end{prop}

\begin{prop}
Si $u_1,u_2,...,u_p$ sont des fonctions de $\Re$ dans K, définies sur un voisinage $\left]-r,r\right[$, avec r>0, de 0.
Si $u_1 \underset{0}\gg u_2 \underset{0}\gg ... \underset{0}\gg u_p$, alors : 
$$u_1+...+u_p \underset{0}\sim u_1$$
\end{prop}
\subsection{Lien entre limite et somme}
\begin{prop}
Soient $h_2,h_3,...,h_p$ fonctions telque :
$$\forall k \in \left\lbrace2,..,p\right\rbrace~ \lim_0 h_k = 0$$
La conséquence suivante est vraie uniquement si p est indépendant de x :
$$\lim_0 h_2+...+h_p = 0$$
\end{prop}
\subsection{Signe et équivalent}
\begin{prop}
Soient u et v deux fonctions de $\Re$ dans $\Re$ définies sur un voisinage de 0, telque :
$$u \underset{0}\sim v$$
alors, $\exists \alpha > 0$ telque $\forall x \in \left]-\alpha,\alpha\right[ $, $u(x)$ et $v(x)$ ont même signe et même points d'annulation.\\
"Un équivalent contrôle localement le signe"
\end{prop}
\subsection{Domination - Grand O}
\begin{de}
Soient u et v deux fonctions de $\Re$ dans K, définies sur un voisinage $\left]-r,r \right[ $, avec r>0, de 0.\\
On dit que u est dominé par v si et seulement si, $\exists r' \in ]0,r[$ et h, une fonction définie par : 
$$h : \left] -r',r' \right[  \rightarrow K$$
avec h bornée sur ]-r',r'[ telque : 
$$\forall x \in \left]-r',r'\right[~ u(x) = v(x).h(x)$$
On le note :
$$u \underset{0}= O(v)$$
\end{de}
\subsection{Dans le cas des suites}
\subsubsection{Domination - Grand O}
\begin{de}
Soient $(u_n)$ et $(v_n)$ deux suites à valeurs dans K.
$$u_n = O(v_n) \Leftrightarrow (\exists n_0 \in N, \exists (h_n)_{n \geq n_0}~ tq~ \forall n \geq n_0~ u_n=v_n.h_n)$$
avec $(h_n)_{n \geq n_0}$ suite bornée.
\end{de}
\section{Fonctions}
\subsection{Fonctions continue sur un segments}
Soit f fonction continue de $\left[a,b\right] $ dans $\Re$.
\begin{prop}
f est bornée sur $\left[a,b\right]$ :
$$\exists M \in \Re^+~ tq~ \forall x \in \left[a,b\right]~ |f(x)| \leq M$$
\end{prop}
\begin{prop}
f est majorée et minorée, et atteint son Sup et son Inf en des points de $\left[a,b\right]$ : 
$$\exists \alpha \in \left[a,b\right]~ tq~ Sup_{\left[a,b\right]} f = f(\alpha)$$
\end{prop}
\begin{prop}
f est uniformement continue sur [a,b]
\end{prop}
\subsection{Fonctions continue par morceaux sur un intervalle}
\begin{prop}
Si f est continue par morceau sur un intervalle I, il en est de même pour |f|.
\end{prop}
\begin{prop}
L'ensemble des application d'un intervalle I, à valeur dans K, continue par morceaux sur I, est une algèbre.\\
C'est à dire que cet espace est stable par addition, produit par un scalaire, et par produit. Mais cet ensemble n'est pas stable par composition.
\end{prop}
\subsection{Théorème des accroissements finis}
La notation $f \in C^n(\left[a,b\right], \Re)$ signifie que f est de classe de $C^n$ de $\left[a,b\right]$ dans $\Re$
\begin{de}
Soit $f \in C^1(\left[a,b\right], \Re)$, alors : 
$$\exists c \in (\left]a,b\right[~ tq~ f(b)-f(a) = f'(c)(b-a)$$
\end{de}
\subsection{Inégalité des accroissement finis}
\begin{de}
Soit $f \in C^1(\left[a,b\right], K)$.\\
Si f' est bornée sur $\left[a,b\right]$, alors : 
$$|f(b) -f(a)| \leq \underset{\left[a,b\right]}Sup |f'|.(b-a)$$
\end{de}
On démontre le lien entre le signe de la dérivée et les variations de la fonction à l'aide de cette inégalité.
\subsection{Théorème de Rolles}
\begin{enon}
Soit f application continue sur [a,b] et dérivable sur ]a,b[.\\
Si f(a) = f(b), alors : 
$$\exists c \in \left]a,b\right[~ tq~ f'(c) = 0 $$
\end{enon}
\subsection{Théorème des valeurs intermédiaire}
Les trois propriétés suivantes sont équivalentes.
\begin{prop}
Soit f application continue de I dans $\Re$, avec I intervalle inclus dans $\Re$.\\
Alors f(I) est aussi un intervalle
\end{prop}
\begin{prop}
Soit f application de A dans $\Re$, définie et continue sur [a,b].\\
Si :
$$f(a).f(b)\leq 0$$
Alors :
$$\exists c \in \left[a,b\right]~ tq~ f(c)=0$$
\end{prop}
\begin{prop}
Soit f application continue de I dans $\Re$.\\
$\forall (x,x') \in I^2$, tous $y$ compris entre f($x$) et f($x'$) est une valeur de f sur [$x,x'$]
\end{prop}
\subsubsection{Cas particuliers}
\begin{prop}
Soit f, fonction de $\Re$ dans $\Re$, continue sur [a,b].\\
Alors f([a,b]) = [m,M], avec : 
$$m = Inf_{\left[a,b\right]}f $$
$$M = Sup_{\left[a,b\right]}f $$
\end{prop}
\begin{prop}
Soit f fonction de $\Re$ dans $\Re$, continue et strictement croissante sur un intervalle I.\\
Alors f induit une bijection de I sur f(I), qui est lui même un intervalle, et sa bijection réciproque, $f^{-1}$, de f(I) dans I, est également continue et strictement croissante.\\
On peut préciser f(I). Quand I possède une borne ouverte, on fait appelle à la limite de f en la valeur de cette borne, et quand I possède une borne fermée, on prend la valeur de f en cette borne. Par exemple : 
$$I = \left]a,b\right] \rightarrow f(I) = \left]\lim_a f,f(b) \right] $$
\end{prop}
\begin{prop}
Soit f fonction de $\Re$ dans $\Re$, croissante sur I.\\
Alors : 
\begin{itemize}
 \item[$\rightarrow$] Soit f n'est pas majorée sur I, alors dans ce cas :  $$f(x) \underset{x \rightarrow\infty}\rightarrow \infty$$
 \item[$\rightarrow$] Soit f est majorée sur I, alors dans ce cas : $$f(x) \underset{x \rightarrow\infty}\rightarrow Sup\underset{I} f$$
\end{itemize}
\end{prop}
\subsection{Lien entre limite et bornée}
\begin{prop}
Soit f fonction de $\Re$ dans K, définie sur un voisinage de $x_0 \in \bar{\Re}$.\\
Si f a une limite finie quand x tend vers $x_0$, alors il existe un voisinage de $x_0$ V telque f soit bornée sur V. 
\end{prop}
\subsection{Étude de Arctan}
Dans le cas d'une étude asymptotique de arctan au voisinage de $\infty$, la propriété suivant peut etre utile : 
\begin{prop}
$$\forall u \in \Re^+ ~ arctan(u)+arctan(\dfrac{1}{u}) = \varepsilon.\dfrac{\pi}{2}$$
avec $\varepsilon$ = +1 si u est positif, -1 si u est négatif
\end{prop}
\subsection{Limite d'une fonction}
\begin{prop}
Soit f une fonction complexe.
$$(\mbox{ f est continue par morceaux }) \Leftrightarrow (\mbox{ Re(f) et Im(f) sont continue par morceaux })$$
\end{prop}
\begin{prop}
Soit f fonction complexe, décomposable en $f = f_1 + if_2$. Soit $(l_1,l_2) \in \Re^2$
$$(\lim_{\infty} f = l=l_1+i l_2) \Leftrightarrow (\lim_{\infty} f_1 = l_1~ et~ \lim_{\infty} f_2 = l_2)$$
\end{prop}
\subsection{Injectivité}
\begin{de}
Soit f application de A dans A'.\\
f est injective si et seulement si :
$$\forall(x,x')\in A~ f(x)=f(x') \Rightarrow x=x'$$
\end{de}
\begin{prop}
Soit f application linéaire entre deux espaces vectoriels.
$$(\mbox{ f est injective }) \Leftrightarrow Ker(f) = \left\lbrace 0 \right\rbrace $$
\end{prop}
\subsection{Surjectivité}
\begin{de}
Soit f application de A dans A'.\\
f est surjective si et seulement si :
$$\forall y \in A'~ \exists x \in A~ tq~ f(x)=y$$
On peut aussi écrire ce si et seulement si sous la forme : 
$$f(A) = A'$$
\end{de}
\subsection{Bijectivité}
\begin{de}
Soit f une application de A dans A'.\\
f est une bijection de A sur A' si et seulement si :
$$\forall y \in A'~ \exists!x\in A~ tq~ f(x)=y$$
Dans ce cas, on peut définir l'application réciproque $f^{-1}$ :
$$f^{-1} : A' \rightarrow A$$
$$y \rightarrow f^{-1}(y)$$
avec $f^{-1}(y)$ l'unique antécédent de y par f.
\end{de}
\begin{prop}
f est une bijection si f est injective et surjective.
\end{prop}
\begin{prop}
Soit f une fonction bijective.\\
Si $x \in A,~ y\in A'$.\\
$$(f(x) = y) \Leftrightarrow (x = f^{-1}(y))$$
Et : 
$$f{-1}of = Id_A$$
$$fof{-1} = Id_{A'}$$
\end{prop}
\subsection{Difféomorphisme}
\begin{de}
Soit f une application $C^k$ d'un intervalle I dans $\Re$, avec $k \in N^*$ (le cas k=0 est écarté, car ce cas possède un nom différent).\\
On dit que f réalise un $C^k$-difféomorphisme de I sur f(I) si et seulement si : 
\begin{itemize}
 \item[$\rightarrow$] f réalise une bijection de I sur f(I)
 \item[$\rightarrow$] f est $C^k$ sur I
 \item[$\rightarrow$] La fonction réciproque, $f^{-1}$ de f(I) dans I, est également $C^k$ sur f(I).
\end{itemize}
On dit que f est un $C^{\infty}$ difféomorphisme de I sur f(I) si f est un $C^k$ difféomorphisme de I sur f(I) pour tous $k \in N^*$
\end{de}
\begin{theo}
Soit f une application de classe $C^k$ de $\Re$ dans $\Re$, avec $k \in N^*$ sur un intervalle $I C \Re$.\\
Alors f est un $C^k$ difféomorphisme de I sur f(I) si et seulement si f' ne s'annule pas sur I.\\
Dans le cas, d'apres le théorème des valeurs intermédiaire, f' garde donc un signe continue sur I. Si $ \forall x \in I$ : 
\begin{itemize}
 \item[$\rightarrow$] f'(x) > 0, alors f est un $C^k$ difféomorphisme strictement croissant
 \item[$\rightarrow$] f'(x) < 0, alors f est un $C^k$ difféomorphisme strictement décroissant
\end{itemize}
\end{theo}
\section{Développements limités}
\subsection{Lien entre développement limité et dérivabilité}
Soit f fonction de $\Re$ dans K définies sur un voisinage $V_0$ de 0.
Supposons que f admet un développement limité d'ordre 1 de la forme : $f(x) = a + bx + o(x)$
\begin{prop}
$$(\mbox{ f admet un développement limité d'ordre 1 en 0 }) \Leftrightarrow ( \mbox{ f est dérivable en 0 })$$
\end{prop}
\begin{prop}
On obtient les égalités suivantes :
$$f(0) = a$$
$$f(0) = b$$
De plus, l'équation de la tangente en 0 au graphe de f est :
$$y=a+bx$$
\end{prop}
\subsection{Position relative de la courbe par rapport à la tangente}
Soit f fonction de $\Re$ dans K définies sur un voisinage $V_0$ de 0.
Supposons que f admet un développement limité d'ordre 2 de la forme : $f(x) = a + bx + cx^p + o(x^p)$, avec $c\neq 0$.
\begin{prop}
La position de la courbe par rapport à sa tangente au voisinage du point d'abscisse O=(0,a),est donnée à l'aide du signe de $c.x^p$ (Voir Signe et équivalent)
\end{prop}
\subsection{Développement limités usuels}
\begin{itemize}
 \item[$\rightarrow$]$(1+x)^{\alpha} = 1 + \alpha x+\dfrac{\alpha (\alpha - 1)}{2!}x^2+o(x^2)$
 \item[$\rightarrow$]$cos(x) = 1 - \dfrac{x^2}{2!}+\dfrac{x^4}{4!}-\dfrac{x^6}{6!}+o(x^6)$
 \item[$\rightarrow$]$sin(x) = 1 - \dfrac{x^3}{3!}+\dfrac{x^5}{5!}-\dfrac{x^7}{7!}+o(x^7)$
 \item[$\rightarrow$]$e^x = 1 +x + \dfrac{x^2}{2!}+\dfrac{x^3}{3!}+\dfrac{x^4}{4!}+o(x^4)$
 \item[$\rightarrow$]$\dfrac{1}{1-x} = 1+x+x^2+...+x^n+o(x^n)$
 \item[$\rightarrow$]$ch(x) = 1 + \dfrac{x^2}{2!}+\dfrac{x^4}{4!}+...+\dfrac{x^{2n}}{2n!}+o(x^{2n+1})$
 \item[$\rightarrow$]$sh(x) = 1 + \dfrac{x^3}{3!}+\dfrac{x^5}{5!}+...+\dfrac{x^{2n+1}}{(2n+1)!}+o(x^{2n+1})$
 \item[$\rightarrow$]$ln(1+x) = x - \dfrac{x^2}{2} + \dfrac{x^3}{3} +o(x^3)$
 \item[$\rightarrow$]$tan(x) = x + \dfrac{x^3}{3}+o(x^3)$
 \item[$\rightarrow$]$Arctan(x) = x- \dfrac{x^3}{3}+\dfrac{x^5}{5}-...+(-1)^n\dfrac{x^{2n+1}}{2n+1}+o(x^{2n+2})$
\end{itemize}
\subsection{Développement asymptotique d'une "échelle de comparaison E"}
\begin{de}
Soit $x_0 \in \bar{\Re}$, et E un ensemble de fonction de $\Re$ dans K, dont chacune est définie sur un voisinage fondamental de $x_0$.\\
On dit que f admet un développement asymptotique dans l'échelle E à la précision o($u_p$), $u \in E$, s'il existe des applications $u_1,...,u_p$ appartenant à E et des scalaires $\lambda_1,\lambda_2,...,\lambda_p \in K^p$, non tous nuls, et un voisinage fondamental V de $x_0$ telque :
$$u_1 \gg ... \gg u_p$$
$$\forall x \in V~ f(x)=\lambda_1.u_1(x)+...+\lambda_p.u_p(x)+o(u_p(x))$$
\end{de}
\subsubsection{Exemples d'échelle de comparaison E}
\begin{itemize}
 \item[$\rightarrow$] Pour $x_0 \in \Re$, l'échelle des dévellopements limités en $x_0$ :
$$E = \left\lbrace x \mapsto (x-x_0)^n, n \in N \right\rbrace $$
 \item[$\rightarrow$] Pour $x_0 \in \Re$ :
$$E = \left\lbrace x \mapsto (x-x_0)^n, n \in Z \right\rbrace $$
 \item[$\rightarrow$] Pour $x_0 = \infty $ :
$$E = \left\lbrace x \mapsto x^{\alpha}(ln(x))^{\beta}e^{P(x)},~ P(x) = \sum_{i=1}^q q_ix^{\gamma_1}, ~(\alpha,\beta) \in \Re^2,~ q \in N^*,~ q_i \in \Re^*,~ \gamma_i \in \Re^{*+} \right\rbrace $$

\end{itemize}

\section{Intégrale}
\subsection{Somme de Riemmann}
Soit f, fonction continue par morceaux de $\left[a,b\right]$ dans K.
Soit $(u_n)$ la somme de Riemmann associé à f
\begin{prop}
Quand n tend vers l'infini, le nombre de termes tend vers l'infini, et chacun des termes tend vers 0. La limite de la somme n'est pourtant pas nul
\end{prop}
\begin{prop}
$$\lim_{n \rightarrow \infty} u_n = \lim_{n \rightarrow \infty} \sum_{k=0}^{n-1} \dfrac{b-a}{n}.f(x_k) = \int_a^b f$$
\end{prop}
\subsection{Inégalité de majoration}
\begin{prop}
Soit f fonction continue par morceaux de [a,b], a<b, dans K.\\
$$|\int_a^b f| \leq \int_a^b |f| $$
\end{prop}
\subsubsection{Croissance de l'intégrale}
\begin{prop}
Soient f,g deux fonctions continue par morceaux de [a,b] a<b dans $\Re$.\\
Si :
$$f \leq g$$
alors :
$$\int_a^b f \leq \int_a^b g$$
\end{prop}
\subsection{Inégalité de Cauchy-Schwarz}
\begin{enon}
Soit u et v deux vecteur d'un $\Re$ espace vectoriel. :
$$<u|v> \leq ||u||.||v||$$
\end{enon}
\begin{prop}
Si f et g sont deux applications continue par morceaux de [a,b] dans K, alors :
$$|\int_a^b f(t)g(t)dt| \leq \sqrt{\int_a^b |f(t)|^2dt}.\sqrt{\int_a^b |g(t)|^2dt}$$
Le produit scalaire sous jacents dans le cas réel est : 
$$<f|g> = \int_a^b f(t).g(t) dt$$
A l'aide de ceci, on peut démontrer l'inégalité des accroissements finis.
\end{prop}
\subsection{Intégrale et négligabilité}
\begin{prop}
Si u et v sont deux applications de $\Re$ dans K, continue sur un voisinage de 0 (pour que l'on puisse définir les intégrales). Alors :
$$u \underset{0}\ll v \Rightarrow  \int_0^x u \underset{0}\ll \int_0^x v$$
\end{prop}
\section{Vrac}
\subsection{Suite géométrique}
\begin{prop}
La somme d'une suite géométrique de raison z est donnée par, pour $z \neq 1$ :
$$1+z+...+z^n = \dfrac{1-z^{n+1}}{1-z}$$
Pour z = 1 : 
$$1+z+...+z^n = n+1$$
\end{prop}
\subsubsection{Application aux matrices}
Soit A une matrice carrée.\\
De même, on obtient, si (I-A) est inversible :
$$I + A + A^2 + ... + A^n = (I-A)^{-1}(I-A^{n+1})$$
$$I + A + A^2 + ... + A^n = (I-A^{n+1})(I-A)^{-1}$$
On observe que les matrices commutent (Ce qui n'est pas le cas général)
\subsection{Suite complexe}
Soit $(u_n)$ une suite de complexe.
\begin{prop}
$$\lim_{n \rightarrow \infty} |u_n| = 0 \Leftrightarrow \lim_{n \rightarrow \infty} u_n = 0$$
\end{prop}
\subsection{Utilisations des inégalités}
Soient $(u_n)$ et $(v_n)$ deux suites qui tendent vers l et l'
\begin{prop}
Si $\forall n \in N~ u_n \leq v_n$, alors $l \leq l'$.\\
Mais si Si $\forall n \in N~ u_n < v_n$, alors $l \leq l'$.\\
On observe donc qu'il est plus "facile" de travailler sur des inégalités large.
\end{prop}
\subsection{Densité}
Soient A et A' deux sous ensembles non vide de C, avec A inclue dans A'.\\
\begin{enon}
On dit que A est dense dans A' si :
$$\forall a' \in A',~ \forall \varepsilon > 0,~ \exists a \in A~ tq~ |a'-a| \leq \varepsilon$$
\end{enon}
\begin{prop}
On dit que A est dense dans A' si tous points de A' est la limite d'une suite de points de A
\end{prop}
\subsection{Formule du binôme et dérivée}
\begin{de}
Soit a et b deux complexes :
$$(a+b)^n = \sum_{k=0}^n \dbinom{n}{k} a^k.b^{n-k}$$
\end{de}
\begin{prop}
On peut étendre la formule du binome au dérivation d'un produit de fonction : 
$$(f.g)^{(n)} = \sum_{k=0}^n \dbinom{n}{k} f^{(k)}.g^{(n-k)}$$
\end{prop}
\begin{prop}
De même, on peut étendre cette formule aux matrices si les deux matrices commutent.\\
Soient A et B deux matrice carrée telque AB=BA : 
$$(A+B)^n = \sum_{k=0}^n \dbinom{n}{k} A^k.B^{n-k}$$
\end{prop}
\subsection{Dérivée successives de cosinus et sinus}
\begin{enon}
Soit n $\in$ N :
$$cos^{(n)}(x) = cos(x+n.\dfrac{\pi}{2})$$
$$sin^{(n)}(x) = sin(x+n.\dfrac{\pi}{2})$$
\end{enon}

\begin{prop}
$\forall k \in N : $
$$cos^{(2k)} = (-1)^k.cos(x)$$
$$sin^{(2k+1)} = (-1)^{k+1}cos(x)$$
\end{prop}
\subsection{Régle de d'Alembert}
Soit ($u_n$) une suite de réels positifs.\\
Supposons que :
$$\lim_{n \rightarrow \infty} \dfrac{u_{n+1}}{u_n} = a$$
\begin{itemize}
 \item[$\rightarrow$] Si a $\in \left[0,1\right[$, ($u_n$) converge vers 0
 \item[$\rightarrow$] Si a>1, $(u_n)$ diverge vers $+\infty$
\end{itemize}
\subsection{Nombre complexe}
\begin{prop}
Si z = x + iy, avec x,y deux réels, alors : 
$$|x| \leq |z|$$
$$|y| \leq |z|$$
\end{prop}
\section{Les polynomes}
\subsection{Polynomes irréductibles}
\begin{prop}
Si K est un corps commutatif, tous polynomes de K[X] s'écrit comme le produit d'un nombres de polynomes irréductible de K[X] et cette écriture est unique à l'ordre près des facteurs.
\end{prop}
\begin{enon}
Les polynomes irréductibles de C[X] sont les polynomes du $1^{er}$ degrés.\\
Les polynomes irréductibles de C[X] unitaire sont les polynome aX+$\lambda$, avec $\lambda$ un complexe et a=1.
\end{enon}
\begin{enon}
Les polynomes irréductibles de $\Re$[X] sont les polynomes du $1^{er}$ degrés et les polynomes du $2^{nd}$ degrés avec discriminant négatif.
\end{enon}
\subsection{Racine et ordre de multiplicité}
\begin{de}
Soit $\lambda$ un complexe.\\
$\lambda$ est une racine d'ordre n de $P \in K[X]$ si et seulement si $(X-\lambda)^n$ divise P et que $(X-\lambda)^{n+1}$ ne le divise pas.
\end{de}
\begin{prop}
Soit $\lambda$ un complexe.\\
$\lambda$ est une racine d'ordre n de $P \in K[X]$ si et seulement si :
$$\forall k \in \left\lbrace0,n-1\right\rbrace~ P^{(k)}(\lambda) = 0 $$
et que
$$P^{(n)}(\lambda) \neq 0$$
\end{prop}
\begin{prop}
Soit z un complexe.\\
On peut factoriser $z^n-1$ sous la forme : 
$$z^n-1 = \prod_{k=0}^{n-1}(z-e^{i\dfrac{k2\pi}{n}})$$
\end{prop}
\begin{prop}
D'après la propriété précédente, et en utilisant le faite que :
$$1 + z + ... +z^{n-1} = \dfrac{z^{n}-1}{z-1}$$
On obtient que : 
$$1 + z + ... +z^{n-1} =  \prod_{k=1}^{n-1}(z-e^{i\dfrac{k2\pi}{n}})$$
\end{prop}
\begin{prop}
Soit $P\in\Re[X]$, et $\lambda$ une racine complexe de P, avec $\lambda$ non réel.\\
Alors :
$$mult_P(\bar{\lambda}) = mult_P(\lambda)$$
\end{prop}

\part{Intégrales, Fonctions}
\setcounter{chapter}{0}
\chapter{Intégrales généralisées, Fonctions intégrables}
\section{Applications continues par morceaux}
En ce qui concerne les intégrales, le programme se limite aux applications continues par morceaux.
\subsection{Définitions}
\begin{de}
Soit f, une application (fonction définies sur tous l'espace de départ) de $\left[a,b\right]$ dans K, avec a et b deux réels, a<b.\\
On dit que f est continue par morceaux sur [a,b] s'il existe une subdivision finie de [a,b] : 
$$a = x_0 < x_1 < ... < x_p = b$$
telque : 
\begin{itemize}
 \item[$\rightarrow$]$\forall i \in \left\lbrace0,...,p-1\right\rbrace$, f est $C^0$ sur ]$x_i,x_{i+1}$[ 
 \item[$\rightarrow$]$\forall i \in \left\lbrace1,...,p-1\right\rbrace$, f admet une limite finie à droite et une limite finie à gauche de $x_i$
 \item[$\rightarrow$]f admet une limite finie à droite en a et une limite finie à gauche en b
\end{itemize}
\end{de}
\begin{de}
On peut définir pour tous i $\in \left\lbrace0,...,p-1\right\rbrace$ des applications continues $\overset{\sim}f_i$ de  $\left[x_i,x_{i+1}\right]$ dans K, définies par : 
\begin{itemize}
 \item[$\rightarrow$]Si $x \in \left] x_i, x_{i+1}\right[~ \overset{\sim}f_i(x) = f(x) $
 \item[$\rightarrow$]$\overset{\sim}f_i(x_i) = \underset{x\rightarrow x_i^+}\lim f(x)$
 \item[$\rightarrow$]$\overset{\sim}f_i(x_{i+1}) = \underset{x\rightarrow x_{i+1}^-}\lim f(x)$
\end{itemize}
$\overset{\sim}f_i$ est donc le prolongement par continuité de f sur $\left[x_i,x_{i+1}\right]$  
\end{de}
Il existe aussi la variante suivante de cette définition : 
\begin{de}
Soit f application de $\left[a,b\right]$ dans K.\\
f est continue par morceaux si il existe une subdivision finie de $\left[a,b\right]$ :
$$a = x_0 < x_1 < ... < x_p = b$$
et des applications continues, avec $i \in \left\lbrace0,...,p-1\right\rbrace$ : 
$$\overset{\sim}f : \left[x_i,x_{i+1}\right] \rightarrow K$$
telque, $\forall i \in \left\lbrace0,...,p-1\right\rbrace$ : 
$$f_{\left]x_i,x_{i+1}\right[} = \overset{\sim} f_i$$
\end{de}
\begin{de}
Soit I un intervalle quelconque de $\Re$ et f, application de I dans K.\\
On dit que f est continue par morceaux sur I si f est continue par morceaux sur tous segments (intervalle borné fermé) $\left[a,b\right] C I$
\end{de}
\begin{prop}
Si f est une application continue par morceaux de $\left[a,b\right]$ dans K, alors :
$$\int_a^b f = \sum_{i=0}^{p-1} \int_{x_i}^{x_{i+1}} \overset{\sim}f_i$$
\end{prop}
\begin{prop}
Soient f,g fonctions de $\left[a,b\right]$ dans K.
Si f est une application intégrable au sens de Riemann sur $\left[a,b\right]$ et si g ne diffère qu'en un nombre finie de point de f, alors g est également intégrable au sens de Riemann et : 
$$\int_a^b g = \int_a^b f$$
\end{prop}
\section{Convergence d'une intégrale}
\subsection{Convergence vers $\infty$}
\begin{de}
Soit f application continue par morceaux de $\left[a,\infty\right[$ dans K.\\
Alors, $\forall x \in \left[a,\infty\right[$, f est continue par morceaux sur $\left[a,x\right]$ et on peut calculer :
$$\int_a^x f$$
C'est à dire que l'on peut définir une nouvelle application F :
$$F : \left[a,\infty\right[ \rightarrow K$$
$$x \mapsto \int_a^x f$$
Lorsque F a une limite finie, quand $x \rightarrow \infty$, on dit que $\int_a^{\infty} f$ converge, et on note : 
$$\int_a^{\infty} f = \lim_{x \rightarrow \infty} F(x)$$
En réalité, on devrai dire que $\int_a^{\infty} f$ existe, ou que $\int_a^x f$ converge quand $x \rightarrow \infty$
\end{de}
\begin{prop}
Si $\alpha \in \Re$ et $a \in \Re^{+*}$ :\\
$\int_a^{\infty} \dfrac{dt}{t^{\alpha}}$ converge si et seulement si $\alpha$ > 1.\\
De plus, on peux calculer la limite par primitivation. 
\end{prop}
\begin{prop}
Soit f une application continue par morceaux de $\left[a,\infty\right[ $ dans K, et si b $\in \left[a,\infty\right[ $
$$\int_a^{\infty} f \mbox{ converge si et seulement si } \int_b^{\infty} f \mbox{ converge}$$
Dans ce cas, on obtient la relation de Chasles : 
$$\int_a^{\infty} f = \int_a^b f + \int_b^{\infty} f$$
\end{prop}
\subsection{Convergence vers 0}
\begin{de}
Soit f application continue par morceaux de $\left]0,a\right]$ dans K.\\
Alors on peut définir F, fonction de $\left]0,a\right]$ dans K, défini par :
$$\forall x \in \left]0,a\right]~ F(x)=\int_x^a f$$
Car f est continue par morceaux sur $\left[ x,a\right]$
On dit que $\int_0^a f$ converge lorsque F($x$) a une limite finie quand x tend vers $0^+$.
En réalité, on devrai dire que $\int_0^{a} f$ existe, ou que $\int_0^a f$ converge quand $x \rightarrow 0^+$
\end{de}
\begin{prop}
Si $\alpha \in \Re$ et $a \in \Re^{+*}$ :\\
$\int_0^a \dfrac{dt}{t^{\alpha}}$ converge si et seulement si $\alpha$ < 1.\\
De plus, on peux calculer la limite par primitivation. 
\end{prop}
\begin{prop}
Soit f une application continue par morceaux de $\left]0,a\right] $ dans K.
$$\forall b \in \left]0,a\right] \int_0^a f \mbox{ converge si et seulement si } \int_0^b f \mbox{ converge}$$
Dans ce cas, on obtient la relation de Chasles : 
$$\int_0^a f = \int_0^b f + \int_b^a f$$
\end{prop}
\section{Résultats spécifiques sur les applications de $\left[a;\infty\right[$ à valeurs dans K}
\subsection{Limite de l'application et convergence de l'intégrale}
Soit f, application continue par morceaux sur $\left[a;\infty\right[$ à valeurs dans K, avec $a \in \Re$.
\begin{prop}
Si f a une limite finie l, l $\in$ K, quand $x \rightarrow \infty$, et si $\int_a^{\infty} f$ converge, alors l=0
\end{prop}
\begin{prop}
Si f est à valeur réelle et si f a une limite l, l $\in \Re+\left\lbrace+\infty;-\infty\right\rbrace $, quand $x \rightarrow \infty$, et si $\int_a^{\infty} f$ converge, alors l=0.
\end{prop}
\begin{prop}
Soit f application continue par morceaux de $\left[a,\infty\right[$ dans $\Re$.\\
$\int_a^{\infty}$ peut être convergente sans que f ait une limite en $\infty$, ou que f soit bornée.
\end{prop}
\subsection{Caractérisation séquentielle d'une limite}
\begin{prop}
Soit f application continue par morceaux de $\left[a,\infty\right[$ dans K. Soit $l \in K$
$$(\lim_{x \rightarrow \infty } f(x) = l) \Leftrightarrow (\forall (x_n) \mbox{ à valeur dans } \left[a,\infty\right[ \mbox{ tendant vers }\infty, ~ f(x_n) \underset{n\rightarrow\infty}\rightarrow l)$$
\end{prop}
\begin{prop}
Soit f application de $\Re$ dans K définies sur un voisinage V de $x_0 \in \bar{\Re}$, et l $\in K$ ( ou l $\in \bar{\Re}$ si K=$\Re$)
$$(\lim_{x \underset{x \in V}\rightarrow x_0 } f(x) = l) \Leftrightarrow (\forall (x_n) \mbox{ à valeur dans V, tendant vers }x_0, ~ f(x_n) \underset{n\rightarrow\infty}\rightarrow l)$$
\end{prop}
\section{Définitions et propriétés générales}
Dans les chapitres suivants, on adopte la notation suivante : 
$$\int_a^{\alpha} f = \lim_{x \rightarrow \alpha} \int_a^x f$$
\begin{de}
Soit a $\in \Re$, et $\alpha \in \Re+\left\lbrace+\infty\right\rbrace $, avec a $\leq \alpha$.\\
Soit f application continue par morceaux de $\left[a,\alpha\right[$.\\
On peut alors définir F :
$$F : \left[a,\alpha\right[ \rightarrow K$$
$$x \mapsto \int_a^x f$$
On dit que $\int_a^{\alpha} f$ converge si et seulement si F(x) a une limite finie dans K quand x tend vers $\alpha$ par valeur inferieur.\\
On note alors $\int_a^{\alpha}$ cette limite.
\end{de}
\begin{prop}
Soit f application continue par morceaux de $\left[a,\alpha\right[$ dans K.\\
Si b$\in \left[a,\alpha\right[$ :
$$\left(\int_a^{\alpha} f \mbox{ converge }\right) \Leftrightarrow \left(\int_b^{\alpha} f \mbox{ converge } \right)$$
Dans ce cas, nous avons la relation de Chasles suivante :
$$\int_a^{\alpha} f = \int_a^b f + \int_b^{\alpha} f$$
\end{prop}
\begin{prop}
Soit f application continue par morceaux de $\left[a,\alpha\right[$ dans K.\\
$\forall \lambda \in \Re$ :  
$$\lambda f : \left[a,\alpha\right[ \rightarrow K$$
$$x\rightarrow\lambda f(x)$$
est une fonction continue par morceaux et :
$$\left(\int_a^{\alpha} f \mbox{ converge }\right) \Rightarrow \left(\int_a^{\alpha} \lambda f \mbox{ converge }\right)$$
Et dans ce cas :
$$\int_a^{\alpha} \lambda f = \lambda \int_a^{\alpha} f$$
\end{prop}
\begin{prop}
Soient f et g deux applications continues par morceaux de $\left[a,\alpha\right[$ dans K.\\
Alors f+g est continue par morceaux.\\
Si $\int_a^{\alpha} f$ et $\int_a^{\alpha} g$ converge, alors $\int_a^{\alpha} f+g$ converge et :
$$\int_a^{\alpha} f+g = \int_a^{\alpha} f + \int_a^{\alpha} g$$
\end{prop}
\begin{prop}
Soit f application de $\left[a,\alpha\right[$ dans C. Soient $f_1 $= Re(f) et $f_2 = Im(f)$.\\
f est continue par morceaux sur $\left[a,\alpha\right[$ $\Leftrightarrow$ $f_1$ et $f_2$ sont continue par morceaux sur $\left[a,\alpha\right[$.\\
Dans ce cas : 
$$\left(\int_a^{\alpha} f \mbox{ converge }\right) \Leftrightarrow \left(\int_a^{\alpha} f_1 ~et~ \int_a^{\alpha}  f_2\mbox{ convergent }\right)$$.
On obtient alors : 
$$\int_a^{\alpha} f = \int_a^{\alpha} f_1 + i\int_a^{\alpha} f$$
\end{prop}
\begin{prop}
Soit f application continue par morceaux de $\left[a,\alpha\right[$ dans K, avec : 
$$\alpha \in \Re\cup\left\lbrace-\infty\right\rbrace $$
$$\beta \in \Re\cup\left\lbrace+\infty\right\rbrace $$
$$\alpha \leq \beta$$
Soit $\gamma \in \left]\alpha,\beta\right[$
Si f est continue par morceaux sur $\left]\alpha,\gamma\right[$ et sur $\left]\gamma,\beta\right[$
On dit que $\int_{\alpha}^{\beta} f$ converge si les intégrales $\int_{\alpha}^{\gamma} f$ et $\int_{\gamma}^{\beta} f$ convergent.\\
On obtient alors que :
$$\int_{\alpha}^{\beta} f = \int_{\alpha}^{\gamma} f+ \int_{\gamma}^{\beta} f$$
\end{prop}
\section{Convergence Absolue, Fonctions intégrables sur un intervalle}
Dans ce chapitre, nous allons nous limiter aux applications continue par morceaux de $\left[a,\alpha\right[$ dans K, avec :
$$\alpha \in \Re\cup\left\lbrace-\infty\right\rbrace $$
$$a \in \Re,~ a \leq \alpha$$
Mais les définitions et résultats se généralise sur des applications continues par morceaux sur $\left]\alpha,\beta\right[$ ou sur $\left]\alpha,\gamma\right[,\left]\gamma,\beta\right[$.
\subsection{Convergence Absolue}
\begin{de}
Soit f application continue par morceaux sur $\left[a,\alpha\right[$ dans K.\\
L'application g :
$$g : \left[a,\alpha\right[ \rightarrow \Re^+$$
$$x \mapsto |f(x)|$$
est aussi continue par morceaux sur $\left[a,\alpha\right[$.\\
On dit que f est intégrable sur $\left[a,\alpha\right[$ ou que $\int_{a}^{\alpha} f$ converge absolument lorsque $\int_{a}^{\alpha} |f|$ converge
\end{de}
\begin{prop}
$$(\int_{a}^{\alpha} f \mbox{ converge absolument} ) \Rightarrow (\int_{a}^{\alpha} f \mbox{converge})$$
\end{prop}
\subsection{Critère de Cauchy}
\subsubsection{Critère de Cauchy pour les suites}
\begin{enon}
On dit qu'une suite $(u_n)$ a valeur dans K vérifie le critère de Cauchy si et seulement si: \\
$\forall \varepsilon > 0, \exists N_0 \in N~ tq~ \forall (p,q) \in N^2~ tq~ p~ et~ q \geq N_0$ :
$$|u_p-u_q| \leq \varepsilon$$
En faite, on obtient une définition équivalente en se limitant aux couples (p,q)$\in N^2$ telque q<p. Ce qui donne : \\
$(u_n)$ est une suite de Cauchy si et seulement si :
$$\forall \varepsilon > 0, \exists N_0 \in N~ tq~ \forall q>p\geq N_0~ |u_p-u_q| \leq \varepsilon$$
\end{enon}
\begin{prop}
Toutes suites $(u_n)$ à valeur dans $\Re$ vérifiant le critère de Cauchy converge.
\end{prop}
\begin{prop}
 Toutes suite de Cauchy à valeur dans C converge dans C
\end{prop}
\subsubsection{Critère de Cauchy pour les fonctions}
\begin{de}
Soit f fonction de $\Re$ dans K, définie sur un voisinage V de $x_0 \in \bar{\Re}$.\\
On dit que f vérifie le critère de Cauchy en $x_0$ si et seulement si :
$$\forall \varepsilon > 0,~ \exists U \mbox{ voisinage de }x_0 \mbox{ dans V tq } \forall(x,x')\in U^2, |f(x)-f(x')| \leq \varepsilon$$
\end{de}
\begin{prop}
Avec les notations précédantes, si f, fonction de $\Re$ dans K, définie au voisinage de $x_0 \in \bar{\Re}$, admet un limite finie en $x_0$, alors f vérifie le critère de Cauchy en $x_0$
\end{prop}
\begin{prop}
Si f, fonction de $\Re$ dans K, définie au voisinage de $x_0 \in \bar{\Re}$, vérifie le critère de Cauchy en $x_0$, alors f admet une limite finie en $x_0$
\end{prop}
\section{Convergence des intégrales de fonctions positives - Intégrabilité}
\subsection{Propriétés fondamentale}
\begin{prop}
Soit f, application continue par morceaux de [a,b[ dans $\Re$.\\
Si f est à valeurs positives sur [a,b[, alors :
$$x \mapsto \int_a^x f(t)dt \mbox{ est croissante}$$  
\end{prop}
\subsubsection{Convergence d'une intégrale de fonction positive par majoration}
\begin{prop}
Soient f et g deux fonction continue par morceaux de [a,b[ dans $\Re$.\\
Si $\forall t \in$[a,b[ $0 \leq f(t)\leq g(t)$, alors : 
$$( \int_a^b g \mbox{ converge } ) \Rightarrow (\int_a^b f \mbox{ converge et } 0 \leq \int_a^bf(t)\leq \int_a^bg(t) )$$ 
$$( \int_a^b g \mbox{ diverge } ) \Rightarrow (\int_a^b f \mbox{ diverge }$$
\end{prop}
\subsubsection{Intégration par domination}
\begin{prop}
Soient f et g deux fonctions continue par morceaux de [a,b[ dans K.\\
Si $f \underset{b^-}= O(g)$ (Grand O), alors :
$$(\mbox{ g intégrable sur [a,b[ })\Rightarrow(\mbox{ f intégrable sur [a,b[ })$$
\end{prop}
\subsubsection{Convergence des intégrales de fonction positive}
\begin{prop}
Soient f et g deux fonction continue par morceaux de [a,b[ dans $\Re$, telque : 
\begin{itemize}
 \item[$\rightarrow$]f et g soit de signe constant au voisinage de $b^-$
 \item[$\rightarrow$]$f \underset{b^-}\sim g$
\end{itemize}
alors :
$$( \int_a^b g \mbox{ converge } ) \Leftrightarrow (\int_a^b f \mbox{ converge }$$
$$( \int_a^b g \mbox{ diverge } ) \Leftrightarrow (\int_a^b f \mbox{ diverge }$$
On dit que ces deux intégrales sont de même nature.
\end{prop}
\begin{prop}
Soit b réel fini > a.\\
Si f est une fonction continue par morceaux de [a,b[ dans K, et si f a une limite finie en $b^-$, alors :
$$\int_a^b f \mbox{ converge }$$
\end{prop}
\subsection{Règles de Riemann}
\subsubsection{En $\infty$}
Soit $a \in \Re$
\begin{enon}
Soit f fonction continue par morceaux de $\left[a,\infty \right[ $ dans K.\\
Si il existe $\alpha > 1$ telque $t^{\alpha}f(t) \underset{t \rightarrow \infty}\rightarrow 0$, alors f est intégrable sur $\left[a,\infty \right[$.
\end{enon}
\begin{prop}
Soit f fonction de $\left[a,\infty \right[$ dans $\Re$, continue par morceaux.\\
Si tf(t) $\underset{t \rightarrow \infty}\rightarrow 0$, alors :
$$\int_a^{\infty} f \mbox{ diverge }$$
\end{prop}
\subsubsection{En 0}
Soit $a \in \Re$
\begin{enon}
Soit f fonction continue par morceaux de $\left]0,a \right] $ dans K.\\
Si il existe $\alpha < 1$ telque $t^{\alpha}f(t) \underset{t \rightarrow O^+}\rightarrow 0$, alors f est intégrable sur $\left]0,a\right]$.
\end{enon}
\begin{prop}
Soit f fonction de $\left]0,a\right]$ dans $\Re$, continue par morceaux.\\
Si tf(t) $\underset{t \rightarrow O^+}\rightarrow 0$, alors :
$$\int_0^{a} f \mbox{ diverge }$$
\end{prop}
\subsection{Intégrale de Bertrand}
\begin{prop}
Soit a>1 et $(\alpha,\beta) \in \Re^2$.
$$\left(\int_a^{\infty} \dfrac{dx}{x^{\alpha}ln(x)^{\beta}} \mbox{ converge }\right) \Leftrightarrow ( \alpha > 1,~ ou~ \alpha=1,\beta>1)$$
\end{prop}
\begin{prop}
Soit $a \in \left]0,1\right[$ et $(\alpha,\beta) \in \Re^2$.
$$\left(\int_0^{a} \dfrac{dx}{x^{\alpha}ln(x)^{\beta}} \mbox{ converge }\right) \Leftrightarrow ( \alpha < 1,~ ou~ \alpha=1,\beta>1)$$
\end{prop}
\section{Intégration par parties - Changement de variable}
\subsection{Intégration par parties}
\begin{de}
Soient u et v deux applications de [a,b[ dans K $C^1$ par morceaux et continue sur [a,b[. 
$$\int_a^b u'.v = \left[u.v\right]_a^b - \int_a^b u.v' $$
Avec : 
$$\left[u.v\right]_a^b = \lim_{x \rightarrow b^-} \left[u.v\right]_a^x$$
\end{de}
\subsection{Changement de variable}
Soit f fonction continue sur [a,b[, à valeur dans K.\\
Soit g fonction $C^1$ sur [a',b'[ à valeur dans [a,b[, avec a=g(a'), b=$\underset{x\rightarrow b^-}\lim g(x)$.\\
$$\int_a^b f(t).dt = \int_{a'}^{b'} f(g(u)).g'(u).du$$
\section{Quelques espaces remarquables}
Dans tous ce chapitre, on considère que I est un intervalle fondamental.
\begin{de}
On dit que I est un intervalle fondamental si : 
$$I = \left[a,b\right] ; I = \left[a,b\right[ ; I = \left]a,b\right] ; I = \left]a,b\right[ $$
Avec, selon les cas :
$$a \in \Re~ ou~ a \in \Re\cup\left\lbrace-\infty\right\rbrace $$
$$b \in \Re~ ou~ b \in \Re\cup\left\lbrace+\infty\right\rbrace $$
Cette notation est une notation personnelle.
\end{de}
\begin{prop}
L'ensemble des applications continue par morceaux de I dans K telque $\int_I f$ converge est un K espace vectoriel
\end{prop}
\begin{prop}
L'ensemble $L^1_{cpm}$, qui est l'ensemble des applications continue par morceaux sur I, à valeur dans K, et intégrable sur I, est un K espace vectoriel. C'est un sous-espace vectoriel de l'espace précédent.
\end{prop}
\begin{prop}
L'ensemble $L^2_{cpm}$, qui est l'ensemble des applications continue par morceaux sur I, à valeur dans K, de carrée intégrable sur I, est un K espace vectoriel.
\end{prop}
\begin{lemme}
Soit a et b deux complexes : 
$$|a+b|^2 \leq 2.(|a|^2 + |b|^2)$$
Si a et b sont réel, ce lemme devient : 
$$(a+b)^2 \leq 2.(a^2 + b^2)$$
\end{lemme}
\section{Remarque concernant le reste}
\begin{de}
Soit f fonction de $\left[a,b\right[$ dans K, avec a réel et $b \in \Re \cup \left\lbrace +\infty \right\rbrace$, continue par morceaux et telque $\int_a^b$ converge.\\
On a donc aussi : 
$$\forall x \in \left[a,b\right[ \int_x^b f \mbox{ converge }$$ 
On peut donc définir le reste intégrale au voisinage de b, notée R(x) :
$$R : \left[a,b\right[ \rightarrow K$$
$$x \mapsto \int_x^b f$$
\end{de}
\begin{prop}
Avec les notations et définition précédantes, on obtient que : 
$$\lim_{x \rightarrow b^-} R(x) = 0$$
\end{prop}

\chapter{Intégrales à paramètres}
\section{Théorème de continuité}
Soit :
$$f : X\times I \rightarrow K $$
$$(x,t) \mapsto f(x,t)$$
avec X et I intervalles de $\mathbb{R}$.\\
On peut définir :
$$F(x) = \int_I f(x,t) dt$$
à condition suffisante que $t \mapsto f(x,t)$ soit $C^0$ par morceaux sur I et que $\int_I f(x,t)dt$ converge.\\
Si cette condition est satisfaite, pour tout $x \in X$, on peut définir une nouvelle application :
$$F : X \rightarrow K$$
$$x \mapsto F(x) $$
Avec : 
$$F(x) = \int_I f(x,t) dt$$
\begin{theo}
Avec les notations précédentes, si : 
\begin{itemize}
 \item[$\rightarrow$] $\forall x \in X$, $t \mapsto f(x,t)$ est continue par morceaux sur I
 \item[$\rightarrow$] $\forall t \in I$, $x \mapsto f(x,t)$ est continue sur X
 \item[$\rightarrow$] Condition de domination : $\exists \varphi : I \rightarrow \mathbb{R}_+$, condition par morceaux, intégrable sur I, telque :
$$\forall (x,t) \in X \times I~ |f(x,t)| \leq \varphi(t)$$
\end{itemize}
Alors F est définie et continue sur X.
\end{theo}
\begin{prop}
En considérant que la continuité est une propriété locale, on peut remplacer la condition de domination par : \\
$\forall [a,b] $c X, $\exists \varphi_{[a,b]} : I \rightarrow \Re_+$ continue par morceaux et intégrale sur I telle que : 
$$\forall x \in [a,b]~ \forall t \in I, |f(x,t)|\leq \varphi_{[a,b]}(t)$$
\end{prop}
\section{Théorème de classe $C^1$}
\begin{theo}
Soit f : 
$$f : \mathbb{R}^2 \rightarrow K$$
$$(x,t) \mapsto f(x,t)$$
Avec K un corps, X et I deux intervalles de $\mathbb{R}$.\\
Si : 
\begin{itemize}
 \item[$\rightarrow$] $\forall x \in X$, $t \mapsto f(x,t)$ est continue par morceaux sur I
 \item[$\rightarrow$] $\forall t \in I$, $x \mapsto f(x,t)$ est $C^1$ sur X et t $\rightarrow \dfrac{\partial f}{\partial x}(x,t)$ est continue par morceaux sur I
 \item[$\rightarrow$] Condition de domination : $\exists \varphi : I \rightarrow \mathbb{R}_+$, condition par morceaux, intégrable sur I, telque :
$$\forall (x,t) \in X \times I~ |\dfrac{\partial f}{\partial x}(x,t)| \leq \varphi(t)$$
\end{itemize}
Alors F : 
$$F : X \rightarrow K$$
$$x \mapsto F(x) $$
Avec : 
$$F(x) = \int_I f(x,t) dt$$
est définie et de classe $C^1$ sur X et :
$$\forall x \in X~ F'(x) = \int_I \dfrac{\partial f}{\partial x} (x,t) dt$$
On appelle ceci formule de dérivation soussigne intégrale de Lipnitz
\end{theo}
\begin{prop}
En considérant que la continuité est une propriété locale, on peut remplacer la condition de domination par : \\
$\forall [a,b] $c X, $\exists \varphi_{[a,b]} : I \rightarrow \Re_+$ continue par morceaux et intégrale sur I telle que : 
$$\forall x \in [a,b]~ \forall t \in I, |\dfrac{\partial f}{\partial x}(x,t)|\leq \varphi_{[a,b]}(t)$$
\end{prop}
\section{Théorème de classe $C^p$ (Hors programme)}
\begin{theo}
Soit F :
$$F : x \mapsto \int_I f(x,t) dt$$ 
avec f une fonction de XxI dans K, avec X et I des intervalles inclus dans $\mathbb{R}$. Si : 
\begin{itemize}
 \item[$\rightarrow$] $\forall t \in I$, $x \mapsto f(x,t)$ est $C^p$ sur X, $p \in \mathbb{N}$.
 \item[$\rightarrow$] Condition de domination : $\forall x \in X, \forall k \in [|0,p|]$, $t \mapsto \dfrac{\partial^k f}{\partial x^k}(x,t)$ est continue par morceaux sur I, et $\forall k \in [|0,p|]$, $\exists \varphi_k$ fonction de I dans $\mathbb{R}_+$, continue par morceaux sur I et intégrable sur I telque : 
$$\forall (x,t) \in X \times I,~ |\dfrac{\partial^k f}{\partial x^k}(x,t)|\leq \varphi_k(t)$$
Alors F est $C^p$ sur X et les dérivées succésives s'obtienne en dérivant sous le signe intégrale
\end{itemize}
\end{theo}
\begin{prop}
Comme dans les théorèmes précédent, en considérant le caractère locale de la classe $C^p$, on peut se ramener pour la condition de domination à tout segment inclu dans X. On peut aussi considéré une famille d'intervalle exaustives.
\end{prop}
\chapter{Approximation uniforme}
\section{Approximation uniforme par des fonctions en escaliers}
\begin{theo}
Soit [a,b] un segment inclu dans $\mathbb{R}$.\\
Pour toute fonctions f de [a,b] dans K ,ou plus généralement dans E, un K espace vectoriel normée, et $\forall \varepsilon > 0$ :
$$\exists \varphi : [a,b] \rightarrow K~ ou~ E$$
fonction en escalier, telle que : 
$$\parallel f - \varphi \parallel_{\infty,[a,b]} < \varepsilon $$
\end{theo}
Il existe plusieurs variantes de théorème :
\begin{theo}
Caractérisation séquentielle : \\
Avec les notations précédentes, quelque soit f, fonction de [a,b] dans K ou E, il existe une suite de fonctions en escalier $(\varphi_n)$ de [a,b] dans K ou E, convergeant uniformement vers f sur [a,b]
\end{theo}
\begin{theo}
L'ensemble $\mathcal{E}$ des fonctions en escalier de [a,b] dans K ou E est dense dans l'ensemble $C_{pm}$ des fonctions continue par morceaux de [a,b] dans K ou E.
\end{theo}
\section{Généralisation aux fonctions continues par morceaux}
Avec les notations précédentes :\\
Soit f une fonction de [a,b] dans K ou E, continue par morceaux. C'est à dire qu'il existe une subdivision, noté $\sigma$ :
$$a = a_0 < \dots < a_p = b$$
telque f soit continue sur $]a_k,a_{k+1}[$ , $k \in [|0,p-1|]$, et f admet une limite à droite et à gauche en $a_k$, à droite en $a_0$, a gauche en $a_p$.\\
On peut donc définir $f_k$, le prolongement par continuité de f sur [$a_k,a_{k+1}$].\\
Soit $\varepsilon > 0$. Il existe $\varphi_k$ en escalier sur [$a_k,a_{k+1}$] telque : 
$$\parallel f_k - \varphi_k \parallel_{\infty,[a,b]} < \varepsilon $$
On peut donc définir $\varphi$, commme coincident avec les $\varphi_k$ et égale à f aux bornes des intervalles. On obtient donc que : 
$$\parallel f - \varphi \parallel_{\infty,[a,b]} < \varepsilon $$
\section{Théorème}
\begin{de}
Soit $\varphi :$ $[a,b] \rightarrow E$. On dit que $\varphi$ est affine si il existe une subdivision :
$$a < x_0 < ... < x_p=b$$
telque : 
$$\forall k \in [|0,p-1|],~ \exists (\overrightarrow{\alpha_k},\overrightarrow{\beta_k}) \in E^2~ tq~ \forall t \in ]x_k,x_{k+1}[~ \varphi(t) = t.\overrightarrow{\alpha_k}+ \overrightarrow{\beta_k}$$
\end{de}
\begin{theo}
Soit f, application continue de [a,b] dans E, un K espace vectoriel normé (En particulier, on peut avoir E = $\mathbb{R}$ ou $\mathbb{C}$).\\
Alors, $\forall \varepsilon > 0$, $\exists \varphi:~ [a,b] \rightarrow E$, continue et affine par morceaux, telque :
$$\parallel f - \varphi \parallel_{\infty,[a,b]} \leq \varepsilon$$
Ce théorème est inutile en pratique.
\end{theo}
\section{Théorème d'approximation uniforme de Weierstrass}
\begin{theo}
Soit $f \in C([a,b],K)$, avec $K=\mathbb{R}~ ou~ \mathbb{C}$.\\
Alors, $\forall \varepsilon > 0,~ \exists P \in K[X]$ telque : 
$$\parallel f- P \parallel_{\infty,[a,b]} \leq \varepsilon$$ 
\end{theo}
Il existe, comme précédement, des variantes de ce théorème : 
\begin{theo}
Caractérisation séquentielle.\\
Pour tout f $\in C([a,b],K)$, $\exists (P_n)$ suite de polynomes $\in K[X]$ convergeant uniformement vers f sur [a,b].
\end{theo}
\begin{theo}
L'ensemble des fonctions polynomiales de [a,b] dans K est dense dans l'ensemble des fonctions continue de [a,b] dans K.
\end{theo}


\part{Suites, Séries}
\setcounter{chapter}{0}
\chapter{Suite de fonctions ou d'application - Convergence uniforme}
\section{Convergence simple}
\begin{de}
Soit A une ensemble (en général, A est un intervalle), et $(E,\parallel~\parallel)$ un K espace vectoriel normé.\\
Soit $(f_n)$ une suite d'application de $A\rightarrow E$.\\
On dit que cette suite converge simplement sur A si : 
$$\forall x \in A~ (f_n(x)) \mbox{ converge dans } (E,\parallel~\parallel)$$
Lorsque que c'est le cas, la limite de $f_n(x)$ quand n tend vers $\infty$ dépend a priori de x.\\
On peut donc définir une nouvelle application : 
$$f : A \rightarrow E$$
$$x \mapsto f(x) = \lim_{n\mapsto\infty}f_n(x)$$
On dit que f est la limite simple de $f_n$ sur A.
\end{de}
\section{Convergence uniforme d'une suite d'applications}
\begin{prop}
L'ensemble B(A,E) des applications bornées d'un ensemble A dans un K espace vectoriel normes (E,$\parallel~\parallel$) est un K sous espace vectoriel des applications de A dans E. De plus : 
$$B(A,E) \rightarrow \Re^+$$
$$f \mapsto \parallel f\parallel_{\infty,A}=\underset{A}Sup\parallel f\parallel$$
est une norme 
\end{prop}
\begin{de}
Soit $(f_n)$ une suite d'application de A, qui est un ensemble de E, un K espace vectoriel normé $(E,\parallel~\parallel)$, et f une application de A dans E.\\
On dit que $(f_n)$ converge uniformement vers f sur A si :
$$\exists n_0 \in N~ tq~ \forall n\geq n_0~ f_n-f \in B(A,E) \mbox{ (bornée) }$$
$$\parallel f_n-f\parallel_{\infty,A}\underset{n\mapsto\infty}\rightarrow0$$
En général, la première condition est évidente. Il faut donc se concentrer sur la deuxième propriété. En pratique, E=$\Re$ et A = I c $\Re$.
\end{de}
\begin{prop}
Soit $(f_n)$ une suite d'application de A, un ensemble, dans (E,$\parallel~\parallel$), un K espace vectoriel normée.\\
Si $(f_n)$ converge uniformement vers f sur A:
$$f : A \rightarrow E$$
Alors $(f_n)$ converge simplement vers f sur A.
\end{prop}
\begin{prop}
Cette propriété est utile pour prouver la non convergence uniforme.\\
Soit $(f_n)$ une suite d'application de A, un ensemble, dans (E,$\parallel~\parallel$) un K espace vectoriel normée.\\
Si $(f_n)$ convergent uniformement vers f, application de A dans E, alors :
$$\forall (x_n) \in A^N~ f_n(x_n) - f(x_n) \underset{n \rightarrow \infty}\rightarrow \overrightarrow{0}$$
De plus, $(x_n)$ n'est pas necessairement convergente et cette notion n'a meme pas de sens si il n'y a pas de distance de défini sur A.
\end{prop}
\section{Théorème classique sous les hypothèses de convergence uniforme}
\subsection{Théorème de continuité}
\begin{theo}
Soit $(f_n)$ une suite d'application de $I\mapsto K$, avec I un intervalle de $\Re$, convergent uniformement vers f, application de I dans K, sur tout le segment $[a,b]$ inclu dans I.\\
Si $\forall n \in N$, $f_n$ est continue en $x_0 \in I$, alors f est continue en $x_0$.\\
Si $\forall n \in N$, $f_n$ est continue sur I, alors f est continue sur I.
\end{theo}
\begin{gene}
Soit $(f_n)$ une suite d'application de A dans E, avec A une partie non vide d'un espace vectoriel normée, E un K espace vectoriel.\\
Si : \\
\begin{itemize}
 \item[$\rightarrow$] $\forall n \in N$ $f_n$ est continue sur A, respectivement en $a \in A$.\\
 \item[$\rightarrow$] $(f_n)$ converge uniformement vers $f : A \rightarrow E$ sur tout compact $\in A$\\
\end{itemize}
Alors, f est continue sur A, respectivement en a.
\end{gene}
\begin{de}
Un compact est une extension du théorème de Bolzano-Weistrass, qui dit que de toute suite convergente on peut extraire une suite croissante.
\end{de}
\subsection{Théorème d'interversion des limites, ou théorème de la double limite}
\begin{theo}
Soit $(f_n)$ une suite d'application de I dans K, avec I un intervalle non vide de $\Re$, convergent uniformement sur I, vers un application f de I dans K.\\
Soit $a \in \bar{\Re}$ un point de I ou une extrémité de I.\\
Si $\forall n \in N$ :
$$f_n(x) \underset{x \rightarrow a}\rightarrow l_n \in K$$
avec $x \in a$, alors : \\
\begin{itemize}
 \item[$\rightarrow$] $(l_n)$ converge vers une limite $l \in K$.\\
 \item[$\rightarrow$] $f(x) \underset{x \rightarrow a}\rightarrow l$, avec $x \in a$.\\
\end{itemize}
\end{theo}
\begin{de}
On dit qu'un espace vectoriel normée qu'il est complet si toute suite de cette espace vérifiant le critère de cauchy converge.\\
Les espaces vectoriel normée complet sont appellé espace de Banach
\end{de}
\begin{gene}
Soit $(f_n)$ une suite d'application de A dans E, avec A une partie non vide d'un espace vectoriel normée, et E un K espace vectoriel normée complet.\\
Si $(f_n)$ converge uniformement sur A vers une application f de A dans E, si $a \in \bar{A}$ c'est à dire si tout voisinage de a rencontre A, et si $\forall n \in N~ f_n(x) \underset{x \rightarrow a}\rightarrow l_n \in E$, avec $x \in A$, alors : \\
\begin{itemize}
 \item[$\rightarrow$] ($l_n$) converge vers $l \in E$.\\
 \item[$\rightarrow$] De plus $f(x) \underset{x \rightarrow a}\rightarrow l$, avec $x \in A$. On peut écrire ceci sous la forme suivante :  
$$\lim_{x \rightarrow a,~ x \in A} \lim_{n\rightarrow \infty} f_n(x) = \lim_{n\rightarrow \infty} \lim_{x \rightarrow a,~ x \in A} f_n(x)$$
\end{itemize}
On peut inverser les limites dans ce cas.
\end{gene}
\subsection{Théorème d'integration sur un segment sous les hypothèses de convergence uniforme}
\begin{theo}
Soit $(f_n)$ une suite d'application continue sur un segment [a,b], à valeur dans K, convergent uniformement sur [a,b] vers f, application de [a,b] dans K.\\
Alors : \\
\begin{itemize}
 \item[$\rightarrow$] f est continue sur [a,b]\\
 \item[$\rightarrow$] $\int_a^b f_n = \int_a^b f$ quand $n\rightarrow\infty$ \\
\end{itemize}
On peut écrire la deuxième conclusion sous la forme : 
$$\lim_{n\rightarrow\infty} \int_a^b f_n = \int_a^b \lim_{n\rightarrow\infty} f_n$$
\end{theo}
\begin{gene}
Ce théorème reste valable lorsqu'on remplace l'ensemble d'arrivé par un K espace vectoriel normée complet E. Ceci suppose d'avoir, au préalable défini $\int_a^b g$ pour g fonction de [a,b] dans E, un espace de banach, au moins continue par morceaux.
\end{gene}
\subsection{Thérorème de classe $C^1$}
\begin{theo}
Soit $(f_n)$ une suite d'application de I, un intervalle de $\Re$, dans K. On suppose que : \\
\begin{itemize}
 \item[$\rightarrow$] $\forall n \in N$ $f_n$ est $C^1$ sur I\\
 \item[$\rightarrow$] La suite $(f_n)$ converge simplement sur I vers une application f de I dans K\\
 \item[$\rightarrow$] La suite $(f'_n)$ converge uniformement sur tout segment [a,b] inclu I vers une application g de I dans \\
\end{itemize}
Alors : \\
\begin{itemize}
 \item[$\rightarrow$] f est de classe $C^1$ sur I\\
 \item[$\rightarrow$] f'=g\\
 \item[$\rightarrow$] $f_n$ converge uniformement vers f sur tout segment inclu dans I.\\
\end{itemize}
La conclusion 2 peut aussi s'écrire sous la forme : 
$$\dfrac{d}{dx} \lim_{n\mapsto\infty} f_n = \lim_{n\mapsto\infty} \dfrac{d}{dx} f_n$$
\end{theo}
\begin{gene}
Le théorème précédent reste vrai quand on remplace l'ensemble d'arrivé par un espace vectoriel normée complet, un espace de Barach  
\end{gene}
\subsection{Théorème de classe $C^p$}
\begin{theo}
Soit $(f_n)$ une suite d'application de I, un intervalle de $\Re$, dans K. Soit $p \in N^*$.\\
On suppose que : \\
\begin{itemize}
 \item[$\rightarrow$] $\forall n \in N$ $f_n$ est $C^p$ sur I
 \item[$\rightarrow$] Les suites $(f_n),(f'_n),...,(f_n^{(p-1)}$ converge simplement sur I vers une application f de I dans K
 \item[$\rightarrow$] La suite $(f_n^{(p)})$ converge uniformement sur tout segment [a,b] inclu I vers une application g de I dans K
\end{itemize}
Alors : \\
\begin{itemize}
 \item[$\rightarrow$] La limite simple f de la suite $(f_n)$ est $C^p$ sur I.
 \item[$\rightarrow$] $\forall k \in \textlbrackdbl\ 1,p \textrbrackdbl$ $f^{(k)}$ est la limite simple de la suite $(f_n^{(k)})$
 \item[$\rightarrow$] $\forall k \in \left[0,p\right]$ $(f_n^{(k)})$ converge uniformement vers $f^{(k)}$ sur tout segment de I.
\end{itemize}
\end{theo}
\begin{gene}
Ce théorème reste valable quand on remplace l'ensemble d'arrivé par un espace vectoriel complet.
\end{gene}
\subsection{Théorème de classe $C^{\infty}$}
\begin{de}
Une application de I, un intervalle inclu dans $\Re$, dans K est dite de classe $C^{\infty}$ sur I si $\forall p \in N^*$ elle est $C^p$ sur I.
\end{de}
On en déduit facilement du théorème $C^p$ précédent que si $(f_n)$ est une suite d'application de I dans K, et si : \\
\begin{itemize}
 \item[$\rightarrow$] $\forall n \in N$ $f_n$ est de classe $C^{\infty}$ sur I\\
 \item[$\rightarrow$] $\forall k \in N$ $(f_n^{(k)})$ converge uniformement sur tout segment inclu dans I\\
\end{itemize}
Alors :\\
\begin{itemize}
 \item[$\rightarrow$] La limite simple f de la suite $(f_n)$ est $C^{\infty}$ sur I.
 \item[$\rightarrow$] $\forall k \in N$ $(f_n^{(k)})$ converge uniformement vers $f^{(k)}$ sur tout segment inclu dans I.
\end{itemize}
Pour définir le classe $C^p$ dans le cas d'un espace E, on utilise le taux de variation
\section{Théorème de convergence monotone et convergence dominé}
\subsection{Théorème de convergence monotone}
\begin{de}
La suite $(f_n)$ d'application de I dans $\Re$ est dite monotone si $\forall x \in I$, $(f_n(x))$ est monotone
\end{de}
\begin{theo}
Soit I un intervalle quelconque de $\Re$ (pas forcément un segment), et $(f_n)$ une suite d'application continue par morceaux sur I, à valeur dans $\Re$, convergent simplement sur I vers f, application de I dans $\Re$, également continue par morceaux. Si : \\
\begin{itemize}
 \item[$\rightarrow$] La suite $(f_n)$ est monotone\\
 \item[$\rightarrow$] Si $f_0$ et f sont intégrable sur I\\
\end{itemize}
Alors : \\
\begin{itemize}
 \item[$\rightarrow$] $\forall n \in N$ $f_n$ est intégrable sur I
 \item[$\rightarrow$] $\int_I f_n \rightarrow \int_I f$ quand $n\rightarrow\infty$
\end{itemize}
La deuxieme conclusion peut s'écrire sous la forme : 
$$\lim_{n\rightarrow\infty} \int_I f_n = \int_I \lim_{n\rightarrow\infty} f_n$$
\end{theo}
\subsection{Théorème de la convergence dominée}
\begin{theo}
Soit I un intervalle quelconque inclu dans $\Re$, et $(f_n)$ une suite d'application de I dans K, continue par morceaux, convergent simplement vers f, application de I dans K, également continue par morceaux.\\
Si $\exists$ g, application de I dans $\Re^+$, continue par morceaux et intégrale sur I, telque (condition de domination): 
$$\forall n \in N,~ \forall x \in I,~ |f_n(x)|\leq g(x)$$
Alors :\\ 
\begin{itemize}
 \item[$\rightarrow$] Les $f_n$ et f sont intégrable sur I
 \item[$\rightarrow$] $\int_I f_n = \int_I f$ quand $n \rightarrow \infty$.
\end{itemize}
On peut écrire cette dernière conséquence sous la forme suivante : 
$$\lim_{n \rightarrow \infty} \int_I f_n = \int_I \lim_{n \rightarrow\infty} f_n$$
\end{theo}
\chapter{Série numérique}
\section{Définitions}
\subsection{Définitions générales}
\begin{de}
Soit ($u_n$) une suite à valeur dans K.\\
On appelle série de terme général $u_n$, et on note $\underset{n}\Sigma u_n$ cette nouvelle suite, de terme générale :
$$S_n = \sum_{k=0}^n u_k = u_0 +u_1 + ... +u_n$$
$S_n$ est appelé somme partiel de rang n de la série $\underset{n}\Sigma u_n$
\end{de}
\begin{de}
On dit que la série $\underset{n}\Sigma u_n$ converge si la suite ($S_n$) des sommes partiel converge dans K.\\
On note alors :
$$\sum_{k=0}^{\infty} u_k$$
cette limite.\\
En général, on note également S cette limite, et on appelle S somme de la série.
\end{de}
\begin{prop}
Si la série $\underset{n}\Sigma u_n$ converge, alors $u_n \rightarrow 0$ quand $ n \rightarrow +\infty$.
\end{prop}
\subsection{Reste d'ordre n}
\begin{de}
Si $\underset{n}\Sigma u_n$ converge, on peut définir $R_n$, le reste d'ordre n de la série $\underset{n}\Sigma u_n$, par :
$$R_n = \lim_{p \rightarrow \infty} \sum_{k=n+1}^p u_k$$
\end{de}
\begin{prop}
On obtient les relations suivantes : 
$$S_n + R_n = S$$
$$\lim_{n \rightarrow \infty} R_n \rightarrow 0$$
\end{prop}
\section{Quelques propriétés générales}
Dans tout ce chapitre, ($u_n$),($v_n$),... sont des suites à valeurs dans K
\begin{prop}
Supposons que $\underset{n}\Sigma u_n$ et $\underset{n}\Sigma v_n$ converge, alors la série de terme général $w_n = u_n + v_n$ converge, et on as :
$$\sum_{k = 0 }^{\infty} w_k = \left(\sum_{k = 0 }^{\infty} u_k\right) + \left(\sum_{k = 0 }^{\infty} v_k\right)$$
\end{prop}
\begin{prop}
Soit $\lambda \in K$.\\
Si $\underset{n}\Sigma u_n$ converge, il en est de même de la série de terme général $w_n = \lambda u_n$, et alors : 
$$\sum_{k = 0 }^{\infty} w_k = \lambda\left(\sum_{k = 0 }^{\infty} u_k\right)$$
\end{prop}
\begin{prop}
Soit $(z_n)$ est une suite à valeur dans C.\\
Si $(x_n) = Re(z_n)$ et $(y_n)= Im(z)$, donc $z_n = x_n + iy_n$, alors : 
$$(\underset{n}\sum z_n \mbox{ converge }) \Leftrightarrow (\sum_n x_n~ et~ \sum_n y_n \mbox{ converge })$$
\end{prop}
\begin{de}
On dit que la série de terme général $u_n$ converge absolument, ou que la suite ($u_n$) est sommable, si la série $\underset{n}\Sigma |u_n|$ converge.
\end{de}
\begin{theo}
L'absolu convergence de la série de terme général $u_n$ implique la converge de la série de terme général $u_n$.\\
Dans ce cas, on as : 
$$|\sum_{k=0}^{\infty} u_k| \leq \sum_{k=0}^{\infty} |u_k|$$
\end{theo}
\begin{prop}
Pour tous $n_0$ entier naturel, on peut modifier les $n_0$ premier termes de la suite ($u_n$) sans modifier la convergence de la série de terme général $u_n$.\\
On peut écrire ceci sous la forme :\\
Si $(v_n)$ vérifie à partir du rang $n_0$ $v_n = u_n$, alors :
$$(\underset{n}\Sigma v_n \mbox{ converge }) \Rightarrow (\underset{n}\Sigma u_n \mbox{ converge })$$
\end{prop}
\begin{prop}
Toute suite $(a_n)$ est une somme partielle d'une série $\underset{n} u_n$, à un terme constant près.
$$\forall n \in N ~a_n = \sum_{k=1}^n (a_k -a_{k-1}) + a_0$$
\end{prop}
\section{Séries à termes réels positifs (ou de signe constant)}
\begin{propfond}
Soit $(u_n)$ une suite à valeur réel positive.\\
Alors la suite des sommes partielle $S_n$ est croissante. Donc : 
\begin{itemize}
 \item[$\rightarrow$] Soit $(S_n)$ est majorée, et alors elle converge vers S = $Sup S_n$
 \item[$\rightarrow$] Soit $(S_n)$ n'est pas majorée, alors $S_n \underset{\infty}\rightarrow \infty$. On ecrit :
$$\sum_{k=0}^{\infty} u_k = +\infty$$
\end{itemize}
\end{propfond}
\begin{prop}
Si $(u_n) $ et $(v_n)$ sont à terme réels positifs telque 
$$0 \leq u_n \leq v_n$$
Alors : 
\begin{itemize}
 \item[$\rightarrow$] $\underset{n}\Sigma v_n$ converge $\Rightarrow$ $\underset{n}\Sigma u_n$ converge et :
$$0 \leq \Sigma_{k=0}^{\infty} u_k  \leq \Sigma_{k=0}^{\infty} v_k$$
 \item[$\rightarrow$] $\underset{n}\Sigma u_n$ diverge $\Rightarrow$ $\underset{n}\Sigma v_n$ diverge
\end{itemize}
\end{prop}
\begin{prop}
Soient $(u_n)$ et $(v_n)$ deux suites à termes réel positifs.\\
Si $u_n = \underset{\infty}= O(v_n)$ (Grand O), et si $\underset{n}\sum v_n$ converge, alors $\underset{n}\sum u_n$ converge
\end{prop}
\begin{prop}
Soient $(u_n)$ et $(v_n)$ deux suites à termes réels positifs, ou simplement de signe constant.\\
Si $u_n \underset{\infty}\sim v_n$, alors:
$$(\underset{n}\sum u_n \mbox{ converge })\Leftrightarrow(\underset{n}\sum v_n \mbox{ converge }) $$
On dit que $\underset{n}\sum u_n$ et $\underset{n}\sum u_n$ ont même nature.
\end{prop}
\subsection{Convergence des séries de Riemann}
\begin{de}
On appelle série de Riemann les séries de terme général, avec $\alpha \in \Re$ :
$$u_n = \dfrac{1}{n^{\alpha}}$$
\end{de}
\begin{prop}
Soit ($u_n$) la suite de terme général :
$$u_n = \dfrac{1}{n^{\alpha}}$$
Alors :
$$(\underset{n}\sum u_n \mbox{converge}) \Leftrightarrow (\alpha > 1)$$
\end{prop}
\begin{prop}
La comparaison série-intégrale (Qui consiste à encadrer la somme partiel, ou le reste partiel, par des intégrales de la fonction) précédent permet d'obtenir un équivalent simple, quand le cas de série de terme général $u_n = \dfrac{1}{n^{\alpha}}$ :
\begin{itemize}
 \item{$\rightarrow$} De $S_n$ dans le cas divergent
 \item{$\rightarrow$} De $R_n$, dans le cas convergent.
\end{itemize}
\end{prop}

\subsection{Règle de Riemann}
\begin{prop}
Soit $(u_n)$ suite à valeur réelle ou complexe.\\
Si $\exists \alpha > 1$ telque $n^{\alpha}u_n \underset{\infty}\rightarrow 0$, alors $\underset{n}\sum |u_n|$ converge. On dit aussi que la suite $(u_n)$ est sommable
\end{prop}
\begin{prop}
Soit $(u_n)$ suite à valeur réelle.\\
Si $nu_n \underset{\infty}\rightarrow \infty$, alors $\underset{n}\sum u_n$ diverge. On a meme : 
$$\sum_{k=0}^{\infty} u_n = \infty$$
\end{prop}
\subsection{Série de Bertrand}
\begin{de}
Ce sont les séries de terme général :
$$u_n = \dfrac{1}{n^{\alpha}.ln(n)^{\beta}}$$
\end{de}
\begin{prop}
Une série de Bertrand converge si et seulement si : 
$$\alpha > 1~ ou~ \alpha=1~ et~ \beta>1$$
\end{prop}
\subsection{Propriété de Cauchy}
Soit u, la fonction définie $\forall n_0 \in N$, par : 
$$u : \left[n_0,\infty \right[ \rightarrow \Re$$
$$t \rightarrow u(t)$$
u est une application continue par morceaux et monotone. On obtient que :
$$(\sum_n u(n) \mbox{ converge }) \Leftrightarrow (\int_{n_0}^{\infty} u \mbox{ converge })$$
On dit que la série et l'intégrale ont même nature.
\subsection{Règle de d'Alembert}
\begin{prop}
Soit $(u_n)$ une suite de réels telque $\forall n \Leftrightarrow \geq n_0$, $u_n > 0$, et telque :
$$\dfrac{u_{n+1}}{u_n}\underset{\infty}\rightarrow l\in \bar{\Re^+}$$
Alors : 
\begin{itemize}
 \item{$\rightarrow$} Si l > 1, alors la série de terme général $u_n$ diverge grossièrement
 \item{$\rightarrow$} Si l < 1, alors la série converge
\end{itemize}
\end{prop}
\section{Séries alternées}
\begin{de}
Soit $(u_n)$ une suite de réels.\\
On dit que $(u_n)$ est alternée si elle est du type :
$$\forall n \in N,~ u_n = (-1)^n.a_n$$
Ou du type : 
$$\forall n \in N,~ u_n = (-1)^{n+1}.a_n$$
Avec $(a_n)$, suite de réels positive. Dan ces deux cas : 
$$a_n = |u_n|$$
La série $\underset{n}\sum u_n$ est dites alternée sur la suite $(u_n)$ est alternée.
\end{de}
\subsection{Critère de Leibniz ou Critère spécial des séries alternée}
\begin{enon}
Soit $\underset{n}\sum u_n$ une série alternée.\\
Si : 
\begin{itemize}
 \item{$\rightarrow$} La suite $(u_n) \underset{\infty}\rightarrow 0$
 \item{$\rightarrow$} (|$u_n$|) est une suite décroissante
\end{itemize}
Alors : 
\begin{itemize}
  \item{$\rightarrow$} La série $\underset{n} \sum u_n$ converge
 \item{$\rightarrow$} $\forall n \in N$, le signe de $R_n$ est le signe de son premier terme, et : 
$$|R_n|\leq |u_{n+1}|$$
\end{itemize}
Ce résultat est valable aussi si on considère que S, la limite de la série, est le reste d'ordre $-1$.
\end{enon}
\begin{prop}
 Sous les hypothèses de la série alternée, la somme S est comprise entre deux sommes partielle consécutive, $S_n$ et $S_{n+1}$, et ceci $\forall n \in N$
\end{prop}
\begin{prop}
Si la série n'est alternée, et ne vérifie le fait que la suite ($|u_n|$) n'est décroissante qu'a partir d'un rang $n_0$, alors la série converge toujours, et le regle sur le reste reste valable, à partir du rang $n_0$.
\end{prop}
Lors d'exercice, le point le plus souvent difficile est de démontrer que la suite des valeurs absolu décroit.
\section{Quelques espaces remarquables}
\begin{prop}
L'ensembles des suites $(u_n) \in K^{N}$ telque $\underset{n}\sum u_n$ converge est un sous espace vectoriel de $K^N$. (Stable par addition et par multiplication par un scalaire). 
\end{prop}
\subsection{Ensemble $l^1(K)$}
\begin{de}
L'ensemble $l^1(K)$ est l'ensemble des suites sommables à valeurs dans K. Cette ensemble est un sous espace vectoriel de l'espace vectoriel vu ci dessus.
\end{de}
\begin{prop}
Soit $(u_n)$ une suite à valeur dans K.\\
Si la suite $(u_n)$ est sommable, la serie $\underset{n}\sum u_n$ converge aussi (L'absolu convergence implique la convergence). $l^1(K)$ est donc inclu dans l'espace vectoriel vu au début de ce chapitre.
\end{prop}
\begin{prop}
Supposons que u=$(u_n)$ et $v=(v_n)$ soient sommable. L'inégalité suivante : 
$$0 \leq |u_n + v_n|\leq|u_n|+|v_n|$$
Montre que u+v est également sommable.
\end{prop}
\begin{prop}
Soit $\lambda$ un scalaire.
$$u \in l^1(K) \Rightarrow \lambda.u \in l^1(K)$$
\end{prop}
\subsection{Ensemble $l^2(K)$}
\begin{de}
L'ensemble $l^2(K)$ est l'ensemble des suites de carrées sommable, c'est à dire telque :
$$\sum_n |u_n^2|$$
converge.\\
On montre que l'ensemble précédent est inclu dans cette ensemble.
\end{de}
\section{Sommation par paquets ou associativité de la sommation}
\begin{de}
Soit $(u_n)$ une suite à valeur dans K telque $\underset{n}\sum u_n$ converge.\\
Soit :
$$S = \sum_{k=0}^{\infty} u_k$$
Donnons nous une application $\phi$ défini par :
$$\phi : N \rightarrow N$$
telque $\phi$ soit une application strictement croissante.\\
Une telle application étant donnée, définisoons à partir de $(u_n)$ et de $\phi$ une nouvelle suite $(v_n)$ de terme général : 
$$v_n = u_{\phi_{(n-1)+1}}+...+u_{\phi_n}$$
\end{de}
\begin{prop}
Avec les notations précédentes, le fait que la série de terme général $u_n$ converge implique que la série de terme général $v_n$ converge, et vers S aussi.
\end{prop}




\chapter{Sommation des relations de comparaison}
\section{Cas sommable}
Dans tous ce chapitre, les relations de sommation sont des relations consernant $\textbf{les restes}$
$$\sum_{k=n+1}^{\infty} u_k$$
\begin{hypo}
Dans l'ensemble de cette fiche, $(u_n)$ et $(v_n)$ sont deux suites définies à partir d'un certain rang $N_0$.
\begin{itemize}
 \item{$\rightarrow$} $(u_n)$ est une suite à valeur C.
 \item{$\rightarrow$} $(v_n)$ est une suite à valeur dans $\Re^+$, ou à valeur réelles et de signe constant.
\end{itemize}

\end{hypo}
\subsection{Négligabilité}
\begin{theo}
Avec les hypothèses précédentes :\\
Supposons que $(v_n)$ soit sommable et que  :
$$u_n \underset{\infty}\ll v_n$$
Alors ($u_n$) est également sommable, et, $\forall n \geq N_0$ : 
$$\sum_{k=n+1}^{\infty} u_k \ll \sum_{k=n+1}^{\infty} v_k$$
\end{theo}
On peut aussi enoncer ce théorème, mais avec les notations de Landau.
\begin{theo}
Sous les hypothèses de départ : \\
Supposons que $(v_n)$ soit sommable et que : 
$$u_n \underset{\infty}= o(v_n) $$
Alors $(o(v_n))$ est sommable et 
$$\sum_{k=n+1}^{\infty} o(v_k) \ll o(\sum_{k=n+1}^{\infty} v_k)$$
\end{theo}
\begin{prop}
Soit $(a_n)$ et $(b_n)$ deux suites. Si : 
$$a_n \underset{\infty}\ll b_n \Leftrightarrow \forall \varepsilon>0 \exists N \in N~ tq~ \forall n \geq N,~ |a_n|\leq\varepsilon |b_n|$$
\end{prop}
\subsection{Domination}
\begin{theo}
Avec les hypothèses précédentes :\\
Supposons que $(v_n)$ soit sommable et que  :
$$u_n \underset{\infty}\preccurlyeq v_n$$
Alors ($u_n$) est également sommable, et, $\forall n \geq N_0$ : 
$$\sum_{k=n+1}^{\infty} u_k \preccurlyeq \sum_{k=n+1}^{\infty} v_k$$
\end{theo}
On peut aussi enoncer ce théorème, mais avec les notations de Landau.
\begin{theo}
Sous les hypothèses de départ : \\
Supposons que $(v_n)$ soit sommable et que : 
$$u_n \underset{\infty}= O(v_n) $$
Alors $(O(v_n))$ est sommable et 
$$\sum_{k=n+1}^{\infty} O(v_k) \ll O(\sum_{k=n+1}^{\infty} v_k)$$
\end{theo}
\begin{prop}
Soit $(a_n)$ et $(b_n)$ deux suites. Si : 
$$a_n \underset{\infty}\preceq b_n \Leftrightarrow \exists M \in \Re{+},~ \exists N \in N,~ tq~ \forall n \geq N,~ |a_n|\leq M|b_n|$$
\end{prop}
\subsection{Equivalence}
Sous les hypothèses du préambule, en particulier sur le fait que $(v_n)$ soit de signe constant à partir d'un certain rang $n_0$ : \\
Si : 
$$u_n \underset{\infty}\sim v_n \mbox{ et } (v_n) \mbox{ est sommable }$$
Alors : 
$$(u_n) \mbox{ est sommable et } \sum_{k=n+1}^{\infty}u_n \underset{\infty}\sim \sum_{k=n+1}^{\infty}v_n$$
\section{Cas non sommable}
Dans tout ce chapitre, les relations de sommation concerne les $\textbf{sommes partielle}$ : 
$$\sum_{n_0}^n u_k$$
\subsection{Négligabilité}
\begin{theo}
Avec les hypothèses précédentes, en particulier $(v_n)$ de signe constant à partir d'un certain rang :\\
Supposons que $(v_n)$ ne soit pas sommable et que  :
$$u_n \underset{\infty}\ll v_n$$
Alors: 
$$\sum_{k=n_0}^{n} u_k \ll \sum_{k=n_0}^{n} v_k$$
Mais nous n'avons pas d'information sur la sommabilité ou la non sommabilité de $(u_n)$.
\end{theo}
On peut aussi enoncer ce théorème, mais avec les notations de Landau.
\begin{theo}
Sous les hypothèses de départ : \\
Supposons que $(v_n)$ ne soit pas sommable et que : 
$$u_n \underset{\infty}= o(v_n) $$
Alors :
$$\sum_{k=n_0}^{n} o(v_k) \ll o(\sum_{k=n_0}^{n} v_k)$$
\end{theo}
\subsection{Domination}
\begin{theo}
Avec les hypothèses précédentes :\\
Supposons que $(v_n)$ ne soit pas sommable et que  :
$$u_n \underset{\infty}\preccurlyeq v_n$$
Alors $\forall n \geq N_0$ : 
$$\sum_{k=n_0}^{n} u_k \preccurlyeq \sum_{k=n_0}^{n} v_k$$
\end{theo}
On peut aussi enoncer ce théorème, mais avec les notations de Landau.
\begin{theo}
Sous les hypothèses de départ : \\
Supposons que $(v_n)$ ne soit pas sommable et que : 
$$u_n \underset{\infty}= O(v_n) $$
Alors : 
$$\sum_{k=n_0}^{n} O(v_k) \ll O(\sum_{k=n_0}^{n} v_k)$$
\end{theo}
\subsection{Equivalence}
Sous les hypothèses du préambule, en particulier sur le fait que $(v_n)$ soit de signe constant à partir d'un certain rang $n_0$ : \\
Si : 
$$u_n \underset{\infty}\sim v_n \mbox{ et } (v_n) \mbox{ n'est pas sommable }$$
Alors, $(u_n)$ n'est pas sommable et : 
$$\sum_{k=n_0}^{n}u_n \underset{\infty}\sim \sum_{k=n_0}^{n}v_n$$

\chapter{Séries d'applications}
\section{Définitions}
\begin{de}
Soit $(u_n)$ une suite d'application d'un ensemble A a valeurs dans un K espace vectoriel normée E, munie de la norme $\parallel~\parallel$.\\
En pratique, A est un intervalle inclu dans $\Re$, et E sera presque toujours K, et donc dans ce cas : 
$$\parallel~\parallel = |~ |$$
À partir de cette suite d'application de A dans E, on peut construire une nouvelle suite d'application $(S_n)$ : 
$$S_n : A \rightarrow E$$
$$x \mapsto S_n(x) = \sum_{k=0}^n u_k(x)$$
Cette nouvelle suite $(S_n)$ s'appelle la série d'application de terme général $u_n$, et se note : $\underset{n} \sum u_n$
\end{de}
\subsection{Convergence simple}
\begin{de}
On dit que $\underset{n} \sum u_n$ converge simplement sur A si la suite d'application $(S_n)$ converge simplement sur A. Lorsque que c'est le cas, on peut définir une nouvelle application :
$$S : A\rightarrow E$$
$$x \mapsto \sum_{k=0}^{\infty} u_k(x)$$
On dit que S, application de A dans E, est la limite simple de la serie. On peut également, lorsqu'il y a convergence simple de la série sur A, définir l'application $R_n$ : 
$$R_n : A \rightarrow E$$
$$x \rightarrow \sum_{k=n+1}^{\infty} u_k(x)$$
Et nous avons les relations suivantes :
$$S = S_n + R_n$$
$$\forall x \in A~ R_n(x) \rightarrow \overrightarrow{0} \in E$$
On dit aussi, pour la dernière relation, que $(R_n)$ converge simplement vers $\tilde{0}$ sur A
\end{de}
\subsection{Convergence absolue}
\begin{de}
On dit que la série d'application $\underset{n}\sum u_n$ converge absolument sur A si :
$$\forall x \in A~ \sum_n \parallel u_n(x)\parallel \mbox{ converge }$$ 
\end{de}
\begin{prop}
Si E = K, la convergence absolue de $\underset{n}\sum u_n$ sur A implique ma convergence simple de cette série d'application sur A.\\
Cette propriété reste vrai sur E est un K espace vectoriel normée complet, dit de Banach 
\end{prop}
\subsection{Convergence uniforme}
\begin{de}
On dit que la série d'application $\underset{n} \sum u_n$ converge uniformement sur A si la suite d'application $(S_n)$ converge uniformement sur A.\\
Lorsque c'est le cas, la série d'application converge simplement sur A.\\
Notons S la limite simple de $(S_n)$. Dire que la série d'application converge sur A signifie que : 
$$\parallel S_n - S \parallel_{\infty,A} \underset{n \rightarrow \infty}\rightarrow 0 \Leftrightarrow \parallel R_n \parallel_{\infty,A} \underset{n \rightarrow \infty}\rightarrow 0$$
Et qu'il existe un rang a partir du quel $R_n$ est bornée.\\
On peut dire ceci de la façon suivante aussi. La serie d'application de terme général $u_n$ converge uniformement sur A : 
\begin{itemize}
 \item[$\rightarrow$] La serie converge simplement sur A (Pour l'existance de S, donc de $(R_n)$)
 \item[$\rightarrow$] La suite $(R_n)$ converge uniforment sur A.
\end{itemize}
\end{de}
\begin{prop}
Si $(S_n)$ converge uniformement sur A vers S, alors :
$$\forall (x_n) \mbox{ à valeur dans A }, S_n(x_n) - S(x_n) \underset{n}\rightarrow \overrightarrow{0} \in E$$
On écrit de même : 
$$\forall (x_n) R_n(x_n) \rightarrow \overrightarrow{0} \in E$$
Cette propriété est utile pour montrer la non convergence uniforme.\\
\end{prop}
\subsection{Convergence normale, ou convergence au sens de Weirstrass}
\begin{de}
On dit qu'une série d'application $\underset{n} \sum u_n$ converge normalement sur A si $u_n$ est bornée sur A, au moins à partir d'un certain indice $n_0$, au quel cas on peut définir sa norme infini, et si 
$$\sum_n \parallel u_n \parallel_{\infty,A} \mbox{ converge }$$
\end{de}
\begin{prop}
Si la série d'application de terme général $u_n$ converge normalement sur A, alors :\\
$$\mbox{La série converge absolument sur A}$$
et si E est complet ( en particulier E=K), alors
$$ \mbox{La série converge uniformement sur A}$$
\end{prop}
\section{Théorème classique sous hypothèse de convergence uniforme}
\subsection{Théorème de continuité}
\begin{theo}
Soit ($u_n$) une suite d'application de I, un intervalle inclu dans $\Re$, dans K.\\
Si $\forall n \in N$ $u_n$ est continue en $x_0 \in I$, respectivement continue sur I, et si $\underset{n} \sum u_n$ converge uniformement sur I, alors :
$$S = \sum_{k=0}^{\infty} u_k \mbox{ est continue en }x_0\mbox{, respectivement sur I}$$
\end{theo}
\begin{theo}
De plus, consient que la continuité est une propriété locale, on peut énoncer le théorème précédent de la façon suivante :\\
Soit $(u_n)$ une suite d'application continue sur I, un intervalle de $\Re$, et à valeur dans K.\\
Si la série de fonctions converge uniformement sur tout segment inclu dans I, alors S est continue sur I
\end{theo}
\begin{de}
Un sous ensemble d'un K espace vectoriel normée E est dit compact si de toute suite $(x_n)$ à valeur dans K, on peut extraire une sous-suite $(x_{\phi(n)}$, avec $\phi$ application strictement croissante de N dans N, convergent vers un élement de K.
\end{de}
\begin{gene}
Soit $(u_n)$ une suite d'application de A dans E, avec A un sous ensemble non vide d'un K espace vectoriel normée E', et E un K espace vectoriel normée.\\
Si : \\
\begin{itemize}
 \item[$\rightarrow$] $\forall n \in N$ $u_n$ est continue sur A\\
 \item[$\rightarrow$] $\underset{n} \sum u_n$ converge uniformenet sur tout compact inclu dans A\\
\end{itemize}
Alors S est continue sur A
\end{gene}
\subsection{Théorème d'inversion de la limite et de la somme}
\begin{theo}
Soit $(u_n)$ une suite d'application de I, un intervalle inclue dans $\Re$, dans K, et $a\in I$, ou a une extrémité, ou a = $\pm \infty$.\\
Si :\\
\begin{itemize}
 \item[$\rightarrow$] $\forall n \in N$ : 
$$\lim_{x \rightarrow a,x\in A} u_n(x) = l_n \in K$$
 \item[$\rightarrow$] $\underset{n} \sum u_n(x)$ converge uniforment sur I\\
\end{itemize}
Alors la série $\underset{n} \sum l_n$ converge et : 
$$S(x) \underset{x\rightarrow a,x\in A}\rightarrow \sum_{k=0}^{\infty} l_k$$
On peut aussi ecrire ceci sous la forme : 
$$\lim_{x \rightarrow a,x\in A} \sum_{k=0}^{\infty}u_k(x) = \sum_{k=0}^{\infty}\lim_{x \rightarrow a,x\in A} u_k(x)$$
\end{theo}
\begin{gene}
Ce théorème reste valable pour des applications $u_n$ de [a,b] dans E, un K espace vectoriel complet, à condition d'avoir défini ce qu'est l'intégrale de $u_n$ dans cette espace.
\end{gene}
\subsection{Théorème de classe $C^1$}
\begin{theo}
Soit $(u_n)$ une suite d'application de I dans K, avec I un intervalle de $\Re$.\\
Si : \\
\begin{itemize}
 \item[$\rightarrow$] $\forall n \in N$, $u_n$ est de classe $C^1$ sur I.\\
 \item[$\rightarrow$] $\underset{n} \sum u_n$ converge simplement sur I \\
 \item[$\rightarrow$] $\underset{n} \sum u'_n$ converge uniformement sur I, ou sur tout segment de I \\
\end{itemize}
Posons :
$$S = \sum_{k=0}^{\infty} u_k$$
Alors :\\
\begin{itemize}
 \item[$\rightarrow$] S est $C^1$ sur I\\
 \item[$\rightarrow$] On peut exprimer S', la dérivé de S de la façon suivante (dérivation terme à terme) : 
$$S' = \sum_{k=0}^{\infty} u'_k$$
 \item[$\rightarrow$] La suite ($u_n$) converge uniforment sur tout segment inclue dans I.
\end{itemize}
\end{theo}
\begin{gene}
Ce théorème reste valable pour des applications $u_n$ de [a,b] dans E, un K espace vectoriel complet.
\end{gene}
\subsection{Théorème de classe $C^p$}
\begin{theo}
Soit $(u_n)$ une suite d'application de I dans K, avec I un intervalle de $\Re$, et $p\in N^*$\\
Si : \\
\begin{itemize}
 \item[$\rightarrow$] $\forall n \in N$, $u_n$ est de classe $C^p$ sur I.\\
 \item[$\rightarrow$] $\forall k \in \left[0,p-1\right]~ \underset{n} \sum u^{(k)}_n$ converge simplement sur I \\
 \item[$\rightarrow$] $\underset{n} \sum u^{(p)}_n$ converge uniformement sur I, ou sur tout segment de I \\
\end{itemize}
Posons :
$$S = \sum_{k=0}^{\infty} u_k$$
Alors :\\
\begin{itemize}
 \item[$\rightarrow$] S est $C^1$ sur I\\
 \item[$\rightarrow$] $\forall k \in \left[0,p-1\right]$, on peut exprimer $S^{(k)}$, la dérivé k-ème de S de la façon suivante (dérivation terme à terme) : 
 $$S^{(k)} = \sum_{t=0}^{\infty} u^{(k)}_t$$
 \item[$\rightarrow$] $\forall k \in \left[0,p\right]$, la suite ($u^{(k)}_n$) converge uniforment sur tout segment inclue dans I.
\end{itemize}
\end{theo}
\begin{gene}
Ce théorème reste valable pour des applications $u_n$ de [a,b] dans E, un K espace vectoriel complet.
\end{gene}
\subsection{Théorème de classe $C^{\infty}$}
\begin{theo}
Soit $(u_n)$ une suite d'application de I dans K, avec I un intervalle de $\Re$, et $p\in N^*$\\
Si : \\
\begin{itemize}
 \item[$\rightarrow$] $\forall n \in N$, $u_n$ est de classe $C^{\infty}$ sur I.\\
 \item[$\rightarrow$] $\forall p \in N~ \underset{n} \sum u^{(p)}_n$ converge uniformement sur I ou sur tout segment de I\\
\end{itemize}
Posons :
$$S = \sum_{k=0}^{\infty} u_k$$
Alors :\\
\begin{itemize}
 \item[$\rightarrow$] S est $C^{\infty}$ sur I\\
 \item[$\rightarrow$] $\forall p \in N$, on peut exprimer $S^{(p)}$, la dérivé p-ème de S de la façon suivante (dérivation terme à terme) : 
 $$S^{(p)} = \sum_{t=0}^{\infty} u^{(p)}_t$$
\end{itemize}
\end{theo}
\begin{gene}
Ce théorème reste valable pour des applications $u_n$ de [a,b] dans E, un K espace vectoriel complet.
\end{gene}

\chapter{Série entière}
\section{Définitions, rayon de convergence}
\subsection{Défintions}
\begin{de}
On appelle série entière une série d'application $\underset{n}\sum u_n$ où les $u_n$ sont des monomes à coefficient réels ou complexe, définis sur $\Re$ ou C, par : 
$$u_n : K \rightarrow K$$
$$ t \mapsto a_n t^n$$
En général, $u_n$ est une application d'un corps dans ce même corps.
\end{de}
\subsection{Abus de notation}
Par abus de notation, car on assimile de faite série d'application et série numérique, on note souvent $\underset{n} \sum a_n.x^n$, respectivement $\underset{n} \sum a_n.z^n$, ou lieu de $\underset{n}\sum u_n$.
\subsection{Rayon de convergence}
\begin{de}
Étant donnée une série entière $\underset{n} \sum a_n.z^n$, c'est a dire étant donnée une suite ($a_n$) de complexe, on appelle rayon de convergence de cette série entière, noté parfois $\rho(\underset{n} \sum a_n.z^n)$ : 
$$\rho(\underset{n} \sum a_n.z^n) = Sup\left\lbrace r \in \Re_+ ~/~ (|a_n|r^n) \mbox{ soit majorée }  \right\rbrace $$
\end{de}
\begin{prop}
Dans le cas ou l'ensemble défini ci-dessus et non majorée, on convient que $\rho(\underset{n} \sum a_n.z^n) = + \infty$
\end{prop}
\begin{prop}
Soit $\underset{n} \sum a_n.z^n$ une série entière, avec $(a_n)$ suite de complexe, et R = $\rho(\underset{n} \sum a_n.z^n)$.
\begin{itemize}
 \item[$\rightarrow$] Si R = $+ \infty$ : La série converge simplement dans C et converge normalement sur tout disque ferme $\bar{B}(0,\alpha)$, avec $\alpha \geq 0$, et meme converge normalement sur tout compact inclu dans C.
 \item[$\rightarrow$] Si R est un réel > 0 : La série converge sur tout disque ouvert B(0,R). La série converge normalement sur tout disque fermé $\bar{B}(0,\alpha)$ inclu dans B(O,R), et meme converge normalement sur tout compact inclu dans B(O,R). 
\end{itemize}
\end{prop}
\begin{voc}
B(O,R) s'appelle le disque ouvert de convergence de la série entière $\underset{n} \sum a_n.z^n$.
\end{voc}
\subsection{Lemme d'Abel}
Soit $z_0 \neq 0$. Si $(|a_n|.|z_0|^n)$ est majorée, $\forall \alpha \in [0,z_0[$, la série $\underset{n} \sum a_n.z^n$ converge normalement sur $\bar{B}(O,\alpha)$
\section{Propriétés utilse au calcul du rayon de convergence}
Dans tout ce chapitre, $(a_n)$ et $(b_n)$ désigne des suites de compact, et $\lambda$ désigne un complexe non nuls
\begin{prop}
\begin{itemize}
 \item[$\rightarrow$] $\rho(\underset{n} \sum a_n.z^n) = \rho(\underset{n} \sum |a_n|.z^n)$
 \item[$\rightarrow$] $\rho(\underset{n} \sum \lambda a_n.z^n) = \rho(\underset{n} \sum a_n.z^n)$
 \item[$\rightarrow$] Si $z_0 \in C$, tq $\underset{n} \sum a_n.z_0^n$ converge, alors $\rho(\underset{n} \sum a_n.z^n) \geq |z_0|$. Il en est de meme si la suite ($a_n.z^n$) est bornée ou si la suite converge 
 \item[$\rightarrow$] Si $z_0 \in C$, tq $\underset{n} \sum a_n.z_0^n$ diverge, alors $\rho(\underset{n} \sum a_n.z^n) \leq |z_0|$. Il en est de meme si la suite ($a_n.z^n$) n'est pas bornée ou si la suite ne tend pas vers 0. 
\end{itemize}
\end{prop}
\begin{prop}
Si $\forall n \geq n_0$, on a : 
$$|a_n| \leq |b_n|$$
Alors : 
$$\rho(\sum_n a_n.z^n) \geq \rho(\sum_n b_n.z^n) $$
\end{prop}
\begin{prop}
Si on a : 
$$|a_n| \underset{n \rightarrow +\infty}\sim |b_n|$$
Alors : 
$$\rho(\sum_n a_n.z^n) = \rho(\sum_n b_n.z^n) $$
\end{prop}
\subsection{Règle de d'Alembert pour les séries entières}
Soit $\underset{n} \sum a_n.z^n$ une série entière.\\
Si $\forall n \geq n_0 $ |$a_n \neq 0$|, et si : 
$$\dfrac{|a_{n+1}|}{|a_n|} \rightarrow l \in \Re_+ \cup \left\lbrace + \infty\right\rbrace $$
Alors : 
$$\rho(\sum_n a_n.z^n) = \dfrac{1}{l}$$
avec les conventions suivantes : 
$$l=0 \Rightarrow \rho(\sum_n a_n.z^n) = +\infty$$
$$l = +\infty \Rightarrow \rho(\sum_n a_n.z^n) = 0$$
\section{Somme et produit de deux séries entières}
\subsection{Somme de deux séries entières}
\begin{prop}
Si $\underset{n} \sum a_n.z^n$ et $\underset{n} \sum b_n.z^n$ converge, alors $\underset{n} \sum (a_n+b_n).z^n$ converge et nous avons : 
$$\sum_{n=0}^{+\infty} (a_n+b_n).z^n = \sum_{n=0}^{+\infty} a_n.z^n + \sum_{n=0}^{+\infty} b_n.z^n $$
\end{prop}
\begin{corr}
Si : 
$$R_A = \rho(\sum_n a_n.z^n)$$
$$R_B = \rho(\sum_n b_n.z^n)$$
$$R_S = \rho(\sum_n (a_n+b_n).z^n)$$
Alors on obtient que : 
$$R_S \geq min(R_A,R_B)$$
De plus, si $R_A \neq R_B$, alors :
$$R_S = min(R_A,R_B)$$
\end{corr}
\subsection{Produit de deux séries entières}
\begin{prop}
Si $\underset{n} \sum a_n$ et $\underset{n} \sum b_n$ sont deux séries numériques, à valeurs dans $\Re$ ou dans C, absolument convergente, et si : 
$$c_n = \sum_{k=0}^n a_k.b_{n-k}$$
Alors : \\
\begin{itemize}
 \item[$\rightarrow$] $\underset{n} \sum c_n$ est absolument convergente.\\
 \item[$\rightarrow$] Nous avons l'égalité suivante : 
$$\left(\sum_{n=0}^{\infty} a_n\right)\left(\sum_{n=0}^{\infty} b_n\right) = \sum_{n=0}^{\infty} c_n $$
\end{itemize}
\end{prop}
\begin{prop}
Si $z \in C$ est telque $\underset{n} \sum a_n.z^n$ et $\underset{n} \sum b_n.z^n$ converge absolument, alors : 
\begin{itemize}
 \item[$\rightarrow$] $\underset{n} \sum c_n.z^n$ est absolument convergente.\\
 \item[$\rightarrow$] Nous avons l'égalité suivante : 
$$\left(\sum_{n=0}^{\infty} a_n.z^n\right)\left(\sum_{n=0}^{\infty} b_n.z^n\right) = \sum_{n=0}^{\infty} c_n.z^n $$
\end{itemize}
\end{prop}
\begin{de}
La suite ($c_n$) défini par : 
$$c_n = \sum_{k=0}^n a_k.b_{n-k}$$
est parfois appelé produit de cauchy des suites ($a_n$) et ($b_n$)
\end{de}
\subsection{Propriétés sur les exponentiels complexes}
\begin{prop}
Soit z et z' deux complexes. Nous avons la propriété suivante :
$$e^z.e^{z'} = e^{z+z'}$$
\end{prop}
\begin{corr}
D'après la propriété précédente, on obtient que :
$$\forall z \in C~ e^z \neq 0~ et~ (e^z)^{-1} = e^{-z}$$
$$\forall z \in C,~ \forall n \in Z,~ (e^z)^n = e^{nz}$$
\end{corr}
\subsection{Continuité}
\begin{prop}
Si $R = \rho(\underset{n} \sum a_n.z^n)$, alors : 
$$z \mapsto \sum_{n=0}^{+\infty} a_n.z^n$$ 
est continue sur B(0,R)
\end{prop}
\section{Classe $C^{\infty}$}
Soit $\underset{n} \sum a_n.x^n$ une série entière de rayon R>0 (on peut avoir $R = \infty$).\\
Alors :
$$S : x \mapsto \sum_{n=0}^{\infty} a_n.x^n$$
est défini au moins sur $]-R,R[$. Cet intervalle est appelé intervalle de convergence.
\begin{prop}
Nous avons les propriétés suivantes : 
$$\rho(\sum_n n.a_n.x^{n-1}) = \rho(\sum_n a_n.x^{n})$$
$$\rho(\sum_n (n+1).a_{n+1}.x^{n}) = \rho(\sum_n a_n.x^{n})$$
\end{prop}
\begin{theo}
Soit $\underset{n} \sum a_n.x^n$ une série entière de rayon R > 0 (On peut avoir $R = +\infty$).\\
Alors sa somme : 
$$S : ]-R,R[ \rightarrow K$$
$$x \mapsto \sum_{n=0}^{+\infty} a_n.x^n$$
est $C^{\infty}$ sur l'intervalle de convergence, et les dérivés $S^{(k)}$ s'obtiennent à l'aide d'une dérivation terme à terme : 
$$\forall x \in ]-R,R[,~ S^{(p)}(x) = \sum \dfrac{n!}{(n-p)!}a_n.x^{n-p}$$
\end{theo}
\begin{corr}
On obtient le corrolaire suivant : 
$$\forall x \in ]-R,R[~ S(x) = \sum_{n=0}^{\infty} \dfrac{S^{(n)}(0)}{n!}x^n$$
\end{corr}
\begin{corr}
Soient $\underset{n} \sum a_n.x^n$ et $\underset{n} \sum b_n.x^n$ deux séries entière de rayon $R_a$ et $R_b$ strictement positif. Si : 
$$\forall x \in ]-r,r[~\underset{n} \sum a_n.x^n = \underset{n} \sum b_n.x^n$$
avec : 
$$0<r\leq min(R_a,R_b)$$
Alors, on en déduit que : 
$$a_n = b_n$$
La conclusion reste valable en supposant seulement que l'égalité des sommes est vérifié pour tout x appartenant à un intervalle de longeur > 0.
\end{corr}
\begin{corr}
Si $\underset{n} \sum a_n.x^n$ est une série entière de rayon R > 0, alors : 
$$\rho(\sum_{n=0}^{\infty}\dfrac{a_n}{n+1}x^{n+1}) = R$$
Et : 
$$x \mapsto \sum_{n=0}^{\infty} \dfrac{a_n}{n+1} x^{n+1}$$
est une primitive de 
$$x \mapsto \sum_{n=0}^{\infty} a_n x^{n}$$
sur $]-R,R[$
\end{corr}
\section{Fonctions développable en série entière}
Dans ce chapitre, on se limite à la variable réelle.
\begin{de}
Soit f fonction de $\mathbb{R}$ dans $\mathbb{R}$ et $x_0 \in \mathbb{R}$.\\
On dit que f est développable en série entière en $x_0$, ou au voisinage de $x_0$, si il existe un r > 0 et une série entière $\underset{n} \sum a_n.x^n$, c'est à dire $\exists (a_n) \in \mathbb{R}^{\mathbb{N}}$, de rayon R $\geq$ r tq :
$$\forall x \in ]-r,r[~ f(x) = \sum_{n=0}^{\infty} a_n.(x-x_0)^n$$
\end{de}
\begin{prop}
Soit f une fonction décomposable en série entière en $x_0$. Son développement est le développement de Taylor en $x_0$, c'est à dire : 
$$\forall x \in ]x_0-r,x_0+r[~ f(x) = \sum_{n=0}^{\infty} \dfrac{f^{(n)}(x_0)}{n!}(x-x_0)^n$$
\end{prop}
\begin{corr}
On peut obtenir un développement limité en $x_0$ de f, à l'ordre n, en tronquant le développement en série entière.
$$f(x) \underset{x \rightarrow x_0}= \sum_{k=0}^n \dfrac{f^{(k)}(x_0)}{k!}.(x-x_0)^k + o((x-x_0)^n)$$
\end{corr}
\subsection{Quelques développpement en série entière classique}
Nous avons les développement classique suivant : 
\begin{itemize}
 \item[$\rightarrow$] $\forall z \in \mathbb{C}$ tq $|z|<1$ : $$\dfrac{1}{1-z} = \sum_{k=0}^{\infty} z^k$$
 \item[$\rightarrow$] $\forall z \in \mathbb{C}$ tq $|z|<1$ : $$\dfrac{1}{1+z} = \sum_{k=0}^{\infty} (-1)^k.z^k$$
 \item[$\rightarrow$] $\forall x \in [-1,1[$ :  $$ln(1-x) = -\sum_{k=1}^{\infty} \dfrac{x^{k}}{k}$$
 \item[$\rightarrow$] $\forall x \in ]-1,1]$ : $$ln(1+x) = \sum_{k=1}^{\infty} (-1)^{k+1}\dfrac{x^{k}}{k}$$
 \item[$\rightarrow$] $\forall x \in \mathbb{R}$ : $$e^x = \sum_{k=1}^{\infty} \dfrac{x^k}{k!}$$
 \item[$\rightarrow$] $\forall z \in \mathbb{C}$ : $$e^z = \sum_{k=1}^{\infty} \dfrac{z^k}{k!}$$
\item[$\rightarrow$] $\forall x \in \mathbb{R}$ : $$ch(x) = \sum_{k=0}^{\infty} \dfrac{x^{2k}}{(2k)!}$$
\item[$\rightarrow$] $\forall x \in \mathbb{R}$ : $$sh(x) = \sum_{k=0}^{\infty} \dfrac{x^{2k+1}}{(2k+1)!}$$
\item[$\rightarrow$] $\forall x \in \mathbb{R}$ : $$cos(x) = \sum_{k=0}^{\infty} (-1)^k \dfrac{x^{2k}}{(2k)!}$$
 \item[$\rightarrow$] $\forall x \in \mathbb{R}$ : $$sin(x) = \sum_{k=0}^{\infty} (-1)^k \dfrac{x^{2k+1}}{(2k+1)!}$$
 \item[$\rightarrow$] $\forall \alpha \in \mathbb{R}-\mathbb{N}, \forall x \in ]-1,1[$ : $$(1+x)^{\alpha} = \sum_{k=0}^{\infty} a_k.x^k$$
avec : 
$$\begin{cases}
  a_0 = 1 \\
  a_k = \dfrac{\alpha.(\alpha-1)...(\alpha-k+1)}{k!}
  \end{cases}
$$
\end{itemize}
\subsection{Développement en série entière des fractions rationnelles}
\begin{prop}
Soit f une fraction rationnelle, f = $\dfrac{P}{Q}$, avec $(P,Q) \in \mathbb{C}[X]^2$, premiers entre eux. Si 0 n'est pas un pôle de f, alors f est un développable en série entière en 0 sur B(0,r), avec :
$$r = \underset{\alpha \in Z(Q)}\min |\alpha|$$
De plus, la série entière en question, qui est la série de Taylor de f, est de rayon r.
\end{prop}
\section{Extension à $\mathbb{C}$ des fonctions trigonométrique}
Nous avons les développement en série entière classique pour les fonctions exp, sh, ch dans $\mathbb{C}$ et de cos et sin dans $\mathbb{R}$. En observant que les rayons de convergence de ces séries sont infini, on peut donc obtenir les fonctions cos et sin dans $\mathbb{C}$, en considérent le développement en série entière identique, mais avec une variable complexe.
\subsection{Lien entre trigonométrie circulaire et trigonométrie hyperbolique}
On montre, en utilisant le développement en série entière, que $\forall z \in \mathbb{C}$ : 
$$ch(iz) = cos(z) $$
$$sh(iz) = i.sin(z)$$
En remplacant z par iz, on obtient : 
$$cos(iz) = ch(z)$$
$$sin(iz) = i.sh(z)$$

\chapter{Compléments sur les séries}
\section{Intégration des séries de fonctions}
\begin{theo}
Théorème d'interversion du signe $\sigma$ et du signe intégrale.\\
Soit $\underset{n} u_n$ une série d'application $u_n$ : I $\rightarrow$ K, avec I un intervalle quelconque, continue par morceaux et convergent simplement sur I. On suppose que chaque $u_n$ est intégrale sur I.\\
Si $\underset{n} \int_I |u_n|$ converge, alors :
\begin{itemize}
 \item[$\rightarrow$] $\underset{n} \sum \int_I u_n$ converge
 \item[$\rightarrow$] $S = \underset{n = 0 }\sum^{\infty} u_n$, supposé continue par morceaux sur I, est intégrable sur I.
 \item[$\rightarrow$] $\int_I S$ = $\underset{n=0}\sum^{\infty} \int_I u_n$
\end{itemize}
La dernière conséquence est bien une propriété d'interversion du signe $\sigma$ et du signe intégrale.
\end{theo}
\section{Rudiment sur les séries doubles}
\begin{de}
Étant donnée un ensemble I, on appelle famille à valeur dans K et indexé par I, et on note $(u_i)_{i \in I}$, une application :
$$I \overset{u}\rightarrow K$$
$$i \rightarrow u_i$$
\end{de}
Soit $(u_{ij})$ une famille de complexe. ($u_{ij}$) est une application :
$$u: \mathbb{N}\times\mathbb{N} \rightarrow K$$
$$(i,j) \mapsto u_{ij}$$
\begin{de}
On dira que :
$$\sum_{i = 0}^{\infty} \sum_{j=0}^{\infty} u_{ij}$$
existe si et seulement si : 
\begin{itemize}
 \item[$\rightarrow$] $\forall i \in \mathbb{N} \underset{j}\sum u_{ij}$ converge
 \item[$\rightarrow$] En supposant la condition précédente vérifié, et en notant :
$$S_i = \sum_{j=0}^{\infty} u_{ij}$$ 
$\underset{i}\sum S_i$ converge
\end{itemize}
\end{de}
\begin{theo}
Soit ($u_{ij}$) une famille à valeur dans $\mathbb{C}$. Si $\underset{i=0}\sum^{\infty} \underset{j=0}\sum^{\infty} |u_{ij}|$ existe, alors on peut permutter les sommmes (elles existent), et on a égalité entre les différentes sommmes doubles.
\end{theo}

\chapter{Séries d'endomorphismes et de matrices}
\section{Définitions et propriétés générales}
\begin{de}
Soit E un K espace vectoriel normé de dimension finie n et $(f_k)_{n \in \mathbb{N}}$ une suite d'endomorphisme de E.\\
On appelle série de terme général $f_k$ la nouvelle suite $(S_k)_{k \in \mathbb{N}}$ défini par : 
$$S_k = \sum_{i=0}^k f_i$$
On note $\underset{k}\sum f_k = (S_k)_{k \in \mathbb{N}}$.\\
On dit que la série converge lorsque $S_k$ à une limite dans $\mathcal{L}(E)$, et on note alors : 
$$S = \sum_{i=0}^{\infty} f_i$$
cette limite. Dans ce cas, nous pouvons définir aussi : 
$$R_k = \lim_{p \rightarrow \infty} \sum_{i=k+1}^p f_i = \sum_{i=k+1}^{\infty} f_i$$
qui est le reste de la série. On as trivialement, lorsque la série converge (Pour pouvoir définir le reste) : 
$$S = S_k + R_k$$
et 
$$R_k \underset{k \rightarrow \infty}\rightarrow \tilde{0}$$
De plus, comme on a supposé E de dimension finie ici, $\mathcal{L}(E)$ est aussi de dimension finie, donc la convergence ne dépend pas de la norme choisie sur $\mathcal{L}(E)$.\\
Nous avons aussi une définition analogue dans le cas d'une suite de matrice.
\end{de}
\begin{prop}
Si B est une base de E et si $A_k = mat_B(f_k)$ alors : 
$$\sum_{i=0}^k A_i = mat_B(\sum_{i=0}^k f_k)$$
et $\underset{k} \sum f_k$ converge si et seulement si $\underset{k}\sum A_k$ converge, et dans ce cas : 
\begin{align*}
 \sum_{k=0}^{\infty} A_k &= mat_B(\sum_{k=0}^{\infty} f_k) \\
  \sum_{i=k+1}^{\infty} A_i &= mat_B(\sum_{i=k+1}^{\infty} f_i) \\
\end{align*}
\end{prop}
\begin{de}
Si $\parallel~\parallel'$ est une norme quelconque sur $\mathcal{L}(E)$ et si $(f_k)_{k \in \mathbb{N}}$ est une suite d'endomorphisme de E, on dit que $\underset{k}\sum f_k$ converge absolument au sens de $\parallel~\parallel'$ si :
$$\underset{k}\sum \parallel f_k\parallel' \text{ converge }$$
Nous avons une définition analogue pour les matrices.
\end{de}
\begin{prop}
Si $\underset{k} \sum f_k$ converge absolument, alors $\underset{k} \sum f_k$ converge dans $\mathcal{L}(E)$. Nous avons la même propriété pour les matrices. 
\end{prop}
\subsection{Exemple : La série géométrique}
\begin{prop}
S'il existe une norme sous-multiplicative $\parallel~\parallel$ sur $\mathcal{L}(E)$ telque $\parallel f\parallel < 1$ alors $\underset{k} \sum f^k$ converge
\end{prop}
\section{Exponentielle d'endomorphisme ou de matrice}
\subsection{Propriétés et définitions}
\begin{de}
Si $f \in \mathcal{L}(E)$, avec E un K espace vectoriel normé de dimension finie N, avec $K = \mathbb{R}$ ou $\mathbb{C}$, alors $\underset{n} \sum \dfrac{f^n}{n!}$ converge dans $\mathcal{L}(E)$ et on note : 
$$e^f = \sum_{k=0}^{\infty} \dfrac{f^n}{n!}$$
De même, si $A \in \mathcal{M}_N(K)$, alors $\underset{n} \sum \dfrac{A^n}{n!}$ converge et on note : 
$$e^A = \sum_{k=0}^{\infty} \dfrac{A^n}{n!}$$
\end{de}
\begin{prop}
Si $u \in C^1(I,K)$, avec I intervalle $\subset \mathbb{R}$, alors, avec $A \in \mathcal{M}_N(K)$ : 
\begin{align*}
 I &\overset{f}\rightarrow \mathcal{M}_N(K)\\
 t &\mapsto e^{u(t)A}
\end{align*}
est $C^1$ sur I et :
\begin{align*}
\forall t \in I~ f'(t)&=u'(t)Ae^{u(t)A} \\
		      &=u'(t)e^{u(t)A}A	
\end{align*}
\end{prop}
\section{Applications de l'exponentielle de matrices à la résolution d'un système différentiel linéaire à coefficient constant}
\begin{de}
Un système différentielle linéaire est un système du type : 
$$\begin{cases}
   x'_1 = a_{11}(t)x_1 + \dots + a_{1n}(t)x_n + b_1(t) \\
  \vdots\\
  x'_n = a_{n1}(t)x_1 + \dots + a_{nn}(t)x_n + b_n(t) \\
  \end{cases}
$$
avec les $a_{ij}$ et les $b_i$ des applications données de I dans K=$\mathbb{R}$ ou $\mathbb{C}$, avec $I \subset \mathbb{R}$.\\
Les $x_j$ sont des applications inconnu que l'on cherche dans l'ensemble $C^1(I,K)$. On remarque que le nombre d'inconnu est égale au nombre d'équations.\\
Ce système est dit à coefficients constant lorsque les $a_{ij}$ sont des applications constantes. Le système est dit homogène lorsque les $b_i$ sont toutes des applications nulles.\\
Dans la suite, on va supposer que le système est à coefficients constants. Notons : 
$$A = (a_{ij})\in \mathcal{M}_n(K)$$
\begin{align*}
 B : I &\rightarrow K^n \\
     t &\mapsto \begin{pmatrix}
                 b_1(t) \\
		 \vdots \\
		 b_n(t)
                \end{pmatrix}
\end{align*}
\begin{align*}
 X : I &\rightarrow K^n \\
     t &\mapsto \begin{pmatrix}
                 x_1(t) \\
		 \vdots \\
		 x_n(t)
                \end{pmatrix}
\end{align*}
Avec ces notations, on obtient que $x_1, \dots, x_n$ sont solution sur I du système différentielle précédent si et seulement si : 
$$X \in C^1(I,K),~ \forall t \in I~ \dfrac{dX}{dt} = AX + B(t)$$
\end{de}
\begin{theo}
Nous avons un théorème de Cauchy-Lipschitz pour les systèmes :\\
Soit $\dfrac{dX}{dt} = A(t)X + B(t)$
, avec A et B défini comme précédement, continue sur un intervalle $I \subset \mathbb{R}$, c'est à dire que toutes leurs composantes sont continues sur I. Alors $\forall X_0 = \begin{pmatrix}
                                                                                                                                                                                x_{01} \\
						     \vdots \\
						     x_{0n}
\end{pmatrix} \in K^n$ et $\forall t_0 \in I$, il existe une unique solution de (S) vérifiant la condition initiale $X(t_0)=X_0$, c'est à dire : 
$$\begin{cases}
   x_1(t_0)=x_{01} \\
   \vdots \\
   x_n(t_0)=x_{0n} \\
  \end{cases}
$$ 
\end{theo}
\begin{coro}
Sous les hypothèses précédente concernant I (intervalle $\subset \mathbb{R}$) et $A \in C(I,\mathcal{M}_n(K))$, l'ensemble des solutions de ($S_0$) sur I est a valeur dans $K^n$ est un K espace vectoriel de dimension n. 
\end{coro}
\begin{coro}
Sous les hypothèses ci-dessus concernant I,A et b, l'ensemble des solutions de (S) sur I à valeurs dans $K^n$ est un espace affinie de direction vectorielle ($S_0$). C'est à dire si u est une solution particuliere de (S) : 
$$Sol_{(S)}(I,K^n)= \left\lbrace U + X,~ X \in Sol_{(S_0)}(I,K^n)\right\rbrace  $$
\end{coro}


\part{Équations différentielles}
\setcounter{chapter}{0}
\chapter{Équations différentielles}
\section{Équations différentielle linéaire du $1^{er}$ ordre}
\begin{de}
Une équation différentielle linéaire du $1^{er}$ ordre est une équation du type : 
$$\alpha(t).y' + \beta(t).y = \gamma(t)~ (E)$$
Avec : 
$$\begin{cases}
   \alpha,\beta,\gamma \in C(I,K),~ I \mbox{ Intervalle} c \mathbb{R} \\ 
   y \in C^1(I,K) \mbox{ est une application incconue }
  \end{cases}
$$
On dit que (E) est résolue, par rapport à y, sur I lorsque $\alpha$ ne s'annule par sur I. Dans ce cas : 
$$(E) \Leftrightarrow y' + a(t)y = b(t)$$
\end{de}
\begin{de}
On appelle équation homogène associé à (E), ou équation sans second membre, noté ($E_0$) : 
$$y' + a(t)y = 0$$
\end{de}
\begin{theo}
Théorème de Cauchy-Lipschitz linéaire du première ordre :\\
Avec les notations et hypothèses précédentes, si (E) est résolue sur I, alors : 
$$\forall t_0 \in I~ \forall y_0 \in K,~ \exists! y \in Sol(E) ~ tq~ y(t_0) = y_0$$ 
\end{theo}
\begin{coro}
Toujours avec les mêmes notation et hypothèses : 
\begin{itemize}
 \item[$\rightarrow$] Sol($E_0$) est un K espace vectoriel de dimension 1
 \item[$\rightarrow$] Sol(E) est un K espace affine de direction Sol($E_0$)
\end{itemize}
\end{coro}
\section{Equations différentielle linéaire du second ordre}
\begin{de}
Une équation différentielle linéaire du $2^{nd}$ ordre est une équation du type : 
$$\alpha(t).y'' + \beta(t).y' + \gamma(t).y = \delta(t)~ (E)$$
Avec : 
$$\begin{cases}
   \alpha,\beta,\gamma,\delta \in C(I,K),~ I \mbox{ Intervalle} c \mathbb{R} \\ 
   y \in C^2(I,K) \mbox{ est une application incconue }
  \end{cases}
$$
On dit que (E) est résolue, par rapport à y, sur I lorsque $\alpha$ ne s'annule par sur I. Dans ce cas : 
$$(E) \Leftrightarrow y'' + a(t)y' + b(t)y = h(t)$$
\end{de}
\begin{theo}
Sous les hypothèses précédent, en particulier le fait que (E) est résolue sur I, alors, $\forall t_0 \in I,~ \forall(y_0,y_0')\in K^2, \exists! y\in Sol(E)$ tq :
$$\begin{cases}
   y(t_0) = y_0 \\
   y'(t_0) = y'_0 \\
  \end{cases}
$$
\end{theo}
\begin{coro}
Sous les hypothèses précédentes : 
$$\begin{cases}
   Sol(E_0) = \mbox{ K espace vectoriel de dimension 2} \\
   Sol(E) = \mbox{ K espace affine de direction vectoriel } Sol(E_0) \\
  \end{cases}
$$
\end{coro}
\subsection{Cas Particuliers}
Considérons une équation différentielle linéaire de $2^{nd}$ ordre à coefficiant constant : 
$$y'' + ay' + by = h(t)~ (E)$$
$$\begin{cases}
   \mbox{ a et b sont des constantes indépendente de la variable du corps K} \\
   h \in C(I,K)
  \end{cases}
$$
Dans ce cas, on résout l'équation caractéristique associé : 
$$r^2 + ar + b = 0$$
\subsubsection{Si K = $\mathbb{C}$}
\paragraph{Si l'équation caractérisitique a deux solutions}
Notons $r_1,r_2$ ces deux solutions disctinct.
$$Sol(E_0) = Vect(t \mapsto e^{r_1.t},t \mapsto e^{r_2.t})$$
\paragraph{Si l'équation caractérisitique a racine double}
Notons r cette solution : 
$$Sol(E_0) = Vect(t \mapsto t.e^{r.t},t \mapsto e^{r.t})$$
\subsubsection{Si K = $\mathbb{R}$}
\paragraph{Si l'équation caractérisitique a deux solutions}
Notons $r_1,r_2$ ces deux solutions disctinct.
$$Sol(E_0) = Vect(t \mapsto e^{r_1.t},t \mapsto e^{r_2.t})$$
\paragraph{Si l'équation caractérisitique a racine double}
Notons r cette solution : 
$$Sol(E_0) = Vect(t \mapsto t.e^{r.t},t \mapsto e^{r.t})$$
\paragraph{Si l'équation caractéristique à deux racines non réelles}
Notons : 
$$r_1 = \alpha + i\beta$$
$$r_2 = \alpha - i\beta$$
$$Sol(E_0) = Vect(t \mapsto e^{\alpha.t}.cos(\beta.t),t \mapsto e^{\alpha.t}.sin(\beta.t))$$
\subsection{Solutions de (E)}
Si l'on possède une solution particulier de ($E_0$), on peut résoudre, si cette solution ne s'annule pas sur I, completement l'équation (E) par la méthode de variation de la constante.
\section{Équations différentielle (non linéaire) du $1^{er}$ ordre, résolue}
\begin{de}
Un ouvert de $\mathbb{B}$ est défini par : 
$$(\mbox{ U est un ouvert }) \Leftrightarrow ( \forall M \in U,~ \exists r > 0~ tq~ B(M,r) c U )$$
Avec : 
$$B(M,r) = \left\lbrace P \in \mathbb{R}^2 / \parallel \overrightarrow{MP} \parallel < \overrightarrow{r}\right\rbrace $$
Le fait que U soit ouvert ne dépend pas de la norme choisie, car toutes les normes sont équivalente entre dimension finies.
\end{de}
Cette équation est une équation du type : 
$$y' = f(y,t) $$
avec f une fonction de $\mathbb{R}^2$ dans $\mathbb{R}$, de classe $C^1$ sur un ouvert U de $\mathbb{R}^2$.
\begin{de}
Une solution de (E) est une application $y \in C^1(I,\mathbb{R})$, avec I intervalle de $\mathbb{R}$ telle que : 
$\begin{cases}
 	\forall t \in I,~ (t,y(t)) \in U \\
        \forall t \in I,~ y'(t) = f(t,y(t)) \\
\end{cases}$
\end{de}
\begin{de}
Une solution y est dite maximale si on ne peut pas la prolonger en une solution de (E) sur un intervalle I' $\supsetneqq$ I.\\
C'est à dire qu'il n'existe pas de solutionz de (E) sur I' $\supsetneqq$ I telque :
$$y = z_{|I}$$
\end{de}
\begin{theo}
Théorème de Cauchy-Lipshitz pour les équations différentielles (non linéaire) du $1^{er}$ ordre résolue.\\
Si $f \in C^1(U,\mathbb{R})$ avec U ouvert de $\mathbb{R}^2$, $\forall(t_0,y_0) \in U$, l'équation (E) : y' = f(t,y) admet une solution maximale et une seule, noté $y \in C^1(I,\mathbb{R})$, avec I un intervalle $\ni$ $t_0$, telque :
$$y(t_0) = y_0$$
De plus, pour une telle solution maximale, I est ouvert. Si $z \in C^1(J,\mathbb{R})$ est une solution de (E) sur J $\ni$ $t_0$ vérifiant z($t_0$) = $y_0$, alors : 
$$\begin{cases}
   J c I \\
   z = y_{|J}
  \end{cases}
$$
Autrement dit, toutes solutions de (E) se prolonge en une unique solution maximale.
\end{theo}
\begin{coro}
 Si $z_1$ et $z_2$ sont des solutions de (E) sur des intervalles $J_1$ et $J_2$ contenant $t_0$ et si : 
$$z_1(t_0) = z_2(t_0)$$
Alors : 
$$z_1{|J_{1}\wedge J_{2}} = z_2{|J_{1}\wedge J_{2}}$$
\end{coro}
\subsection{Équation à variable séparable}
\begin{de}
C'est une équation différentielle du $1^{er}$ ordre équivalente à une équation du type :
$$f(y).y' = g(t)~ (E)$$
Avec f et g des fonctions de $\mathbb{R}$ dans $\mathbb{R}$.\\
Si I est un intervalle sur lequel f ne s'annulent pas :
$$(E) \Leftrightarrow y' = \dfrac{g(t)}{f(y)} = F(t,y)$$
Si f est continue sur I, et ne s'annule pas sur I, et si g est continue sur J $\in \mathbb{R}$, alors F précédement définie est continue sur $JxI$. On le démontre à l'aide de fonctions composées.
\end{de}
\subsection{Énoncé simplifié du théorème de Cauchy-Lipschitz pour les équations différentielle du $1^{er}$ ordre autonomes}
\begin{de}
Une équation différentielle est dites autonome si elle est indépendente de t. C'est à dire si c'est une équation du type : 
$$y' = f(y)~ (E)$$
\end{de}
Si $f \in C^1(J,\mathbb{R})$, avec J intervalle ouvert de $\mathbb{R}$.\\
$\forall t_0 \in \mathbb{R}$, $\forall y_0 \in J$, il existe une solution maximale et une seule de (E), noté $y \in C^1(I,\mathbb{R})$ telle que $y(t_0)=y_0$.\\
De plus, l'intervalle de définition d'une telle solution maximale est un ouvert $\ni t_0$.\\
Toute solution z de (E) sur un intervalle $I' \in I$ vérifiant le meme condition initiale est la restriction sur I' de y.\\
\begin{coro}
Si $z_1$ et $z_2$ sont deux solutions de (E) sur des intervalles $J_1$ et $J_2$, vérfiant une même condition initiale en $t_0 \in J_1 \cap J_2$, alors $z_1$ et $z_2$ coincident sur $J_1 \cap J_2$
\end{coro}
$y' = f(y)$ est un cas particulier de y' = F(t,y) avec :
$$F : \mathbb{R}^2 \rightarrow \mathbb{R}$$
$$(t,y) \mapsto f(y)$$
On peut donc appliquer directement le théorème de Cauchy-Lipschitz en considérant comme ouvert U $\mathbb{B}\times J$
\section{Équation différentielle (non linéaire) du $2^{nd}$ ordre, résolue}
C'est une équation du type : 
$$y'' = f(t,y,y')~ (E)$$
\begin{theo}
Théorème de Cauchy-Lipschitz pour une équation différentielle (non linéaire) du $2^{nd}$ ordre.\\
Si f $\in C^{1}(U,\mathbb{R})$, avec U un ouvert de $\mathbb{R}^2$, alors :
$\forall{(t_0,y_0,y'_0)} \in U$, il existe une unique solution maximale de (E), noté y $\in C^2(I,\mathbb{R})$, avec I un intervalle de $\mathbb{R}$, c'est à dire que : 
$$\begin{cases}
   \forall t \in I,~ (t,y(t),y'(t)) \in U \\
   \forall t \in I,~ y''(t) = f(t,y(t),y'(t)) \\
  \end{cases}
$$
Vérifiant les conditions initiale suivante :
$$\begin{cases}
   y(t_0) = y_0 \\
   y'(t_0) = y'_0 \\
  \end{cases}
$$
De plus, l'intervalle I de définition d'une telle solution maximale est ouvert.\\
Toute solution z de (E) sur un intervalle J $\ni t_0$ et vérifiant les mêmes conditions initiales est la restrictions de J sur y.\\
$$\begin{cases}
   J c I \\
   \forall t \in J~ z(t)=y(t) \\
  \end{cases}
$$
\end{theo}
\begin{coro}
Si deux solutions $z_1$ et $z_2$ de (E) sur $J_1$ et $J_2$ vérifiant la même condition initiale en $t_0$, alors $z_1$ et $z_2$ coincident sur $J_1 \cap J_2$ 
\end{coro}
\subsection{Énoncé simplifié pour les équations différentielles autonomes}
C'est une équation du type : 
$$y'' = f(y,y')$$
Si f $\in C^1(W,\mathbb{R})$, W un ouvert de $\mathbb{R}^2$, alors $\forall t_0 \in \mathbb{R}$, $\forall (y_0,y'_0) \in W$, (E) admet une solution et une seule, maximale, de (E), noté $y \in C^1(I,\mathbb{R})$, I intervalle $\in t_0$, c'est à dire :
$$\begin{cases}
   \forall t \in I,~ (y(t),y'(t)) \in W \\
   \forall t \in I,~ y''(t) = f(y(t),y'(t))\\
  \end{cases}
$$
vérifiant la condition initiale.\\
De plus, l'intervalle de définition d'une telle solution maximale est ouvert. Toute solution de (E) se prolonge en une unique solution maximale
\begin{coro}
Si $z_1$ et $z_2$ sont deux solutions de (E), sur $J_1$ et sur $J_2$, $\ni t_0$, et si ces solutions vérifient les même conditions initiale, alors : 
$$\forall t \in J_1\cap J_2,~ z_1(t) = z_2(t)$$
\end{coro}
\section{Système différentielles (non linéaire) autonome du $1^{er}$ ordre, de deux équations à deux inconnus}
\begin{de}
Ce sont les systèmes différentielles du type : 
$$(S) : \begin{cases}
    x' = f(x,y) \\
    y' = g(x,y) \\
  \end{cases}
$$
Le système est dit autonome car la variable t dont dépendent les deux fonctions x et y ne figurent pas dans les équations.
\end{de}
Dans ce chapitre, nous faisons les hypothèses suivantes : f et g sont deux applications de $C^1(U,\mathbb{R})$, avec U un ouvert de $\mathbb{R}^2$, c'est à dire que f et g admettent des dérivées partielles du $1^{er}$ ordre, continues sur U.
\begin{de}
On dit que x et y sont des solutions de (S) sur I $c \mathbb{R}$ si x et y $\in C^1(I,\mathbb{R})$ telque : 
$$\forall t \in I, (x(t),y(t)) \in U$$
$$\forall t \in I \begin{cases}
                   x'(t) = f(x(t),y(t)) \\
		   y'(t) = g(x(t),y(t)) \\
                  \end{cases}
$$
Dans ce cas, x et y sont solutions de (S) sur I c $\mathbb{R}$.
\end{de}
\subsection{Ecriture synthétique du système (S)}
Notons X la fonction suivante :
$$X : \mathbb{R} \rightarrow \mathbb{R}^2$$
$$t \mapsto X(t)$$
Avec : 
$$X(t) = \begin{pmatrix}
  x(t) \\
  y(t) \\
\end{pmatrix}
$$
X est $C^{1}(I,\mathbb{R}^2)$ si x et y sont elle même $C^1$ sur I, avec I intervalle de $\mathbb{R}$. Dans ce cas : 
$$\forall t \in I~ X'(t) = \begin{pmatrix}
  x'(t) \\
  y'(t) \\
\end{pmatrix}$$
D'autre part, définissons : 
$$F : \mathbb{R}^2 \rightarrow \mathbb{R}^2$$
$$(x,y) \mapsto F(x,y)$$
Avec : 
$$F(x,y) = \begin{pmatrix}
  f(x,y) \\
  g(x,y) \\
\end{pmatrix}
$$
De plus, F est de classe $C^1$ de U sur $\mathbb{R}^2$, c'est à dire admet des dérivées partielles continues sur U si et seulement si f et g sont $C^{1}(U,\mathbb{R})$.\\
Dans ce cas, on obtient les dérivées partielles de F par dérivée composantes par composantes.
\begin{de}
F est appelé champs de vecteur de classe $C^1$ sur l'ouvert U de $\mathbb{R}^2$
\end{de}
A l'aide de ces notations, on obtient que x et y sont solutions de (S) sur I si et seulement si : 
$$X \in C^{1}(I,\mathbb{R}^2)$$
$$\forall t \in I,~ X(t) \in U$$
$$\forall t \in I,~ X'(t) = F(X(t))$$
On résume donc les trois conditions en disant que X est solution sur I de l'équation différentielle vectorielle : 
$$X' = F(X)~ (E)$$
\begin{de}
Une solution X de (E) est dite maximale si elle n'est pas prolongable en une solution de (E) sur un intervalle I' $\supsetneqq$ I
\end{de}
\begin{theo}
Théorème de Cauchy-Lipschitz.\\
Sous les hypothèses précédente, c'est à dire essentiellement $F \in C^{1}(U,\mathbb{R}^2)$, avec U un ouvert de $\mathbb{R}^2$, l'équation (E) :
$$X' = F(X)$$
admet $\forall t_0 \in \mathbb{R}$ et tout $X_0 = (x_0,y_0) \in U$ une unique solution maximale X : 
$$X : I \rightarrow \mathbb{R}^2$$
$$t \mapsto X(t)$$
Avec : 
$$X(t) = \begin{pmatrix}
  x(t) \\
  y(t) \\
\end{pmatrix}$$
De classe $C^1$ sur I telle que X($t_0$) = $X_0$. De telles solutions maximales de (E) s'appelle des courbes intégrale de champs F. De plus : 
\begin{itemize}
 \item[$\rightarrow$] L'intervalle de définition d'un solution maxe de (E) est ouvert
 \item[$\rightarrow$] Toutes solutions de (E) sur un intervalle J est la restriction à J d'une solution maximale
\end{itemize}
\end{theo}
\begin{coro}
Si $Z_1$ et $Z_2$ sont deux solutions de (E) défini sur $J_1$ et $J_2$, vérifiant les mêmes conditions initiales, alors : 
$$\forall t \in J_1 \cap J_2~ Z_1(t) = Z_2(t) $$
\end{coro}
\part{Réduction d'endomorphismes}
\setcounter{chapter}{0}
\chapter{Réduction des endomorphismes et des matrices - Première Partie}
\section{Système linéaire}
Considérons un système linéaire (s) à n équations et à n inconnes. On peut écrire (S) sous sa forme matricielle : 
$$AX = B$$
Avec : 
A la matrice des coefficiant, X la matrice des inconnue et B la matrice des seconds membres.\\
\begin{de}
Un système linéaire admet une unique solution si et seulement si A est inversible, donc si det(A) $\neq$ 0 ou rang (A) = n.\\
Dans ce cas, on dit que le système linéaire est inversible, ou que c'est un système de Cramer. L'unique solution est donnée par : 
$$\Omega = A^{-1}.B$$
\end{de}
\section{Détermination de l'inverse de A}
\subsection{Méthode directe}
Pour déterminer l'inverse de A, avec A une matrice inversible, on peut utiliser la formule suivante (Voir fiche de révision Sup) : 
$$A^{-1} = \dfrac{1}{det(A)}.^t Com(A)$$
Cependant, la complexité de cette méthode ( c'est à dire le nombre d'opération élémentaire à effectuer ) est équivalente en l'infini à $n^2.n!$. Ceci rend cette méthode totalement inutilisable pour n superieur à quelque unité.
\subsection{Méthode du pivot de Gauss}
À l'aide d'opération élémentaire, on peut modifier le système pour obtenir un système triangulaire. Avec cette méthode, la complexité de l'algorithme de résolution du système est équivalente en l'infini à $\dfrac{n^3}{3}$. La complexité est donc tout à fait acceptable.
\subsubsection{Détails de la méthode général}
\begin{de}
Une famille est un ensemble ordonée, qui accepte les répétitions
\end{de}
Sachant que A est inversible, la famille ($C_1,..C_n$) des colonnes de A est libre, on peut donc trouver un pivot non nul.\\
On fixe un pivot non nul (à l'aide de permutation si besoin), et élimine l'inconnu du pivot dans toutes les autres ligne par substitution. Et on intère la méthode pour toutes les inconnus, jusqu'a obtenir un système triangulaire.
\subsubsection{Défault de la méthode du pivot de Gauss}
Cette méthode est extrèmement instable numériquement. Il y a des erreurs d'arrondi lors des calculs (incontournable), mais ces erreurs peuvent être multiplié par un facteur extremement grand si le pivot est "petit". Cette méthode n'est donc pas fiable pour les grands système.
\subsection{Méthode de Jacobi}
La méthode de Jacobi s'applique à la résolution des systèmes Strictement diagonalement dominant
\begin{de}
Un système (S), ou la matrice A, est dit Strictement diagonalement dominant ( notée Sdd) si : 
$$\forall i \in [1,n] |a_{ii}| > \underset{j \neq i}\sum |a_{ij}|$$
\end{de}
\subsubsection{Détails de la méthode général}
La méthode de Jacobi consite à réecrire le systeme Sdd sous la forme suivante : 
On résoud ce systeme en considérant que les coefficiant non diagonaux dans le systeme de départ sont nul. On obtient donc une valeur approché de la solution, on la note $X_0$ par exemple. Puis on itere le procédé avec la formule suivante : 
$$X_{k+1} = A.X_k + b$$
Cette formule devient par récurrence : 
$$\forall k \in N~ X_k -\Omega = A^k ( X_0 - \Omega)$$
Le comportement de $(X_k)$ dépend donc principalement de $A^k$.\\
On montre que si M, la matrice des coefficiants, est Sdd, alors : 
$$\lim_{k\rightarrow \infty} A^k = \overrightarrow{0}$$
Avec $\overrightarrow{0}$ l'élément nul de l'espace $M_n$, l'espace des matrices carrée d'ordre n.\\
On obtient donc que :
$$\forall X_0 \in K^N ~ \lim_{k\rightarrow \infty}X_k = \Omega$$
La convergence de la méthode de Jacobi est donc indépendante de l'approximation initale. Cette méthode est donc stable numériquement.
\section{Valeur propres, vecteurs propre, sous-espace vectoriel propre}
\subsection{Vecteurs propres}
\begin{de}
Soit E un K espace vectoriel et $f \in \mathcal(L)(E)$, le groupe des endomorphisme de E dans E.\\
On dit que $\overrightarrow{x}\in E$ est un vecteur propre de f si $f(\overrightarrow{x})$ est parralèle à $\overrightarrow{x}$, c'est à dire si :
$$\exists \lambda \in K~ tq~ f(\overrightarrow{x}) = \lambda \overrightarrow{x}$$
Avec $\lambda$ qui a priori dépend de $\overrightarrow{x}$.
\end{de}
\begin{prop}
\begin{itemize}
 \item[$\rightarrow$] Si $\overrightarrow{x} = \overrightarrow{0}$, alors $\forall \lambda \in K~ f(\overrightarrow{0}) = \lambda.\overrightarrow{0}$
 \item[$\rightarrow$] Si $\overrightarrow{x} \neq 0$, alors il existe au plus un $\lambda \in K$ tq $f(\overrightarrow{x}) = \lambda\overrightarrow{x}$
\end{itemize}
\end{prop}
\subsection{Valeur propre}
\begin{de}
On appele valeur propre de l'endomorphisme tout $\lambda \in K$ tq : 
$$\exists \overrightarrow{x}\in E - \left\lbrace 0 \right\rbrace~ tq~ f(\overrightarrow{x}) = \lambda \overrightarrow{x}$$
On enlève $\overrightarrow{0}$ pour la propriété vu ci-dessus.
\end{de}
\subsection{Propriétés et définitions}
\begin{de}
Les vecteurs $\overrightarrow{x} \in E$ tq $f(\overrightarrow{x}) = \lambda \overrightarrow{x}$ sont appelé vecteur propre associé à la valeur propre.\\
Leurs ensembles est égale à $Ker(f-\lambda Id)$. C'est un sous espace vectoriel de E, appelé sous espace vectoriel associé à la valeur propre $\lambda$.\\
L'ensemble des valeurs propres de f est appelé spectre de f, notée $S_p(f)$ :
$$\lambda \in S_p(f) \Leftrightarrow Ker(f-\lambda.Id) \neq \left\lbrace \overrightarrow{0} \right\rbrace $$
\end{de}
\begin{theo}
Des sous espaces vectoriel propre, d'un endomorphisme f, associés à des valeur propre deux à deux différentes sont en somme direct. 
\end{theo}
\begin{prop}
Nous avons les propriétés suivantes : 
\begin{itemize}
 \item[$\rightarrow$] f est injective $\Leftrightarrow$ Ker f = $\left\lbrace \overrightarrow{0} \right\rbrace $
 \item[$\rightarrow$] f est injective $\Leftrightarrow 0 \notin S_p(f)$
\end{itemize}
Si la dimension de E est fini, nous avons la propriété suivante : 
\begin{itemize}
  \item[$\rightarrow$] f est bijective $\Leftrightarrow 0 \notin S_p(f)$
\end{itemize}
\end{prop}
\begin{prop}
Dans un système aux vecteurs propres ( C'est à dire un système définissant l'espace Ker(f-$\lambda Id$), les équations sont toujours liée entre elles, autrement dit elle sont linéairement dépendentes. Ce système n'est donc pas un système de Cramer, ce n'est donc pas un système inversible.
\end{prop}

\subsection{Cas des matrices}
On appele $v_p,\overrightarrow{v_p}$, sous espace propre, spectre de A le $v_p,\overrightarrow{v_p}$, sous espace propre, spectre de f, l'endomorphisme canoniquement associé à A : 
$$f : K^n \rightarrow K^n$$
$$ X \mapsto A.X$$
Donc, par définition : 
$$\lambda \in S_p(A) \Leftrightarrow \exists X \in K^n,~ X \neq \overrightarrow{0}~ tq~ AX = \lambda X$$
\section{Cas où E est de dimension finie : Polynôme caractéristique}
Dans tout ce chapitre, E est un K espace vectoriel de dimension n et $f \in \mathcal{L}(E)$
\begin{prop}
On montre que $S_p(f)$ est l'ensemble des racines dans K du poylome $P_f \in K[X]$, défini par :
$$P_f(X) = det(f-X.id)$$
Ce polynome est aussi notée $\chi_f$. On défini aussi ce polynome par : 
$$P_f(X) = det(X.id-f)$$
Ces deux définitions sont équivalente, sauf qu'il y a un rapport $(-1)^n$ entre les deux, car : 
$$det(-g) = (-1)^n.det(g)$$
Par extension au matrice, on obtient que :
$$P_f(X) = det(A-X.I_n)$$
\end{prop}
\begin{prop}
L'ensembles des valeurs propre d'un endomorphisme est aussi l'ensemble des racines du polyôme $P_f$.
\end{prop}
\begin{de}
Soit $f \in \mathcal{L}(E)$, avec E un K espace vectoriel de dimension finie n.\\
On appelle polynome caractéristique de f le polynome : 
$$P_f(X) = det(f-\lambda.id) \in K[X]$$ 
\end{de}
\begin{de}
Soit A $\in \mathcal{M}_n(K)$.\\
On appelle polynome caractéristique de A : 
$$P_A = det (A - X.I_n)$$
\end{de}
\begin{prop}
Si f est l'endomorphisme de $K^n$ canoniquement associé à A, alors : 
$$P_A(X) = P_f(X)$$
\end{prop}
\begin{prop}
Soit f $\in \mathcal{L}(E)$, E un K espace vectoriel de dimension n. On obtient que : 
$$P_f(X) = (-1)^n\left[X^n-Trace(f).X^{n-1}+\dots+(-1)^n.det(f)\right] $$
Les coefficiants dans les $\dots$ ne sont pas à connaitre.
\end{prop}
\begin{theo}
Soit f $\in \mathcal{L}(E)$, avec E un K espace vectoriel de dimension n.\\
Alors : 
$$\forall \lambda \in S_p(f),~ 1\leq dim(Ker(f-\lambda.Id) \leq mult_{P_f}(\lambda)$$
Avec $mult_{P_f}(\lambda)$ la multiplicité de $\lambda$ dans les racines de $P_f$
\end{theo}
\subsection{Relations entre les racines d'un polynome et ses racines}
\begin{prop}
Soit P un polynome scindé de la forme : 
\begin{eqnarray*}
  P(X) & = & (X-\lambda_1)\dots(X - \lambda_n)\nonumber \\
   & = & X^n+\alpha_{n-1}.X^{n-1}+...+\alpha_1+\alpha_0 \nonumber \\
\end{eqnarray*}
On obtient les relations suivantes : 
\begin{eqnarray*}
  \alpha_{n-1} & = & -\sum_{i=1}^n \lambda_i \\
  \alpha_{n-2} & = & -\sum_{\underset{i<j}{i,j=1}}^n \lambda_i.\lambda_j \\
  \vdots & = & \vdots\\
  \alpha_0 & = & (-1)^n\lambda_1\dots\lambda_n\\
\end{eqnarray*}
\end{prop}
\section{Diagonalisabilité}
\subsection{Définitions}
\begin{de}
Soit $f \in \mathcal{L}(E)$, avec E un K espace vectoriel de dimension n.\\
On dit que f est diagonalisable s'il existe une base B de E tq $mat_B(f)$ soit diagonale : 
\[ mat_B(f) = \begin{pmatrix}
  \lambda_1 &  & (0) \\
   & \ddots &  \\
  (0)&  & \lambda_n \\

\end{pmatrix}
\]
Quitte à réordonnée les vecteurs de B, il existe aussi une base B' de E telque : 
\[mat_{B'}(f) = \begin{blockarray}{cccccccccc}
        | \leftarrow & -\overset{n_1}-- &\rightarrow | & | \leftarrow & -\overset{n_2}-- &\rightarrow|& \dots  & |\leftarrow & -\overset{n_p}-- & \rightarrow| \\
        \begin{block}{(cccccccccc)}
           \lambda_1 &  &  & & & & & & & (0) \\
   	   & \ddots & & & & & & & \\
  	   &  & \lambda_1 & &  & & & & &\\
  	   &  &  & \lambda_2 & & & & & & \\
  	   &  &  &  & \ddots & & & & &\\
 	   &  &  &  & & \lambda_2 & & & & \\
 	   &  &  &  & &  & \ddots & & &\\
 	   &  &  &  & &  &  & \lambda_p & &\\
 	   &  &  &  & &  & & & \ddots &\\
  	  (0) &  &  &  & &  &  & & & \lambda_p\\
        \end{block}
        \end{blockarray}
\]

Avec les $\lambda_i$ deux à deux distincts. On obtient alors que : 
\begin{eqnarray*}
  P(X) & = & (\lambda_1-X)^{n_1}\dots(\lambda_p-X)^{n_p}\nonumber \\
   & = & (-1)^n (X-\lambda_1)^{n_1}\dots(X-\lambda_p)^{n_p} \nonumber \\
\end{eqnarray*}
Le polynome est donc scindé.
Comme les $\lambda_i$ sont deux à deux distincts, on obtient que : 
$$n_i = multi_{P_f}(\lambda_i)$$
De plus, nous avons les résultats suivants :
\begin{itemize}
 \item[$\rightarrow$]Ker($f-\lambda_1.Id)$ est le sous espace vectoriel engendré par les $n_1$ premiers vecteurs de B'
 \item[$\rightarrow$]Ker($f-\lambda_2.Id)$ est le sous espace vectoriel engendré par les $n_2$ vecteurs suivants de B'
 \item[$\rightarrow$]Etc ....
\end{itemize}
Enfin, on obtient que les sous espaces vectoriel propres sont supplémentaire.
\end{de}
\begin{theo}
Soit f $\in \mathcal{L}(E)$, avec E un K espace vectoriel de dimension finie n. Les conditions suivantes sont équivalente :
\begin{itemize}
 \item[$\rightarrow$] f est diagonalisable
 \item[$\rightarrow$] $P_f$ est scindé et $\forall \lambda \in Sp(f)$, $dim(f-\lambda.Id)$ = $mult_{P_f}(\lambda)$
 \item[$\rightarrow$] Les sous espaces propres de f sont supplémentaire
 \item[$\rightarrow$] $\underset{\lambda \in S_p(f)}\sum dim(Ker(f-\lambda.Id)) = dim(E)$
\end{itemize}
\end{theo}
\subsection{Cas particulier des valeurs propres simples}
\begin{prop}
Soit $f\in \mathcal{L}(E)$, E un K espace vectoriel de dimension finies n.\\
Si $\lambda \in S_p(f)$ est une racine simple de $P_f$, on obtient que : 
$$dim(Ker(f-\lambda.Id) = 1 $$
Les sous espaces associé à une valeur propre simple sont donc des droites vectoriel, appelé droite propre
\end{prop}
\begin{prop}
Si $P_f$ est un polynome scindé, c'est à dire que f admet n valeurs propres simple, alors on as : 
$$\forall k \in \left\lbrace 1,...,n \right\rbrace~ dim(Ker(f-\lambda_k.Id)) = multi_{P_f}(\lambda_k) = 1 $$
On en déduit donc que f est diagonalisable et que tout ses sous espaces vectoriel propres sont des droites vectoriel.
\end{prop}
\subsection{Cas d'une matrice}
\begin{de}
Soit $A \in \mathcal{M}_n(K)$. On dit que A est diagonalisable si f, l'endomorphisme canoniquement associé à A est diagonalisable
\end{de}
\begin{prop}
Soit $A \in \mathcal{M}_n(K)$. Nous avons la propriété suivantes : 
$$\mbox{( A est diagonalisable )} \Leftrightarrow (\exists P \in GL_n(K)~ tq~ P^{-1}.A.P \mbox{ soit diagonale })$$
\end{prop}
\begin{theo}
Soit $A \in \mathcal{M}_n(\Re)$, une matrice symétrique réelle, alors A est orthonormalement diagonalisable, c'est à dire que A est diagonalisable ($P_A$ est scindé dans $\Re[X]$ ) et ses espaces propres, qui sont supplémentaire, sont deux à deux orthogonaux dans $\Re^n$ euclidien canonique, c'est à dire munie du produit scalaire canonique. 
\end{theo}
\section{Trigonalisabilité}
\begin{de}
Soit $f \in \mathcal(E)$, avec E un K espace vectoriel de dimension finie n.\\
On dit que f est trigonalisable si il existe une base B de E telque $mat_f(B)$ soit triangulaire superieur.
\end{de}
\begin{de}
Soit $A \in \mathcal{M}_n(K)$.\\
On dit que A est trigonalisable si l'endomorphisme canoniquement associée à A est trigonalisable.
\end{de}
\begin{prop}
Soit $A \in \mathcal{M}_n(K)$. Nous avons la propriété suivantes : 
$$\mbox{( A est trigonalisable )} \Leftrightarrow (\exists P \in GL_n(K)~ tq~ P^{-1}.A.P \mbox{ soit triangulaire superieur })$$
\end{prop}
\begin{theo}
Soit $f \in \mathcal{L}(E)$, avec E une K espace vectoriel de dimension finie n.\\
$$\mbox{(f est trigonalisable)} \Leftrightarrow (P_f \mbox{ est scindé dans } K[X])$$
\end{theo}
\begin{corr}
Soit E un $\mathcal{C}$ espace vectoriel de dimension finie, alors tout $f \in \mathcal{L}(E)$ est trigonalisable.
De même, toute matrice $A \in \mathcal{M}_n(\mathcal{C})$ est trigonalisable
\end{corr}


\chapter{Réduction des endomorphismes et des matrices - Deuxième Partie}
\section{Polynomes d'endomorphisme ou de matrice}
Dans ce chapitre, toutes les relations vu sont transposable aux matrices
\begin{de}
Soit f une application linéaire de E dans E, avec E un K espace vectoriel.\\
Soit P $\in K[X]$ défini par : 
$$P = a_0 + a_1.X+\dots+a_p.X^p$$
Avec $\forall i~ a_i \in K$. On défini :
$$P(f) = a_0.Id + a_1.f +\dots+a_p.f^p$$
Avec : 
\begin{itemize}
 \item[$\rightarrow$] $fofo\dots of = f^k$
 \item[$\rightarrow$] $f^0 = Id$
\end{itemize}
De meme, si A $\in \mathcal{M}_n(K)$ : 
$$P(A) = a_0.I_n + a_1.A + a_p.A^p$$
\end{de}
\begin{prop}
Soit B est une base de E, avec E un K espace vectoriel de dimension fini, f$\in \mathcal{L}(E)$ et $A = mat_B(f)$. On obtient : 
$$P(A) = mat_B(P(f))$$
\end{prop}
\begin{prop}
Soit $P_1$ et $P_2 \in K[X]$, $\lambda \in K$, $f\in \mathcal{L}(E)$.\\
Nous avons les résultats suivants :
$$(P_1+P_2)(f) = P_1(f)+P_2(f)$$
$$(\lambda P_1)(f) = \lambda P_1(f)$$
$$(P_1.P_2)(f) = P_1(f) o P_2(f)$$
De ce dernier résultat, on obtient que : 
$$P_1(f)oP_2(f) = P_2(f)oP_1(f)$$
\end{prop}
\begin{de}
Soit f un endomorphisme de E, avec E un K espace vectoriel et :
$$K[f] = \left\lbrace P(f), P\in K[X] \right\rbrace $$
Soit $A \in \mathcal{M}_n(K)$.\\
On note : 
$$K[A] = \left\lbrace P(A), P \in K[X] \right\rbrace $$
\end{de}
\begin{prop}
Les espaces défini ci dessus sont des sous algèbres commuative de respectivement $(\mathcal{L}(E),+,\lambda.,o)$ et $(\mathcal{M}_n(K),+,\lambda.,x)$
\end{prop}
\section{Idéaux de K[X]}
\begin{de}
On appelle idéale de K[X] toute partie non vide $\mathcal{I}$ de K[X] tq : 
\begin{itemize}
 \item[$\rightarrow$] $\mathcal{I}$ est stable par +
 \item[$\rightarrow$] $\forall P \in \mathcal{I}$ et $\forall Q \in K[X]$, PQ $\in \mathcal{I}$
\end{itemize}
De cette définition, on obtient que : 
$$P \in \mathcal{I} \Rightarrow -P \in \mathcal{I}$$
$$0 \in \mathcal{I}$$
Avec ici 0, le polynome constant nul.
\end{de}
\subsection{Exemple}
Nous avons les ensembles suivants, qui sont des idéaux triviaux : 
$$\mathcal{I} = \left\lbrace 0 \right\rbrace $$
Celui ci constitue l'idéal nul.
$$K[X] = \mathcal{I}$$
\begin{de}
Soit P un polynome de K[X]. On défini l'idéal engendré par P, noté [P], par : 
$$[P] = \left\lbrace PQ, Q \in K[X]\right\rbrace $$
C'est donc l'ensemble constitué des multiples de P.
\end{de}
\subsection{Définitions et théorème}
\begin{de}
Un idéal engendrée par un seul polynome, du type [P], est appelé idéal principale
\end{de}
\begin{theo}
Tout idéal de K[X] est principale. On dit donc que l'anneau K[X] est principale. 
\end{theo}
\begin{de}
Soit $\mathcal{I}$, un idéal de K[X], donc un idéal idéal.\\
On appelle générateurs de $\mathcal{I}$ les polynomes $\omega$ telque : 
$$\mathcal{I} = [\omega]$$
\end{de}
\begin{prop}
Les générateurs $\omega$ se déduisent les uns des autres par multiplication par une constante non nulle $\lambda \in K^*$.\\
De plus, si $\mathcal{I} \neq \left\lbrace 0 \right\rbrace $, alors les générateurs ont tous le même degrés : 
$$deg(\omega) = min\left\lbrace deg(P), P \in \mathcal{I}-\left\lbrace  0\right\rbrace  \right\rbrace  $$
\end{prop}
\begin{prop}
De la propriété précédente, on déduit que si : 
\begin{itemize}
 \item[$\rightarrow$] $\omega \in \mathcal{I}$
 \item[$\rightarrow$] deg($\omega) min\left\lbrace deg(P), P \in \mathcal{I}-\left\lbrace  0 \right\rbrace\right\rbrace $
Alors : 
$$\mathcal{I} = [\omega]$$ 
\end{itemize}
\end{prop}
\begin{de}
L'unique générateur unitaire d'un idéal $\mathcal{I}$ non nul est appelé polynome minmale de l'idéal $\mathcal{I}$
\end{de}
\subsection{Application au pgcd de deux polynomes, et à l'algorithme d'Euclide}
Soit $P_1$ et $P_2$ deux polynomes de K[X] non tous les deux nuls.\\
On sait que [$P_1,P_2$] est un idéal de K[X], donc un idéal principale.\\
On obtient donc qu'il existe un unique $\omega \in K[X] - \left\lbrace 0 \right\rbrace $ telque : 
$$[P_1,P_2] = [\omega]$$
On montre que $\omega$ est le pgcd de $P_1$ et $P_2$
\subsubsection{Algorithme d'Euclide}
Cet algoritme se base sur la propriété suivante : 
$$\forall Q \in K[X] ~ [P_1,P_2] = [P_1,P_2 + Q.P_1]$$
\subsection{Polynome annulateur d'un endomorphisme ou d'une matrice, Polynome minimal d'un endomorphisme ou d'une matrice}
\begin{de}
Soit $f \in \mathcal{L}(E)$, avec E un K espace vectoriel.\\
On dit que $P \in K[X]$ annule f, ou que P est un polynome annulateur de f, ou encore que f annule P si : 
$$P(f) = \tilde{0}$$
De meme, si $A \in \mathcal{M}_n(K)$, on dit que P annule A si :
$$P(A) = \begin{pmatrix}
          0 & \dots & 0 \\
          \vdots & \ddots & \vdots \\
          0 & \dots & 0
         \end{pmatrix}
$$
\end{de}
\begin{prop}
L'ensemble $A_{nn}(f)$ (Notation non standard), des polynomes annulateur de f, défini par :
$$A_{nn}(f) = \left\lbrace P \in K[X]~ tq~ P(f) = \tilde{0}\right\rbrace $$
Cet ensemble est un idéal de K[X].
\end{prop}
\begin{de}
On appelle polynome minimale de f, noté $\omega_f$, l'unique polynome unitaire telque que : 
$$A_{nn}(f) = [\omega]$$
On défini de même le polynome minimal d'une matrice
\end{de}
\begin{prop}
Soit $f \in \mathcal{L}(E)$, avec E un K espace vectoriel de dimension finie, B une base de E et A = $mat_{B}(f)$.\\
On obtient dans ce cas que : 
$$\omega_A = \omega_f$$
\end{prop}
\begin{prop}
Soit E une K espace vectoriel de dimension n.\\
On obtient que, $\forall f \in \mathcal{L}(E)$ : 
$$A_{nn} \neq \left\lbrace O\right\rbrace $$
Donc que :
$$\omega_f \neq 0$$
\end{prop}
\begin{prop}
Soit $f \in \mathcal{L}(E)$, avec E un K espace vectoriel.\\
Si $\omega_f \neq 0$ et d = deg($\omega_f$), alors (Id,f,...,$f^{d_1}$) est une base de K[f], en particulier : 
$$dim K[f] = deg(\omega_f)$$
\end{prop}
\subsection{Théorème de Cayley-Hamilton}
\begin{theo}
Soit $f \in \mathcal{L}(E)$, avec E un K espace vectoriel de dimension fini. On obtient alors que :
$$P_f(f) = \tilde{0}$$
Nous avons les équivalences suivantes : 
$$(P_f(f) = \tilde{0}) \Leftrightarrow (P_f \in A_{nn}(f) )\Leftrightarrow (\omega_f | P_f)$$
\end{theo}
\begin{corr}
 Si E est une K espace vectoriel de dimension n, et $f \in \mathcal{L}(E)$, alors deg $w_f \leq n$
\end{corr}
\subsection{Relation entre valeurs propres et racines des polynomes annulateurs}
\begin{prop}
Soit $f \in \mathcal{L}(E)$, avec E un K espace vectoriel ( E peut être un espace de dimension infini).\\
Si f annule $P \in K[X]$, alors :
$$S_p(f) c Z(P)$$
Avec Z(P) l'ensemble des zéros de P, c'est à dire l'ensemble des racines de P.
\end{prop}
\begin{lemme}
Si $f(\overrightarrow{x}) = \lambda.\overrightarrow{x}$, alors :
$$\forall P \in K[X]~ P(f)(\overrightarrow{x}) = P(\lambda)(\overrightarrow{x})$$
\end{lemme}
\begin{prop}
Si $f \in \mathcal{L}(E)$, avec E un K espace vectoriel de dimension finies, alors :
$$S_p(f) = Z(w_f)$$
Cependant, ceci ne nous donne bien évidement aucune information sur la multiplicité des racines.
\end{prop}
\begin{prop}
Soit f et g deux endomorphisme de E dans E. Si : 
$$fog = gof$$
C'est à dire, si les deux endomorphimes communent, alors Ker(g) et Im(g) sont stable par f
\end{prop}
\section{Lemme des noyaux}
\begin{prop}
Soit $f \in \mathcal{L}(E)$, avec E un K espace vectoriel. Soit $P_1$ et $P_2$ deux polynomes de $K[X]$, premiers entre eux. On obtient alors : 
$$Ker(P_1.P_2)(f) = Ker(P_1)(f) \oplus Ker(P_2)(f)$$
\end{prop}
\begin{gene}
Soit $f \in \mathcal{L}(E)$, avec E un K espace vectoriel. Soit $P_1,\dots,P_k$ des polynomes de K[X] deux à deux premiers entre eux. Soit P = $P_1\dots P_k$, on obtient alors : 
$$Ker(P)(f) = Ker(P_1)(f) \oplus \dots \oplus Ker(P_k)(f)$$
\end{gene}
\subsection{Application fondamentale de Lemme des noyaux}
Soit $f \in \mathcal{L}(E)$, avec E un K espace de dimension n telque $P_f$ soit scindé sur K[X]. On peut donc écrire $P_f$ sous la forme : 
$$P_f(X) = (-1)^n(X-\lambda_1)^{n_1}\dots(X-\lambda_p)^{n_p}$$
En utilisant conjointement le lemme des noyaux et le théorème de Cayley-Hamilton, on obtient que : 
$$P_f(f) = (-1)^n.(f-\lambda_1.Id)^{n_1}o\dots o (f-\lambda_p.Id)^{n_p} = \tilde{0}$$
Grâce au lemme des noyaux, on obtient que :
$$E = E_1 \oplus \dots \oplus E_p$$
Avec :
$$E_k = Ker(f-\lambda_k.Id)^{n_k}$$
On peut obtenir à partir de tout ceci une matrice diagonale par bloc, de la forme : 
\[mat_{B}(f) = \begin{blockarray}{cccccccccc}
        | \leftarrow & -\overset{n_1}-- &\rightarrow | & | \leftarrow & -\overset{n_2}-- &\rightarrow|& \dots  & |\leftarrow & -\overset{n_p}-- & \rightarrow| \\
        \begin{block}{(cccccccccc)}
           \lambda_1 &  &  & & & & & & & (0) \\
   	   & \ddots & & & & & & & \\
  	   &  & \lambda_1 & &  & & & & &\\
  	   &  &  & \lambda_2 & & & & & & \\
  	   &  &  &  & \ddots & & & & &\\
 	   &  &  &  & & \lambda_2 & & & & \\
 	   &  &  &  & &  & \ddots & & &\\
 	   &  &  &  & &  &  & \lambda_p & &\\
 	   &  &  &  & &  & & & \ddots &\\
  	  (0) &  &  &  & &  &  & & & \lambda_p\\
        \end{block}
        \end{blockarray}
\]
\section{Endomorphismes et matrices nilpotants}
\begin{de}
Soit $f \in \mathcal{L}(E)$.\\
f est dit nilpotant si il existe $p \in N^*$ telque : 
$$f^p = \tilde{0}$$
La définition est analogue pour les matrices
\end{de}
\begin{prop}
Soit E un C espace vectoriel de dimension n et $f \in \mathcal{L}(E)$, alors les conditions suivantes sont équivalentes :
\begin{itemize}
 \item[$\rightarrow$] f est nilpotant
 \item[$\rightarrow$] $S_p(f) = \left\lbrace 0 \right\rbrace $
 \item[$\rightarrow$] $\exists$ B base de E telque $mat_B(f)$ soit une matrice stricement triangulaire, c'est à dire triangulaire avec tous ces termes diagonaux nuls.
\end{itemize}
\end{prop}

\begin{de}
On appelle indice de nilpotence de f le plus petit entier $\nu$ telque : 
$$f^{\nu} = \tilde{0}$$
Si E est un C espace vectoriel de dimension n, on obtient que : 
$$\nu \geq n$$
Plus précisement, on obtient que : 
$$w_f = X^{\nu}$$
\end{de}
\section{Nouveaux critères de trigonabilité}
\begin{theo}
Soit $f \in \mathcal{L}(E)$, avec E un K espace vectoriel de dimension n. Les conditions suivantes sont équivalentes : 
\begin{itemize}
 \item[$\rightarrow$] f est trigonalisable
 \item[$\rightarrow$] f annule un polynome scindé sur K
 \item[$\rightarrow$] $w_f$ est scindé sur K
\end{itemize}
\end{theo}
\subsection{Réduction de Dunford}
Si :
$$w_f = (X - \lambda_1)^{m_1}...(X - \lambda_p)^{m_p}$$
avec les $\lambda_k$ deux à deux distinct, alors on obtient que, d'après le lemme de noyaux : 
$$E = E_1 \oplus ... \oplus E_P$$
avec : 
$$E_k = Ker(f + \lambda_k.id)^{m_k}$$
On peut donc choisir une base de chaqu'un des $E_k$, notée $B_k$, telque :
$$mat_{B_k}f_{\parallel E_k} = \begin{pmatrix}
          \lambda_k &  & (a_{ij}) \\
	   & \ddots & \\
          (0) &  & \lambda_k
         \end{pmatrix}$$
C'est donc une matrice triangulaire superieur. De plus, on a : 
$$B = B_1 \vee \dots \vee B_P$$
qui est une base de E. On obtient donc que : 
\[mat_B f = \begin{blockarray}{cccc}
        \overset{n'_1}\leftrightarrow & \dots & \overset{n'_p}\leftrightarrow  \\
        \begin{block}{(ccc)c}
        \mesbloc{A_{1}}  &  & (0) & \updownarrow n'_1   \\
          & \ddots &  &       \\
        (0) &   & \mesbloc{A_{p}} & \updownarrow n'_p      \\
        \end{block}
        \end{blockarray}
\] 
Avec $A_k = mat_{B_k}f_{\parallel E_k}$. On montre de plus qu'en réalité : 
\begin{itemize}
 \item[$\rightarrow$] $n'_k$ est en réalité egale à $mult_{Pf}(\lambda_k)$
 \item[$\rightarrow$] $E_k$ = Ker(f-$\lambda$.id$)^{n_k}$
\end{itemize}
\section{Nouveau critère de diagonabilité}
\begin{theo}
Soit $f \in \mathcal{L}(E)$, avec E un K espace vectoriel de dimension finies. Les propositions suivantes sont équivalente : 
\begin{itemize}
 \item[$\rightarrow$] f est diagonalisable
 \item[$\rightarrow$] f annule un polynome scindé sur K à racine simple
 \item[$\rightarrow$] $w_f$ est scindé sur K à racine simple
\end{itemize}
De plus, nous savons que : 
$$Z(w_f) = S_p(f)$$
La première proposition s'écrit donc : 
$$w_f(X) = (X-\lambda_1)...(X-\lambda_p)$$
ou 
$$S_p(f) = \left\lbrace \lambda_1,...,\lambda_p\right\rbrace $$
\end{theo}


\part{Espaces vectoriels normés}
\setcounter{chapter}{0}
\chapter{Espaces vectoriels normés - Première partie}
\section{Norme - Distance -  Définitions}
\subsection{Définitions}
\begin{de}
Soit E un K espace vectoriel.\\
On appelle norme de E toutes application de $E \rightarrow \mathbb{R}^+$, que l'on note N ou $\parallel~ \parallel$, et vérifiant les axiomes suivants : \\
\begin{itemize}
 \item[$\rightarrow$]Si $\overrightarrow{x} \in E$, N($\overrightarrow{x}$) = 0 $\Rightarrow \overrightarrow{x} = \overrightarrow{0}$\\
 \item[$\rightarrow$]$\forall \lambda \in K$, $\forall \overrightarrow{x} \in E$, N($\lambda\overrightarrow{x}$ = |$\lambda$|N($\overrightarrow{x}$)\\
 \item[$\rightarrow$]$\forall(\overrightarrow{x},\overrightarrow{y}) \in E^2$ N($\overrightarrow{x}+\overrightarrow{y}$) $\leq$ N($\overrightarrow{x}$) + N($\overrightarrow{y}$)\\
\end{itemize}
\end{de}
\subsection{Conséquence immédiate des axiomes}
\begin{itemize}
 \item[$\rightarrow$] N($\overrightarrow{0}$) = 0\\
 \item[$\rightarrow$] $\forall (\overrightarrow{x_1},...,\overrightarrow{x_p}) \in E^p$, N($\overrightarrow{x_1}+...+\overrightarrow{x_p}) \leq N(\overrightarrow{x_1}) + ... + N(\overrightarrow{x_p})$\\
 \item[$\rightarrow$] $\forall (\overrightarrow{x},\overrightarrow{y}) \in E^2$ |$\parallel\overrightarrow{x}\parallel - \parallel\overrightarrow{y}\parallel$| $\leq$ $\parallel\overrightarrow{x}+\overrightarrow{y}\parallel$\\
 \item[$\rightarrow$] De meme : $\forall (\overrightarrow{x},\overrightarrow{y}) \in E^2$ |$\parallel\overrightarrow{x}\parallel - \parallel\overrightarrow{y}\parallel$| $\leq$ $\parallel\overrightarrow{x}-\overrightarrow{y}\parallel$
\end{itemize}
\section{Distance}
\subsection{Définitions}
\begin{de}
Soit $\varepsilon$ un ensemble quelconque, non vide.\\
On appelle distance sur $\varepsilon$ toutes applications :
$$d : \varepsilon \times \varepsilon \rightarrow \mathbb{R}^+$$
$$(\overrightarrow{x},\overrightarrow{y}) \mapsto d(\overrightarrow{x},\overrightarrow{y})$$
vérifiant les axiomes suivantes :\\ 
\begin{itemize}
 \item[$\rightarrow$] Si x et y sont dans $\varepsilon$, $d(x,y)=0 \Leftrightarrow x=y$\\
 \item[$\rightarrow$] $\forall(x,y) \in \varepsilon^2$, d(x,y) = d(y,x)\\
 \item[$\rightarrow$] $\forall(x,y,z) \in \varepsilon^3$, d(x,z) $\leq$ d(x,y)+d(y,z)\\
\end{itemize}
\end{de}
\subsection{Conséquence}
Des axiomes précédents, on peut étendre l'axiome n°2 : 
$$\forall(x_1,...,x_p) \in \varepsilon^p,~ d(x_1,...,x_p) \leq d(x_1,x_2) + ... + d(x_{p-1},x_p)$$
\subsection{Distance déduite d'une norme}
\begin{de}
Soit $\varepsilon$ un K espace affine et $\parallel~\parallel$ une norme sur $\overrightarrow{\varepsilon}$, l'espace vectoriel associé à l'espace affine.\\
On appelle distance déduite de $\parallel~\parallel$  sur $\varepsilon$ l'application : 
$$d : \varepsilon \times \varepsilon \rightarrow \mathbb{R}^+$$
$$(x,y) \mapsto d(x,y) = \parallel x - y\parallel$$
\end{de}
\section{Exemple classique de normes dans un espaces de dimension finies}
\subsection{La norme $\parallel~\parallel_{\infty}$ sur $K^n$}
\subsubsection{Définition générale}
Soit X la matrice défini par : 
$$X =
\begin{pmatrix}
x_1 \\
. \\
. \\
x_n \\
\end{pmatrix} \in K^n = M_{n,1}(K)$$
On associe le n-upplet à une matrice colonnes.\\
On défini la norme $\parallel X \parallel_{\infty}$ = max(|$x_k$|)
\subsubsection{Cas d'un espace de dimension n}
Soit E un K espace vectoriel de dimension n, et B=($\overrightarrow{e_1},...,\overrightarrow{e_n}$) une base de E.\\
Si:
$$\overrightarrow{x} = x_1.\overrightarrow{e_1}+...+x_n\overrightarrow{e_n}$$
On défini $\parallel\overrightarrow{x}\parallel_{\infty,B}$ = max(|$x_k$|)
\subsection{La norme $\parallel~\parallel_{1}$ sur $K^n$}
\subsubsection{Définition générale}
Soit X la matrice défini par : 
$$X =
\begin{pmatrix}
x_1 \\
. \\
. \\
x_n \\
\end{pmatrix} \in K^n = M_{n,1}(K)$$
On associe le n-upplet à une matrice colonnes.\\
On défini la norme par : 
$$\parallel X \parallel_{1} = \sum_{k=1}^n |x_k|$$
\subsubsection{Cas d'un espace de dimension n}
Soit E un K espace vectoriel de dimension n, et B=($\overrightarrow{e_1},...,\overrightarrow{e_n}$) une base de E.\\
Si:
$$\overrightarrow{x} = x_1.\overrightarrow{e_1}+...+x_n\overrightarrow{e_n}$$
On défini :
$$\parallel\overrightarrow{x}\parallel_{1,B} = \sum_{k=1}^n |x_k|$$
\subsection{La norme $\parallel~\parallel_{2}$ sur $K^n$}
\subsubsection{Définition}
Avec les même notations :
$$\parallel X \parallel_{2} = \sqrt{\sum_{k=1}^n |x_k|}$$
\subsubsection{Cas particulier : K = $\mathbb{R}$}
La norme défini ci dessus vérifie dans ce cas : 
$$\parallel~\parallel_{2} = \sqrt{<X|X>}$$
Si :
$$X =
\begin{pmatrix}
x_1 \\
. \\
. \\
x_n \end{pmatrix}$$
et :
$$Y = \begin{pmatrix}
y_1 \\
. \\
. \\
y_n \\ \end{pmatrix}$$
Alors : 
$$<X|Y> = \sum_{k=1}^n x_k.y_k$$
Ceci est le produit scalaire canonique de $\mathbb{R}^n$. C'est l'unique produit scalaire sur $\mathbb{R}^n$ qui fasse de la base canonique une base orthonormée.\\
Dans ce cas, l'inégalité triangulaire s'appelle l'inégalité de Minkowski
\section{Convergence au sens d'une norme ou d'une distance - Norme équivalente}
\subsection{Au sens d'une distance}
\begin{de}
Soit $(\varepsilon,d)$ un espace métrique, et ($x_n$) une suite de point de $\varepsilon$.\\
On dit que $(x_n)$ converge vers $x \in \varepsilon$ au sens de la distance d si et seulement si : 
$$d(x_n,x) \underset{n \mapsto \infty}\rightarrow 0$$
\end{de}
\subsection{Au sens d'une norme}
\begin{de}
Soit (E,$\parallel~\parallel$) un K espace vectoriel normé, et $(\overrightarrow{x_n})$ une suite d'éléments de E.\\
On dit que la suite $(\overrightarrow{x_n})$ converge vers $\overrightarrow{l} \in E$ au sens de la norme $\parallel~\parallel$ si et seulement si :
$$\parallel \overrightarrow{x_n} - \overrightarrow{l} \parallel \underset{n \mapsto \infty }\rightarrow 0$$
En théorie, cette définition ramène le problème à un problème de convergence dans $\mathbb{R}^+$. La notion de norme sert à unifier les études de convergence.
\end{de}
\begin{prop}
Cette propriété est un cas particulier de la définition au sens d'une distance, en utilisant la norme déduit de la norme.
\end{prop}
\subsection{Norme équivalente}
\begin{de}
Soit $\parallel~\parallel$ et $\parallel~\parallel$ deux normes sur un même K espace vectoriel de E.\\
Ces deux normes sont dites équivalente si il existe $\alpha$ et $\beta$ deux réels strictement positif telque : 
$$\parallel~\parallel'\leq\alpha\parallel~\parallel$$
$$\parallel~\parallel \leq \beta \parallel~\parallel'$$ 
\end{de}
\begin{de}
On peut écrire cette définition sous la forme suivante : \\
Ces deux normes sont équivalente si les deux applications suivantes :
$$E - \left\lbrace 0 \right\rbrace \rightarrow \mathbb{R}^+$$
$$\overrightarrow{x} \mapsto \dfrac{\parallel\overrightarrow{x}\parallel'}{\parallel\overrightarrow{x}\parallel}$$
et 
$$E - \left\lbrace 0 \right\rbrace \rightarrow \mathbb{R}^+$$
$$\overrightarrow{x} \mapsto \dfrac{\parallel\overrightarrow{x}\parallel}{\parallel\overrightarrow{x}\parallel'}$$
sont majorées.
\end{de}
\begin{prop}
Il résulte de ce qui précède que si $\parallel~\parallel$ et $\parallel~\parallel'$ sont deux normes équivalente sur le K espace vectoriel E : 
$$\overrightarrow{x_n} \underset{\infty}\rightarrow \overrightarrow{x}~ dans~ (E,\parallel~\parallel) \Leftrightarrow \overrightarrow{x_n} \underset{\infty}\rightarrow \overrightarrow{x}~ dans~ (E,\parallel~\parallel')$$
\end{prop}
\begin{prop}
Nous avons aussi la propriété réciproque :\\
Soient $\parallel~\parallel$ et $\parallel~\parallel'$ deux normes sur un meme K espace vectoriel E telque pour toutes suite $(\overrightarrow{x_n})\in E^N$, et pour tous $\overrightarrow{x}\in E$ :
$$\overrightarrow{x_n} \underset{\infty}\rightarrow \overrightarrow{x}~ dans~ (E,\parallel~\parallel) \Leftrightarrow \overrightarrow{x_n} \underset{\infty}\rightarrow \overrightarrow{x}~ dans~ (E,\parallel~\parallel')$$
Alors ces deux normes sont équivalentes.
\end{prop}
\section{Convergence dans les K espaces vectoriel de dimension finies}
\begin{theo}
Si E est un K espace vectoriel de dimension finie, alors toutes les normes sur E sont équivalentes
\end{theo}
\begin{corro}
La converge vers $\overrightarrow{x}\in E$ d'une suite $(\overrightarrow{x_n}) \in E^N$ ne dépend pas de la norme choisie si E est un K espace vectoriel de dimension finie.
\end{corro}
\begin{prop}
Soit B=($\overrightarrow{e_1},...,\overrightarrow{e_p}$) une base du K espace vectoriel E. Si :
$$\overrightarrow{x_n} = x_n^1\overrightarrow{e_1} + ...+x_n^p.\overrightarrow{e_p}$$
$$\overrightarrow{x} = x^1\overrightarrow{e_1} + ...+x^p.\overrightarrow{e_p}$$
Alors : 
$$\overrightarrow{x_n} \underset{\infty}\rightarrow \overrightarrow{x} \Leftrightarrow x_n^1\underset{\infty}\rightarrow x^1,...,x_n^p\underset{\infty}\rightarrow x^p$$
On peut écrire ceci de la façon suivante : \\
$\overrightarrow{x_n} \rightarrow \overrightarrow{x}$ si et seulement si il y a convergence composant par composant.
\end{prop}


\chapter{Espace vectoriel normé - Deuxième partie}
\section{Interieur d'un ensemble, ensemble ouvert}
\begin{de}
Soit (E,d) un espace métrique. En général, E est un K espace vectoriel munie d'une norme $\parallel~\parallel$, et d est la distance déduite de cette norme.\\
Soit A un ensemble non vide de E. On dit que $a \in E$ est intérieur à A, si il existe r>0 telque :
$$B(a,r) c A$$
\end{de}
\begin{de}
On appelle interieur de A, et on le note $\overset{o}A$, l'ensemble des points interieurs à a.
\end{de}
\begin{de}
A est dit ouvert si tout point de A est interieur à A. C'est à dire si :
$$A c \overset{o}A$$
Or, par définition, nous avons : 
$$\overset{o}A c A$$
Donc, nous avons l'équivalence suivante : 
$$(\mbox{ A est un ouvert }) \Leftrightarrow (A = \overset{o}A)$$
\end{de}
\begin{prop}
Nous avons les propriétés suivantes : 
\begin{itemize}
 \item[$\rightarrow$] $\emptyset$ est un ouvert par convention.
 \item[$\rightarrow$] E est un ouvert
 \item[$\rightarrow$] Une intersection finie d'ouvert est un ouvert.
 \item[$\rightarrow$] Une réunion quelconque (finie ou infinie) d'ouvert est un ouvert.
\end{itemize}
\end{prop}
\begin{prop}
$\overset{o}A$ est le plus grand (Au sens de l'inclusion) ouvert inclu dans A.
\end{prop}
\subsubsection{Exemple}
Nous avons un exemple classique : 
\begin{itemize}
 \item[$\rightarrow$] Les boules ouvertes d'une espace métrique sont des ouverts
\end{itemize}

\section{Adérence d'un ensemble, ensemble fermé}
Considérons toujours A, un sous ensemble non vide d'un espace métrique (E,d).
\begin{de}
On dit que $x \in E$ est adhérent à A s'il existe une suite ($a_n$) d'élements de A telque : 
$$a_n \underset{n \rightarrow \infty}\rightarrow x$$
x peut être ou ne pas être un élement de A.\\
Si d est une distance déduite d'une norme $\parallel~\parallel$, on peut écrire cette condition sous la forme : 
$$\parallel a_n - x \parallel \underset{n\rightarrow +\infty}\rightarrow 0$$
Par définition, tout point de A est adhérent à A
\end{de}
\begin{prop}
Nous avons la propriété suivante : 
$$( x \mbox{ est adhérent à A }) \Leftrightarrow (\forall \varepsilon > 0,~ B(x,\varepsilon) \cap A \neq \emptyset$$
On dit que x est adhérent à A si pour tout $\varepsilon >0$, B($x,\varepsilon$) rencontre A.
\end{prop}
\begin{de}
L'ensemble des points adhérent à A, s'appelle l'adhérence à A, et est noté $\overline{A}$. Par définition, nous avons donc : 
$$\overset{o}A c A c \overline{A}$$
\end{de}
\begin{de}
Un sous ensemble A d'un ensemble normé (E,d) est dit fermé s'il est égale à son adhérence, c'est à dire si A=$\overline{A}$. Et comme par définition nous avons l'une des inclusions, on obtient que A est fermé si et seulement si : 
$$\overline{A} c A$$
\end{de}
\begin{de}
On appelle complémentaire de A, et on le note $C_A$, l'ensemble défini par : 
$$C_A = E - A$$
\end{de}
\begin{prop}
Nous avons la propriété suivante :
\begin{center}
 A est fermé $\Leftrightarrow$ $C_A$ est ouvert
\end{center}
\end{prop}
\begin{prop}
Caractérisation séquentielle : \\
A est fermé si et seulement si toute suite convergente d'élement de A à sa limite dans A.
\end{prop}
\begin{prop}
Nous avons les propriétés suivantes : 
\begin{itemize}
 \item[$\rightarrow$] $\emptyset$ est un fermé
 \item[$\rightarrow$] E est un fermé
 \item[$\rightarrow$] La réunion d'un nombre fini de fermés est un fermé
 \item[$\rightarrow$] Une intersection quelconque de fermés est un fermé
\end{itemize}
\end{prop}
\begin{prop}
Nous avons les propriétes suivantes :
\begin{itemize}
 \item[$\rightarrow$] $\overline{A}$ est le plus petit fermé (au sens de l'inclusion) contenant A.
 \item[$\rightarrow$] Dans un espace vectorielle normé, la boule fermé $\overline{B}(x_0,r)$ est l'adhérence de B($x_0,r$). Celà n'est pas nécessairement vrai dans un espace métrique.
\end{itemize}
\end{prop}
\subsubsection{Exemples}
Il y a un exemple classique de fermé : 
\begin{itemize}
 \item[$\rightarrow$] Les boules fermés d'un espace métrique sont des fermés.
\end{itemize}
\section{Frontière}
\begin{de}
Soit A une partie non vide d'un espace métrique (E,d). On appelle frontière de A, et on le note parfois $\partial A$, l'ensemble défini par : 
$$\partial A = \overline{A} \cap \overline{C_A}$$
\end{de}
\begin{prop}
Nous avons la propriété suivante :
$$\partial A = \overline{A} - \overset{o}A$$
\end{prop}
\section{Diamètre d'une partie bornée}
\begin{de}
Une partie A d'un espace métrique (E,d) est dit borné si : 
$$\exists M \in \mathbb{R}_+,~\exists x_0 \in E,~tq~ \forall x \in A d(x_0,x) \leq M$$
Cette définition est équivalente à : 
$$\exists M \in \mathbb{R}_+,~\exists x_0 \in E,~tq~ A c \overline{B}(x_0,M)$$
\end{de}
\begin{prop}
Si cette condition est remplie, alors : 
$$\forall x_1 \in E,~ \exists M_1 \in \mathbb{R}_+,~ tq~\forall x \in A,~ d(x_1,x) \leq M_1$$
\end{prop}
\section{Ensembles compacts}
\section{Définitions et propriétés}
\subsection{Définition de Bolzano-Weierstrass}
\begin{de}
Un sous ensemble C d'un espace vectoriel normée E (ou un espace métrique (E,d), avec d la distance déduite de la norme), munie de la norme $\parallel~\parallel$, est dit compact si de toute suite $(\gamma_n)$ à valeur dans C, on peut extraire une sous-suite $(\gamma_{\phi(n)})$, avec $\phi$ une application strictement croissante de N dans N, qui converge vers un éléments de C.
\end{de}
\subsection{Théorème}
\begin{theo}
Tout segment [a,b] inclu dans $\Re$ est compact, c'est à dire que de toute suite bornée on peut extraire une suite qui converge.
\end{theo}
\begin{theo}
Nous avons les propriétes suivantes :
\begin{itemize}
 \item[$\rightarrow$] Tout compact est fermé et borné
 \item[$\rightarrow$] Tout fermé inclus dans un compact est compact
 \item[$\rightarrow$][$a_1,b_1$]$\times...\times$[$a_n,b_n$] est un compact de ($\mathbb{R}^n,\parallel~\parallel_{\infty}$), ou même de ($\mathbb{R}^n,\parallel~\parallel$) car toutes les normes sont équivalentes en dimension finie.
\end{itemize}
\end{theo}
\begin{theo}
Dans un espace vectoriel de dimension finie, tout ensemble, non vide, fermé et borné est un compact.
\end{theo}
\begin{prop}
Si $(x_n)$ est une suite d'élements d'un espace métrique convergeant vers l, alors : 
$$K = \left\lbrace x_n, n \in \mathbb{N} \right\rbrace \cup \left\lbrace l\right\rbrace  $$
est un compact.
\end{prop}
\begin{coro}
Soit f une application de E dans E', avec E et E' des espaces vectoriels normés ou des espaces métriques, définie sur une partie non vide A c E.\\
f est continue sur A si et seulement si f est continue sur tout compact inclu dans A.
\end{coro}
\begin{theo}
Soit f une fonction de E dans E', avec E et E' deux K espaces vectoriels (ou espace métrique) munie respectivement des normes $\parallel~\parallel$ et $\parallel~\parallel'$, définie et continue sur un compact C de E. Alors :
\begin{itemize}
 \item[$\rightarrow$] f(C) est un compact de E'
 \item[$\rightarrow$] f est bornée sur C, c'est à dire : $$\exists M \in R^+~ tq~ \forall \gamma \in C,~ \parallel f(\gamma)\parallel' \leq M$$
 \item[$\rightarrow$] f est uniformement continue sur C
 \item[$\rightarrow$] Si E'=$\Re$, alors f atteint ces bornes.
\end{itemize}
\end{theo}
\chapter{Espace Prehilbertiens, Espaces euclidiens}
Ce chapitre se réfère aux chapitres de MPSI et de MP sur les espaces vectoriels normés. Certaines propriétés établie précédement ne seront pas re-mentionné, mais font partie intégrante de ce chapitre. Dans le cours de MPSI, qui contient la grande majorité des élements non revu ici, on considère un espace euclidien. Mais ces propriétés, si elle n'ont pas été reproduite ici, s'étendent aux espaces $\mathbb{R}$ prehilbertien.
\section{Norme euclidienne}
\begin{de}
Soit E un $\mathbb{R}$ espace vectoriel.\\
On appelle norme euclidienne sur E une norme N telle qu'il existe un produit scalaire : 
$$\varphi : E\times E \rightarrow \mathbb{R}$$
vérifiant : 
$$\forall \overrightarrow{x} \in E,~ N(\overrightarrow{x}) = \sqrt{\varphi(\overrightarrow{x},\overrightarrow{x})}$$
\end{de}
\begin{de}
Un $\mathbb{R}$ espace vectoriel munie d'un produit scalaire, c'est à dire le couple (E,<|>), s'appelle un espace préhilbertien réel.\\
On appelle espace euclidien un espace préhilbertien réel de dimension finie.
\end{de}
\begin{prop}
On montre que l'application :
$$E \rightarrow \mathbb{R}_+$$
$$\overrightarrow{x} \mapsto \parallel\overrightarrow{x}\parallel = \sqrt{<\overrightarrow{x}|\overrightarrow{x}>}$$
est une norme sur E. Cette norme est appelée norme euclidienne déduite du produit scalaire <|>.
\end{prop}
\section{Propriétés Elémentaire}
\subsection{Identités de polarisation}
\begin{de}
On appelle identité de polarisation une égalité qui permet d'exprimer le produit scalaire au moyen de la norme euclidienne seule. Nous avons donc les égalités suivantes : 
$$<\overrightarrow{x}|\overrightarrow{y}> = \dfrac{1}{2}(\parallel\overrightarrow{x} + \overrightarrow{y}\parallel^2 - \parallel\overrightarrow{x}\parallel^2 - \parallel\overrightarrow{y}\parallel^2)$$
$$<\overrightarrow{x}|\overrightarrow{y}> = \dfrac{1}{4}(\parallel\overrightarrow{x} + \overrightarrow{y}\parallel^2 - \parallel\overrightarrow{x} - \overrightarrow{y}\parallel^2)$$
\end{de}
\subsection{Identités du Parallélogramme et de la médiane}
\begin{enon}
En généralisant la propriété en géométrie élémentaire, on obtient que dans le cas d'un $\mathbb{R}$ espace vectoriel prehilbertien, on a :
$$2.(\parallel\overrightarrow{x}\parallel^2 + \parallel\overrightarrow{y}\parallel^2) = \parallel\overrightarrow{x}+ \overrightarrow{y}\parallel^2 + \parallel\overrightarrow{x} - \overrightarrow{y}\parallel^2$$
On obtient aussi une égalité de la médiane, mais elle n'a pas de valeur ajouté par rapport à l'égalité précédente.
\end{enon}
\begin{prop}
Si une norme vérifié l'égalité ci-dessus, alors c'est une norme euclidienne
\end{prop}
\section{Forme linéaire dans un espace euclidien}
\begin{prop}
Si E est un $\mathbb{R}$ espace vectoriel préhilbertien, muni du produit scalaire <|>, alors :\\
$\forall \overrightarrow{e} \in E$, l'application :
$$\varphi_{\overrightarrow{e}} : E \rightarrow \mathbb{R}$$
$$\overrightarrow{x} \rightarrow <\overrightarrow{e}|\overrightarrow{x}>$$
est une forme linéaire continue non nulee si et seulement si $\overrightarrow{e} \neq \overrightarrow{0}$
\end{prop}
\begin{prop}
Si E est un $\mathbb{R}$ espace vectoriel euclidien, alors pour toute forme linéaire $\varphi \in E^*$, l'ensemble des formes linéaires sur E, il existe un unique vecteur $\overrightarrow{e} \in E$ telque : 
$$\varphi = \varphi_e$$
C'est à dire telque : 
$$\forall \overrightarrow{x} \in E, <\overrightarrow{e}|\overrightarrow{x}>$$
\end{prop}
\begin{coro}
Tout hyperplan d'un $\mathbb{R}$ espace vectoriel euclidien est l'orthogonal d'une droite vectorielle.
\end{coro}
\section{Théorème de projection orthogonale sur un sous espace de dimension finie}
Cette section s'appuit fortement sur la section "Projection orthogonale" dans le livre de révision de Mathématiques de MPSI.
\subsection{Inégalité de Bessel}
Avec les notations présenté dans l'ouvrage MPSI, on a : 
$$\parallel p(\overrightarrow{x}) \parallel^2 = \sum_{i=1}^p <\overrightarrow{e_i}|\overrightarrow{x}>^2$$
On obtient donc que : 
$$\sum_{i=1}^p <\overrightarrow{e_i}|\overrightarrow{x}>^2 \leq \parallel\overrightarrow{x}\parallel^2$$
Ceci constitue l'inégalité de Bessel.
\subsection{Norme d'un projecteur orthogonal subordonnée à la norme euclidienne}
Soit E un $\mathbb{R}$ espace vectoriel préhilbertien, et $\parallel~\parallel$ la norme euclidienne de E.\\
Soit F un sous espace vectoriel de E, de dimension fini, et p le projecteur orthogonal sur F. Alors : 
$$\parallel p \parallel_{*} = 1$$
Si F $\neq \left\lbrace \overrightarrow{0}\right\rbrace$, avec, par définition : 
$$\parallel p \parallel_* = \underset{\overrightarrow{x} \in E - \left\lbrace \overrightarrow{0}\right\rbrace}\sup \dfrac{\parallel p(\overrightarrow{x})\parallel}{\parallel\overrightarrow{x}\parallel}$$
\subsection{Projection orthogonale sur une droite vectorielle}
Soit $D=Vect(\overrightarrow{u})$ une droite vectorielle d'un $\mathbb{R}$ espace vectoriel préhilbertien E, et p le projecteur orthogonal sur D. Alors : 
$$\forall \overrightarrow{x} \in E,~ p(\overrightarrow{x}) = \dfrac{<\overrightarrow{x}|\overrightarrow{u}>}{\parallel\overrightarrow{\overrightarrow{u}}\parallel^2}.\overrightarrow{u}$$
\subsection{Théorème de la base orthonormée incomplète dans un espace vectoriel euclidien}
Soit E un $\mathbb{R}$ espace vectoriel euclidien de dimension n et ($\overrightarrow{e_1},...,\overrightarrow{e_p}$) un système orthonormé de E. Si p<n, alors il existe un système orthonormée ($\overrightarrow{e_{p+1}},...,\overrightarrow{e_n}$) telque ($\overrightarrow{e_1},...,\overrightarrow{e_n}$) soit une base orthonormée de E.
\section{Orthogonal d'une partie, sous-espaces orthogonaux}
\begin{coro}
Pour que $\overrightarrow{x}$, un vecteur de E, soit orthogonal à un sous espace vectoriel F, il faut et il suffit que $\overrightarrow{x}$ soit orthogonal à une famille génératrice de F.
\end{coro}
\subsection{Propriétés}
Soit E un $\mathbb{R}$ espace vectoriel prehilbertien. Soit A et B deux parties de E, F et G de sous espaces vectoriel de E. Nous avons les propriétés suivantes : 
\begin{itemize}
 \item[$\rightarrow$] A c B $\Rightarrow$ $A^{\bot}$ c $B^{\bot}$
 \item[$\rightarrow$] $(F + G)^{\bot} = F^{\bot} + G^{\bot}$. Cette propriété se généralise pour un plus grand nombre de sous espace.
 \item[$\rightarrow$] $F^{\bot} + G^{\bot}$ c $(F\cap G)^{\bot}$. Si E est un espace de dimension finie, il y a égalité. Cette propriété se généralise elle aussi. 
 \item[$\rightarrow$] F c $F^{\bot\bot}$. Il y a égalité dans le cas d'un espace de dimension finie.
\end{itemize}
\subsection{Sous-espaces Vectoriels orthogonaux}
\begin{de}
On dit que des sous espaces vectoriels $F_1,...,F_p$ d'un $\mathbb{R}$ espace vectoriel prehilbertien E sont supplémentaire orthogonaux s'ils sont supplémentaire et orthogonaux deux à deux. On le note : 
$$E = F_1 \overset{\bot}\oplus ... \overset{\bot}\oplus F_p$$
\end{de}
\begin{prop}
Si les $F_i$ précédents sont des sous espaces vectoriel deux à deux orthogonaux, et si $B_i$ est une famille orthogonale (respectivement orthonormée) de $F_i$, alors $B_1 \vee ... \vee B_p$ est une famille orthogonale ( respectivement orthonormée) de $F_1 \overset{\bot}\oplus ... \overset{\bot}\oplus F_p$.\\
Il en est de même si on considère une base au lieu d'une famille
\end{prop}
\begin{prop}
Si $B_1,...,B_p$ sont des familles orthogonales (respectivement orthnormée) telque $\forall i \neq j \in [1,p]$, tout vecteur de $B_i$ soit orthogonal à tout vecteur de $B_j$, alors les $F_i = Vect(B_i)$ sont des sous espaces vectoriels deux à deux orthogonaux et $B_1\vee...\vee B_p$ est une base orthogonale (respectivement orthonormée) de :
$$F_1 \overset{\bot}\oplus ... \overset{\bot}\oplus F_p$$
\end{prop}
\section{Orthonormalisation de Gram-Schmidt}
\begin{theo}
Soit $(u_i)$ une famille libre d'un $\mathbb{R}$ espace vectoriel préhilbertien E, avec $i \in I = [1,n]$ ou $i \in I = \mathbb{N}^*$.\\
Alors, il existe une unique famille orthonormée ($e_i$) telle que : 
\begin{itemize}
 \item[$\rightarrow$] $\forall i \in I,~ Vect(\overrightarrow{e_1},...,\overrightarrow{e_i}) = Vect(\overrightarrow{u_1},...,\overrightarrow{u_i})$
 \item[$\rightarrow$] $\forall i \in I,~ <\overrightarrow{u_i}|\overrightarrow{e_i}> = 0$
\end{itemize}
L'unicité provient de la seconde condition.
\end{theo}
\subsection{Traduction matricielle}
\begin{prop}
$\forall A \in Gl_n(\mathbb{R}),~ \exists!(Q,R) \in \mathcal{M}_n(\mathbb{R})^2$ avec Q orthogonale et R triangulaire supérieur à diagonale strictement positive telque : 
$$A = QR$$
A l'aide de cette propriété, on peut traduire matriciellement l'orthonormalisation de Gram-Schmidt.
\end{prop}
\section{Endomorphismes Orthogonaux, Matrices orthogonales}
Dans cette section, on généralise les résultats vu en MPSI, dans le cas ou l'espace de départ n'est pas forcémement l'espace d'arrivé.
\subsection{Isométries vectorielles}
\begin{prop}
Soient E et E' deux $\mathbb{R}$ espace vectoriel préhilbertiens de f une application de E dans E', qui n'est pas supposé linéaire. On a alors équivalence entre les deux conditions suivantes : 
\begin{itemize}
 \item[$\rightarrow$] f conserve le produit scalaire : $\forall \overrightarrow{x},\overrightarrow{x}' \in E^2~ <f(\overrightarrow{x})|f(\overrightarrow{x}')> = <\overrightarrow{x}|\overrightarrow{x}'>$
 \item[$\rightarrow$] f est linéaire et conserve la norme : $\forall \overrightarrow{x} \in E,~ \parallel f(\overrightarrow{x}\parallel = \parallel\overrightarrow{x}\parallel$
\end{itemize}
\end{prop}
\begin{de}
Une telle application f est appelé isométrie vectorielle de E dans E'.
\end{de}
\begin{prop}
Toute isométrie vectorielle est injective.
\end{prop}
\subsection{Matrices orthogonales}
\begin{de}
Une matrice orthogonale est une matrice $A \in \mathcal{M}_n(\mathbb{R})$ telque $^tA.A = I_n$. On note $O_n(\mathbb{R})$ l'ensembles des matrices orthogonales de $\mathcal{M}_n(\mathbb{R})$. D'apres la caractérisation précédente, on obtient que l'inverse d'une matrice orthogonale est sa transposé.
\end{de}
\begin{prop}
$(O_n(\mathbb{R}),\times)$ est un sous groupe de ($Gl_n(\mathbb{R},\times)$. Nous avons donc les propriétés suivantes : 
\begin{itemize}
 \item[$\rightarrow$] $I_n$ est une matrice orthogonale
 \item[$\rightarrow$] Le produit de deux matrices orthogonales est une matrice orthogonales
 \item[$\rightarrow$] L'inverse d'une matrice orthogonale est une matrice orthogonale
\end{itemize}
\end{prop}
\begin{prop}
Soit E un espace vectoriel euclidien. Nous avons les propriétés suivantes : 
\begin{itemize}
 \item[$\rightarrow$] Si $f \in O(E)$, alors la matrice de f dans n'importe quelle base orthonormée est orthogonale
 \item[$\rightarrow$] Si $f \in \mathcal{L}(E)$, et si il existe une base orthonormée dans laquelle f est représenté par une matrice orthogonale, alors $f \in O(E)$
\end{itemize}
\end{prop}
\subsubsection{Lien avec les bases orthonormées}
\begin{prop}
Nous avons les propriétés suivantes :
\begin{itemize}
 \item[$\rightarrow$] Si E est un espace vectoriel euclidien, la matrice de passage d'une base orthonormée à une autre base orthonormée est une matrice orthogonale.
 \item[$\rightarrow$] Si B est une base orthonormée de E et si la matrice de passage entre B et B' est une matrice orthogonale, alors B' est aussi une base orthonormée de E.
 \item[$\rightarrow$] $A \in \mathcal{M}_n(\mathbb{R})$ est une matrice orthogonale si et seulement si ses colonnes forment une base orthonormée dans $\mathbb{R}^n$ euclidien canonique. Il en est de même si on considère les lignes.
\end{itemize}
\end{prop}
\subsubsection{Déterminant d'un endomorphisme orthogonal}
\begin{prop}
Si $A \in O_n(\mathbb{R})$, alors : 
$$det(A) = \pm 1$$
\end{prop}
\begin{prop}
Si $f \in O(E)$, avec E un $\mathbb{R}$ espace vectoriel euclidien, alors :
$$det(f) = \pm 1$$
\end{prop}
\begin{prop}
Les endomorphismes orthogonaux de déterminant +1 sont dits directs ou positif.\\
Les endomorphismes orthogonaux de déterminant -1 sont dits indirects ou négatif.\\
\end{prop}
\subsubsection{Valeurs propres d'un endomorphisme orthogonal ou d'une matrice orthogonale}
Soit $A \in O_n(\mathbb{R})$
\begin{prop}
Les valeurs propres complexes d'une matrice orthogonale sont de module 1.
\end{prop}
\subsubsection{Symétries orthogonales}
\begin{de}
Soit E un $\mathbb{R}$ espace vectoriel euclidien de dimension finie n et F un sous espace vectoriel de E. Nous avons donc : 
$$E = F \oplus F^{\bot}$$
On peut donc définir la symétrie orthogonale s par rapport à F et parralèlement de $F^{\bot}$ : 
$$s : E \rightarrow E$$
$$\overrightarrow{x} = \overrightarrow{x_1} + \overrightarrow{x_2} \rightarrow s(\overrightarrow{x}) = \overrightarrow{x_1} - \overrightarrow{x_2}$$
Avec $\overrightarrow{x_1} \in F$ et $\overrightarrow{x_2} \in F^{\bot}$.
\end{de}
\begin{prop}
Nous avons la propriété suivante : 
$$s \in O(E)$$
\end{prop}
\subsubsection{Réduction orthonormale d'un endomorphisme orthogonal}
\begin{theo}
Soit $f \in O(E)$, avec E un $\mathbb{R}$ espace vectoriel euclidien. Alors, il existe au moins une base orthonormée B de E telle que : 
\[mat_B(f) = \begin{blockarray}{ccc}
        \begin{block}{(ccc)}
         \mesbloc{R_1}  &  & (0)   \\
                            &\ddots      &  \\
         (0)                &  & \mesbloc{R_p}  \\
        \end{block}
        \end{blockarray}
\] 
Avec $R_k = (1)$ ou $R_k = (-1)$ ou :
\[R_k = \begin{blockarray}{cc}
        \begin{block}{(cc)}
         cos(\theta_k)  &  -sin(\theta_k)   \\
         sin(\theta_k)  & cos(\theta_k)  \\
        \end{block}
        \end{blockarray}
\] 
$O_k \neq 0~ [\pi]$. Dans ce dernier cas, $R_k$ représente une rotation d'angle $\theta_k$
\end{theo}

\chapter{Adjoint d'un endomorphisme, Endomorphisme et matrice symétrique}
\section{Rappel}
Soit E un $\mathbb{R}$ espace vectoriel, B=($\overrightarrow{e_1},\dots,\overrightarrow{e_n}$) une base orthonormé de E. Soit x et x' deux vecteurs de E. Soit X et X' les matrices de x et x' dans B. On obtient que : 
$$<x|x'> = ^tX.X'$$
\section{Propriétés - Définition d'un adjoint}
\begin{prop}
Soit E un $\mathbb{R}$ espace vectoriel euclidien de $f \in \mathcal{L}(E)$. Alors :
$$\exists ! f^* \in \mathcal(E)~ tq~ \forall (\overrightarrow{x},\overrightarrow{y})\in E^2~ <f(\overrightarrow{x})|\overrightarrow{y}> = <\overrightarrow{x}|f^*(\overrightarrow{\overrightarrow{y}})>$$
\end{prop}
\begin{de}
$f^*$ est appelé l'adjoint de f, pour un produit scalaire défini.
\end{de}
\begin{prop}
Si E est un $\mathbb{R}$ espace vectoriel euclidien, et B une base orthonormée de E, alors : 
$$mat_B f^* = ^t(mat_B f)$$ 
\end{prop}
\begin{de}
Un endomorphisme f d'un $\mathbb{R}$ espace vectoriel euclidien E est dit auto-adjoint ou symétrique si :
$$f^* = f$$
\end{de}
\begin{prop}
Soit $f \in \mathcal{L}(E)$, avec E un $\mathbb{R}$ espace vectoriel euclidien.
\begin{itemize}
 \item[$\rightarrow$] Si f est symétrique, alors quelque soit la base orthonormée B de E, la matrice de f dans cette base est symétrique.
 \item[$\rightarrow$] S'il existe une base orthonormée B de E telque la matrice de f soit symétrique, alors f est symétrique
\end{itemize}
\end{prop}
\section{Propriétés élémentaires}
Dans tout ce paragraphe, E désigne un $\mathbb{R}$ espace vectoriel euclidien. f et g désigne des endomorphisme de E.
\begin{prop}
Nous avons les propriétés suivantes :
\begin{itemize}
 \item[$\rightarrow$] $(f+g)^* = f^* + g^*$
 \item[$\rightarrow$] $\forall \lambda \in \mathbb{R}$ $(\lambda.f)^* = \lambda.f^*$
 \item[$\rightarrow$] $(fog)^* = g^*of^*$
 \item[$\rightarrow$] $(f^*)^* = f$
\end{itemize}
\end{prop}
\begin{prop}
Nous avons les propriétés suivantes : 
$$Ker(f^*) = (Im(f))^{\bot}$$
$$Im(f^*) = (Ker(f))^{\bot}$$
\end{prop}
\begin{prop}
Si $\lambda \in Sp(f)$ et $\lambda' \in Sp(f^*)$. Si $\lambda \neq \lambda'$, alors : \\
Ker($f-\lambda.Id$) et Ker($f^* - \lambda'.Id$) sont orthogonaux.
\end{prop}
\begin{prop}
Soit F un sous espace vectoriel de E.
$$(\mbox{ F est stable par f }) \Leftrightarrow ( F^{\bot}\mbox{ est un sous espace vectoriel stable par f })$$
\end{prop}
\subsection{Cas ou f et g sont symétrique}
\begin{coro}
L'ensemble des endomorphismes symétriques de E est un sous espace vectoriel de $\mathcal{L}(E)$.
\end{coro}
\begin{coro}
Nous avons les égalités suivantes : 
$$Ker(f) = (Im(f))^{\bot}$$
$$Im(f) = (Ker(f))^{\bot}$$
\end{coro}
\begin{coro}
Si $\lambda$ et $\lambda'$ sont deux valeurs propres distinct d'un endomorphisme f symétrique, alors : \\
Ker($f-\lambda.Id$) et Ker($f - \lambda'.Id$) sont orthogonaux.
\end{coro}
\begin{coro}
Si f est symétrie et si F est un sous espace vectoriel de E : 
$$(\mbox{ F est stable par f }) \Leftrightarrow ( \mbox{ F est un sous espace vectoriel stable par f })$$
\end{coro}
\section{Théorème d'orthogonalisation}
\begin{theo}
Soit E un $\mathbb{R}$ espace vectoriel euclidien de dimension finie n. Soit f un endomorphisme symétrique de E. Alors : 
\begin{itemize}
 \item[$\rightarrow$] $P_f$ est scindé sur $\mathbb{R}$, c'est à dire que le spectre complexe de f est égale au spectre réel de f.
 \item[$\rightarrow$] Les sous espaces vectoriels propre de f sont deux à deux orthogonaux et supplémentaires. En particulier, f est diagonalisable.
 \item[$\rightarrow$] Il existe une base orthonormée de diagonalisation de f, c'est à dire une base orthonormée de E formée de vecteurs propres de f.
 \item[$\rightarrow$] Réciproquement, si f est un endomorphisme de E orthonormalement diagonalisable, c'est à dire si f admet une base orthonormée de diagonalisation, alors f est symétrique.
\end{itemize}
\end{theo}
\subsection{Corollaire matricielle}
Si $A \in \mathcal{M}_n(\mathbb{R})$ est symétrique, alors A est orthonormalement diagonalisable, c'est à dire qu'il existe $P \in O_n(\mathbb{R})$ telque $P^{-1}.A.P$ soit diagonale. Dans ce cas $P^{-1} = P^{\bot}$
\section{Caractéristation par l'adjoint de certains endomorphismes classiques d'un espace euclidien}
Dans tout ce paragraphe, E désigne un $\mathbb{R}$ espace vectoriel euclidien, $f \in \mathcal{L}(E)$.
\begin{prop}
f est un projecteur orthogonal si et seulement si : 
$$\begin{cases}
   fof = f \\
   f^* = f \\
  \end{cases}
$$
\end{prop}
\begin{prop}
Nous avons la propriété suivante : 
$$(f \in O(E) ) \Leftrightarrow (fof^* = Id_E)$$
\end{prop}
\begin{prop}
Nous avons la propriété suivante : 
$$\mbox{ f est symétrique orthogonale } \Leftrightarrow \begin{cases}
                                                           f^2 = Id \\
							   f^* = f \\
                                                          \end{cases}
$$
\end{prop}

\chapter{Formes quadratiques}
\section{Formes quadratiques sur $\mathbb{R}^n$}
\begin{de}
On appelle forme quadratique sur $\mathbb{R}^n$ toute application polynomiale homogène du second degrès de $\mathbb{R}^n$ dans $\mathbb{R}$, donc une application défini par : 
$$\mathbb{R}^n \overset{q}\rightarrow \mathbb{R}$$
$$X \mapsto q(X)$$
Si :
$$X = \begin{pmatrix}
       x_1\\
 	\vdots \\
	x_n
      \end{pmatrix}
$$
Alors : 
$$q(X) = \sum_{i=1}^n a_{ii}.x_i^2 + 2.\sum_{1 \leq i < j \leq n} a_{ij}.x_j.x_i$$
Avec : 
$$\forall i > j,~ a_{ij} = a_{ji}$$
\end{de}
\subsection{Forme bilinéaire symétrique associé à une forme quadratique}
\begin{prop}
Si q est une forme quadratique sur $\mathbb{R}^n$, il existe une unique forme bilinéaire symétrique $\varphi$ sur $\mathbb{R}^n$ telle que : 
$$\forall X \in \mathbb{R}~ q(X) = \varphi(X,X)$$
$\varphi$ est appelé forme polaire de q.\\
On montre que l'unicité est du à l'identité de polarisation, qui permet d'exprimer $\varphi$ explicitement en fonction de q.
\end{prop}
\subsection{Expressions matricielles de q et de $\varphi$}
\begin{prop}
Soient q et $\varphi$ une forme quadratique et sa forme polaire associé sur $\mathbb{R}^n$, alors : 
$$\forall X \in \mathbb{R}^n~ q(X) = ^tX.A.X$$
$$\forall(X,X') \in \mathbb{R}^n~ \varphi(X,X') = ^tX.A.X'$$
Avec A matrice symétrique.
\end{prop}
\begin{prop}
Il existe une unique matrice $A \in \mathcal{M}_n(\mathbb{R})$ symétrique telque : 
$$\forall X \in \mathbb{R}^n~ q(X) = ^tX.A.X$$
A est appelé la matrice de la forme quadratique q dans la base canonique, et se note :
$$A = mat_{can}(q)$$
De même, A est l'unique matrice symétrique de $\mathcal{M}_n(\mathbb{R})$ telque : 
$$\forall(X,X') \in (\mathbb{R}^n)^2~ \varphi(X,X') = t^X.A.X'$$
On obtient :
$$A = mat_{can}(q) = mat_{can}(\varphi) = (a_{ij})$$
Avec : 
$$\forall (i,j) \in [|1,n|]^2~ a_{ij} = \varphi(\overrightarrow{e_i},\overrightarrow{e_j})$$
Avec ($\overrightarrow{e_1},...,\overrightarrow{e_n}$) la base canonique de $\mathbb{R}^n$.
\end{prop}
\subsection{Endomorphisme symétrique de $\mathbb{R}^n$ associé à une forme quadratique q}
$\mathbb{R}^n$ est munie du produit scalaire canonique. q est une forme quadratique sur $\mathbb{R}^n$, de forme polaire $\varphi$, de matrice A sur la base canonique.
\begin{prop}
Il existe un unique endomorphisme f de $\mathbb{R}^n$ symétrique telque : 
$$\forall X \in \mathbb{R}^n~ q(X) = <f(X)| X >$$
Cet endomorphisme vérifie aussi : 
$$\forall (X,X') \in (\mathbb{R}^n)^2~ <f(X),X'> = <X|f(X')> = \varphi(X,X')$$
f est appelé l'endomorphisme symétrique associé à q dans $(\mathbb{R}^n,<|>)$. De plus, nous avons : 
$$mat_{cam}(f)=mat_{cam}(q)$$
\end{prop}
\subsection{Théorème de réduction orthonormale d'une forme quadratique de $\mathbb{R}^n$}
\begin{theo}
$\mathbb{R}^n$ étant munie de son produit scalaire canonique euclidien, et q étant une forme quadratique sur $\mathbb{R}^n$, il existe B=($\overrightarrow{u_1},...,\overrightarrow{u_n}$) base orthonormée de $\mathbb{R}^n$ telque si :
$$X = x_1.\overrightarrow{u_1} + ... + x_n.\overrightarrow{u_n}$$
alors : 
$$q(X) = \lambda_1.x_1'^2 + ... + \lambda_n.x_n'^2$$
ou les $\lambda_i$ sont des valeurs propres associés à f, l'endomorphisme symétrique canoniquement associé à q. Une telle base B s'obtient en orthonormalisent f. En résumé, par un changement de base, on fait "disparaitre" les termes rectangles ($x_i.x_j$ ...)
\end{theo}
\subsection{Version matricielle du théorème de réduction orthonormale}
Soit q une forme quadratique sur $\mathbb{R}^n$, et $\varphi$ sa forme polaire. Soit A la matrice de q dans la base canonique, et f l'endomorphisme symétrique canoniquement associé à q dans $\mathbb{R}^n$ euclidien canonique.\\
Si $X \in \mathbb{R}^n$ : 
$$q(X) = ^tX.A.X = <f(X) | X>$$
\subsubsection{Changement de base}
Soit B une nouvelle base de $\mathbb{R}^n$, B=($\overrightarrow{u_1},...,\overrightarrow{u_n}$) et P la matrice de passage entre la base canonique et B. Soit $X' = mat_B(X)$. On sait que : 
$$X = PX'$$
Donc : 
$$q(X) = ^tX.(^tP.A.P).X'$$
Posons A' = $^tP.A.P$. A' est symétrique et c'est l'unique matrice symétrique vérifiant l'expression précédent $\forall X \in \mathbb{R}^n$. Par définition : 
$$A' = mat_B(q)$$
\begin{prop}
Si A=($a'_{ij}$), alors :
$$\forall (i,j) \in [|1,n|]^2~a'_{ij} = \varphi(\overrightarrow{u_i},\overrightarrow{u_j}) $$
De plus, si B est une base orthonormée, alors : 
$$A- = ^tP.A.P = P^{-1}.A.P$$
\end{prop}

\section{Généralisation : Formes quadratiques sur un espace vectoriel E}
\begin{de}
Soit E un $\mathbb{R}$ espace vectoriel quelconque, non nécessairement de dimension finie. On appelle forme quadratique sur E une application q : E $\rightarrow$ $\mathbb{R}$ tel qu'il existe $\varphi$ application de $E\times E \rightarrow \mathbb{R}$ bilinéaire symétrique vérifiant : 
$$\forall \overrightarrow{x} \in E, q(\overrightarrow{x}) = \varphi(\overrightarrow{x},\overrightarrow{x})$$
\end{de}

\begin{prop}
Avec les notations précédentes, $\varphi$ est alors unique et appelé forme polaire q.
\end{prop}
\begin{prop}
En conservant les notations précédentes : 
$$\forall (\overrightarrow{x},\overrightarrow{y}) \in E^2~ \varphi(\overrightarrow{x},\overrightarrow{y}) = \dfrac{1}{4}[q(\overrightarrow{x} + \overrightarrow{y}) - q(\overrightarrow{x}-\overrightarrow{y})]$$ 
\end{prop}
\begin{prop}
Si q est une forme quadratique sur E, $\forall \lambda \in \mathbb{R}$ :
$$\forall \overrightarrow{x} \in E~ q(\lambda.\overrightarrow{x}) = \lambda^2.q(\overrightarrow{x})$$ 
\end{prop}
\subsection{Exemples}
Les formes quadratiques défini à la section précédente sur $\mathbb{R}^n$ sont des formes quadratiques au sens de la nouvelle définition générale. Réciproquement, toute forme quadratique sur $E = \mathbb{R}^n$ au sens de la nouvelle définition est une application polynomiale homogène du $2^{nd}$ degrès. En résumé, les formes quadratiques définies au paragraphe 1 sur $\mathbb{R}^n$ sont des cas particuliers des formes quadratiques défini globalement.
\subsection{Expression dans une base, matrice d'une forme quadratique dans une base}
\begin{prop}
Soit q, application de E dans $\mathbb{R}$, une forme quadratique sur un $\mathbb{R}$ espace vectoriel E de dimension n. $\varphi$ est sa forme polaire. Soit B=($\overrightarrow{u_1},...,\overrightarrow{u_n}$) une base de E. Si : 
$$\begin{cases}
   \overrightarrow{x} = x_1.\overrightarrow{u_1} + ... + x_n.\overrightarrow{u_n} \\
   \overrightarrow{y} = y_1.\overrightarrow{u_1} + ... + y_n.\overrightarrow{u_n} \\
  \end{cases}
$$
Alors : 
$$
\begin{cases}
 q(\overrightarrow{x}) = \sum_{i=1}^n a_{ii}.x_i^2 + 2.\sum_{1 \leq i < j \leq n} a_{ij}.x_i.x_j \\
 \varphi(\overrightarrow{x},\overrightarrow{y}) = \sum_{(i,j) \in [|1,n|]^2} a_{ij}.x_i.x_j \\
\end{cases}
$$
Réciproquement, si q est une application E dans $\mathbb{R}$, alors : 
$$\forall \overrightarrow{x} \in E~ q(\overrightarrow{x}) \sum_{i=1}^n a_{ii}.x_i^2 + 2.\sum_{1\leq i<j\leq n} a_{ij}.x_i.x_j$$
alors q est une forme quadratique et sa forme polaire $\varphi$ est définie par : 
$$\varphi : E\times E \rightarrow \mathbb{R}$$
$$(\overrightarrow{x},\overrightarrow{y}) \mapsto \sum_{(i,j) \in [1,n]^2} a_{ij}.x_i.x_j$$
\end{prop}
\begin{de}
Avec les notations précédentes, la matrice A=($a_{ij}$) telque :
$$\forall (i,j) \in [1,n]~ a_{ij}= \varphi(\overrightarrow{u_i},\overrightarrow{u_j})$$
A est appelé matrice de q ou de $\varphi$ dans la base B.
\end{de}
\begin{prop}
Avec les notations et définitions précédentes, A est l'unique matrice symétrique de $\mathcal{M}_n(\mathbb{R})$ telque :
$$\forall \overrightarrow{x} = \sum_{i=1}^n x_i.\overrightarrow{u_i} \in E, q(\overrightarrow{x}) = ^tX.A.X$$
Avec X la matrice de $\overrightarrow{x}$ dans B.
\end{prop}
\subsection{Changement de Base}
Soit E un $\mathbb{R}$ espace vectoriel de dimension finie n, B et B' deux bases de E. Soit P la matrice de passage entre B et B'. q est une forme quadratique sur E. A est la matrice de q dans B, A' la matrice de q dans B' : 
$$A' = t^P.A.P$$
\begin{de}
Si A et A' deux matrices de $\mathcal{M}_n(\mathbb{R})$. On dit que A et A' sont congruente s'il existe $P \in Gl_n(\mathbb{R})$ telque : 
$$A' = ^tP.A.P$$
On défini ainsi une relation d'équivalence sur E. Deux matrices congruentes sont en particulier équivalentes. Donc deux matrices congruentes ont même rang.
\end{de}
\begin{de}
Soit q une forme quadratique sur un $\mathbb{R}$ espace vectoriel de dimension finie. On appelle rang de q le rang de sa matrice sur une base B de E. Cette définition ne dépend pas de la base B choisie.
\end{de}
\begin{de}
Si E est un $\mathbb{R}$ espace vectoriel de dimension finie n et q une forme quadratique sur E. On dit que q est non dégénéré si rang(q)=n, c'est à dire si la matrice de q sur une base est inversible.
\end{de}
\subsection{Endomorphisme symétriques associés à une forme quadratique dans un espace vectoriel euclidien}
\begin{prop}
Soit q une forme quadratique sur un $\mathbb{R}$ espace vectoriel euclidien E, alors il existe un unique endomorphisme f$\in \mathcal{L}(E)$, symétrique ( pour le produit scalaire de E) telque : 
$$\forall \overrightarrow{x} \in E~ q(\overrightarrow{x}) = <f(\overrightarrow{x})|\overrightarrow{x}>$$
En outre, quelque soit la base B orthonormée de E : 
$$mat_B(f) = mat_B(q)$$
De plus, si $\varphi$ est la forme polaire de q :
$$\forall (\overrightarrow{x},\overrightarrow{y})\in E^2,~ \varphi(\overrightarrow{x},\overrightarrow{y}) = <f(\overrightarrow{x}) | \overrightarrow{x}> = <\overrightarrow{x} | f(\overrightarrow{x})>$$
\end{prop}
\subsection{Réduction orthonormale d'une forme quadratique dans un espace vectoriel euclidien}
\begin{theo}
Soit E un $\mathbb{R}$ espace vectoriel euclidien de dimension n, q une forme quadratique sur E, alors il existe une base B orthonormée de E et des réels $\lambda_1,...,\lambda_n$ telque : 
$$\forall \overrightarrow{x} = x_1.\overrightarrow{u_1} + ... + x_n.\overrightarrow{u_n} \in E~ q(\overrightarrow{x}) = \sum_{i=1}^n \lambda_i.x_i^2$$
Une telle base B s'obtient en orthonormalisent l'endomorphisme symétrique f associé à q.\\
Avec les notations précédentes, si $\overrightarrow{y} = y_1.\overrightarrow{u_1} + ... + y_n.\overrightarrow{u_n}$ et si $\varphi$ est la forme polaire de q : 
$$\varphi(\overrightarrow{x},\overrightarrow{y}) = \sum_{i=1}^n \lambda_i.x_i.y_i$$ 
\end{theo}
\section{Application à la réduction de coniques}
\begin{de}
Soit E un $\mathbb{R}$ espace affine de dimension 2. Soit $\mathcal{R}$ un repère de E. On appelle conique au sens large un sous ensemble $(\Gamma)$ de E admettent dans $\mathcal{R}$ une équation du type : 
$$P(x,y) = 0$$
Avec P polynome du second degrès : 
$$P(x,y) = a.x^2 + 2.b.x.y + c.y^2 + u.x + v.y + h$$
Avec $(a,b,c) \neq (0,0,0)$. Soit $\Omega = (O,\overrightarrow{i},\overrightarrow{j})$. On défini une forme quadratique q par : 
$$\overrightarrow{E} \rightarrow \mathcal{R}$$
$$x.\overrightarrow{i} + y.\overrightarrow{j} \mapsto a.x^2 + 2.b.x.y + c.y^2$$
Avec $\overrightarrow{E}$ l'espace vectoriel associé à E. Nous avons donc : 
$$mat_{(\overrightarrow{i},\overrightarrow{j})}(q) = \begin{pmatrix}
                                                      a & b \\
						      b & c \\
                                                     \end{pmatrix}
$$
On défini de même une forme linéaire l par : 
$$\overrightarrow{E} \rightarrow \mathcal{R}$$
$$x.\overrightarrow{i} + y.\overrightarrow{j} \mapsto u.x + v.y$$
Avec ces notations, l'équation de $(\Gamma)$ s'écrit : 
$$M \in (\Gamma) \Leftrightarrow q(\overrightarrow{OM}) + l(\overrightarrow{OM}) + h = 0$$
\end{de}
\begin{prop}
Si $\mathcal{R}'$ est un autre repère de (E) si ($\Gamma$) a pour équation P(x,y)=0, avec P polynome du $2^{nd}$ degrès. dans le repère $\mathcal{R}$, alors ($\Gamma$) a pour équation :
$$Q(x,y) = 0$$
avec Q polynome du $2^{nd}$ degrès. Autrement dit, la définition ci-dessus ne dépend pas du repère choisi.
\end{prop}
\subsection{Cas général}
On suppose ici E euclidien et $\mathcal{R}$ un repère orthonormée. En réduisant d'abord orthonormalement q, ce qui revient à un changement de repère par rotation, puis en effectuant éventuellement un deuxième changement de repère par translation, on montre qu'il existe un repère orthonormée $\mathcal{R}_1$ dans lequel $(\Gamma)$ à une équation du type (a,b>0):
\begin{itemize}
 \item[$\rightarrow$] Coniques non dégénéré :
\begin{itemize}
 \item[$\rightarrow$]$\dfrac{x_1^2}{a^2} + \dfrac{y_1^2}{b^2} = 1$. C'est une ellipse. q a deux valeurs propres de même signe
 \item[$\rightarrow$]$\dfrac{x_1^2}{a^2} - \dfrac{y_1^2}{b^2} = 1$. C'est une hyperbole. q a deux valeurs propres de signe contraire
 \item[$\rightarrow$]$x_1^2 = 2.p.y_1$. C'est une parabole. q a 1 valeur propre nulle.
\end{itemize}
 \item[$\rightarrow$] Coniques dégénéré :
\begin{itemize}
 \item[$\rightarrow$] $\dfrac{x_1^2}{a^2} + \dfrac{y_1^2}{b^2} = 0$. ($\Gamma$) est réduit au point (0,0).
 \item[$\rightarrow$] $\dfrac{x_1^2}{a^2} + \dfrac{y_1^2}{b^2} = \alpha < 0$. ($\Gamma$) est égale au vide. 
 \item[$\rightarrow$] $\dfrac{x_1^2}{a^2} - \dfrac{y_1^2}{b^2} = 0$. ($\Gamma$) est la réunion de deux droites sécantes.
\end{itemize}
\end{itemize}
Pour déterminer la conique associé à $(\Gamma)$, on peut utiliser la méthode suivantes : Soit A la matrice de q ou de f dans une base orthonormée de dimension 2. Soit $\lambda_1$ et $\lambda_2$ les valeurs propres associé à A. On montre que :
\begin{itemize}
 \item[$\rightarrow$] $\lambda_1$ et $\lambda_2$ non nulle et de même signe $\Leftrightarrow$ det(A) > 0
 \item[$\rightarrow$] $\lambda_1$ et $\lambda_2$ non nulle et de signe contraire $\Leftrightarrow$ det(A) < 0
\end{itemize}
De plus, si $(\Gamma)$ est une conique dégénéré, les sous espaces propres vectoriels de f fournissent les dimensions vectorielles des axes de la conique.\\
Dans le cas d'un hyperbole, on obtient les droites parallèles au asymptotes en annulant q.
\section{Catalogue des quadriques dans un $\mathbb{R}$ espace affine E (euclidien) de dimension 3}
\begin{de}
On appelle quadrique de (E) un ensemble ($\Sigma$) telque qu'il existe un repère $\mathbb{R}$ de E dans lequel ($\Sigma$) a pour équation P(x,y,z) = 0, avec P un polynome à coefficiant réele du $2^{nd}$ degrès.
\end{de}
\begin{prop}
Dans ce cas, $\forall \mathcal{R}'$ repère de (E), ($\Sigma$) admet également une équation polynomiale de degrès 2 dans $\mathcal{R}'$. 
\end{prop}
\subsection{Catalogue des quadriques}
On suppose E euclidien, $\mathcal{R}$ un repère orthonormée. Pour simplifier l'équation de $(\Sigma)$, on commence par réduire orthonormalement la forme quadratique q, c'est à dire diagonalisé orthonormalement l'endomorphisme symétrique f qui va de $\overrightarrow{E}$ dans $\overrightarrow{E}$ associé à q. On sait que l'on peut eliminer les termes rectangles. On va classer les quadriques possibles suivant, essentiellement, le nombre de valeur propre non nulle. 
\subsubsection{Les trois valeurs propres sont non nulle}
En développent les expressions, on montre que l'on obtient qu'il $\exists \Omega \in E~ /~ Si~ M\underset{\mathcal{R''}}\equiv (X,Y,Z)$, alors : 
$$M \in (\Sigma) \Leftrightarrow \lambda_1.X^2 + \lambda_2.Y^2 + \lambda_3.Z^2 = h'$$
On montre que ceci équivaut à : 
$$\begin{cases}
   \varepsilon_1.\dfrac{X^2}{a^2} + \varepsilon_2.\dfrac{Y^2}{b^2} + \varepsilon_3.\dfrac{Z^2}{c^2} = 1~ si~ h'\neq 0. \\
   \varepsilon_1.\dfrac{X^2}{a^2} + \varepsilon_2.\dfrac{Y^2}{b^2} + \varepsilon_3.\dfrac{Z^2}{c^2} = 0~ si~ h'= 0. \\
  \end{cases}
$$
Avec $\varepsilon_1 = \varepsilon_2 = \varepsilon_3 = \pm 1$. Quitte à permutter les vecteurs $\overrightarrow{u_1},\overrightarrow{u_2},\overrightarrow{u_3}$ de la base B', nous avons les possibilités suivantes :
$$(\varepsilon_1,\varepsilon_2,\varepsilon_3) = (+1,+1,+1)~ (1) $$
$$(\varepsilon_1,\varepsilon_2,\varepsilon_3) = (+1,+1,-1)~ (2) $$
$$(\varepsilon_1,\varepsilon_2,\varepsilon_3) = (+1,-1,-1)~ (3) $$
$$(\varepsilon_1,\varepsilon_2,\varepsilon_3) = (-1,-1,-1)~ (4) $$
Dans tous les cas suivants, on montre que l'on obtient les cas suivantes à partir de cas plus simple, au moyen d'affinité. 
\subparagraph{Cas (1)}
$\Sigma$ est défini par : 
$$\dfrac{X^2}{a^2} + \dfrac{Y^2}{b^2} + \dfrac{Z^2}{c^2} = 1$$
La quadrique se déduit de la sphère unité par au plus 3 affinité droites. On obtient un ellipsoïde (allongé ou aplati). $\Omega$ est un center de symétrie. Les axes de coordonnée dans $\mathcal{R}''$ en sont les axes de symétries.
\subparagraph{Cas (1')}
$(\Sigma)$ est défini par : 
$$\dfrac{X^2}{a^2} + \dfrac{Y^2}{b^2} + \dfrac{Z^2}{c^2} = 0$$
C'est une quadrique dégénéré associé à 1 point :
$$(\Sigma) = \left\lbrace \Omega \right\rbrace $$
\subparagraph{Cas (2)}
$(\Sigma)$ est défini par : 
$$\dfrac{X^2}{a^2} + \dfrac{Y^2}{b^2} - \dfrac{Z^2}{c^2} = 1$$
$(\Sigma)$ est un hyperboloide elliptique à une nappe
\subparagraph{Cas (2')}
$(\Sigma)$ est défini par : 
$$\dfrac{X^2}{a^2} + \dfrac{Y^2}{b^2} - \dfrac{Z^2}{c^2} = 0$$
$(\Sigma)$ est le cône asymptote de l'hyperbolide elliptique précédent.
\subparagraph{Cas (3)}
$(\Sigma)$ est défini par : 
$$\dfrac{X^2}{a^2} - \dfrac{Y^2}{b^2} - \dfrac{Z^2}{c^2} = 1$$
$(\Sigma)$ est un hyperbolide elliptique de révolution à deux nappes
\subparagraph{Cas (3')}
$(\Sigma)$ est défini par : 
$$\dfrac{X^2}{a^2} - \dfrac{Y^2}{b^2} - \dfrac{Z^2}{c^2} = 0$$
$(\Sigma)$ est le cone asymptote du cas précédent.
\subparagraph{Cas (4)}
$(\Sigma)$ est défini par : 
$$-\dfrac{X^2}{a^2} - \dfrac{Y^2}{b^2} - \dfrac{Z^2}{c^2} = 1$$
$(\Sigma)= \emptyset$
\subparagraph{Cas (4')}
$(\Sigma)$ est défini par : 
$$-\dfrac{X^2}{a^2} - \dfrac{Y^2}{b^2} - \dfrac{Z^2}{c^2} = 0$$
$(\Sigma) = \left\lbrace \Omega \right\rbrace $
\subsubsection{Deux valeurs propres distinct, la troisième nulle}
Quitte à réordonnée les vecteurs propres, $\overrightarrow{u_1},\overrightarrow{u_2},\overrightarrow{u_3}$ de la vase B' on peut supposé :
$$\lambda_1 \neq 0,~ \lambda_2 \neq 0,~ \lambda_3=0$$
De la même façon que précédement : 
$$\begin{cases}
   \varepsilon_1.\dfrac{X^2}{a^2} + \varepsilon_2.\dfrac{Y^2}{b^2} = Z ~ si~ \omega_0'\neq 0. \\
   \varepsilon_1.\dfrac{X^2}{a^2} + \varepsilon_2.\dfrac{Y^2}{b^2} = 1~ si~ \omega_0' = 0,~ h'\neq 0. \\
   \varepsilon_1.\dfrac{X^2}{a^2} + \varepsilon_2.\dfrac{Y^2}{b^2} = 0~ si~ \omega_0' = 0,~ h'= 0. \\
  \end{cases}
$$
Quitte à permutter $\overrightarrow{u_1},\overrightarrow{u_2}$, on peut avoir les cas suivants :
$$(\varepsilon_1,\varepsilon_2) = (+1,+1)~ (1)$$
$$(\varepsilon_1,\varepsilon_2) = (-1,-1)~ (2)$$
$$(\varepsilon_1,\varepsilon_2) = (+1,-1)~ (3)$$
\subparagraph{Cas (1)}
$(\Sigma)$ est défini par : 
$$\dfrac{X^2}{a^2} + \dfrac{Y^2}{b^2} = Z$$
$(\Sigma)$ est un paraboloide elliptique
\subparagraph{Cas (2)}
$(\Sigma)$ est défini par : 
$$-\dfrac{X^2}{a^2} - \dfrac{Y^2}{b^2} = Z$$
On peut se ramener au cas précédent en changent de repère.
\subparagraph{Cas (3)}
$(\Sigma)$ est défini par : 
$$\dfrac{X^2}{a^2} - \dfrac{Y^2}{b^2} = Z$$
$(\Sigma)$ est un paraboloide hyperbolique.\\
Dans les cas suivants, l'équation est du type : 
$$f(X,Y)=0$$
On montre que ces surfaces sont des cylindres.
\subparagraph{Cas (4)}
$(\Sigma)$ est défini par : 
$$\dfrac{X^2}{a^2} + \dfrac{Y^2}{b^2} = 0$$
$(\Sigma)$ est l'intersection des plans X=0 et Y=0. C'est à dire :
$$(\Sigma) = (\Omega.Z) = \Omega + Vect(\overrightarrow{u_3})$$
C'est une quadrique dégénéré.
\subparagraph{Cas (5)}
$(\Sigma)$ est défini par : 
$$\dfrac{X^2}{a^2} - \dfrac{Y^2}{b^2} = 0$$
$(\Sigma)$ est la réunion de deux plans parallèle à Oz.
\subparagraph{Cas (6)}
$(\Sigma)$ est défini par : 
$$-\dfrac{X^2}{a^2} - \dfrac{Y^2}{b^2} = 1$$
$(\Sigma) = \emptyset$
\subparagraph{Cas (7)}
$(\Sigma)$ est défini par : 
$$\dfrac{X^2}{a^2} - \dfrac{Y^2}{b^2} = 1$$
$(\Sigma)$ est un cylindre hyperbolique de génératrice parallèle à $\Omega z$.
\subparagraph{Cas (8)}
$(\Sigma)$ est défini par : 
$$\dfrac{X^2}{a^2} + \dfrac{Y^2}{b^2} = 1$$
$(\Sigma)$ est un cylindre hyperbolique de génératrice parallèle à $\Omega z$.
\subsubsection{Une seule valeur propre non nulle}
Quitte à réordonnée les vecteurs propres, $\overrightarrow{u_1},\overrightarrow{u_2},\overrightarrow{u_3}$ de la vase B' on peut supposé :
$$\lambda_1 \neq 0,~ \lambda_2 = 0,~ \lambda_3=0$$
\subparagraph{Cas 1}
Si $(v',w') \neq (0,0)$. Dans un certain repère, on montre que l'équation de $(\Sigma)$ est donnée par : 
$$X^2 - 2.p.Y = 0$$
$(\Sigma)$ est un cylindre parabolique de génératice paralèlle ($\Omega z$)
\subparagraph{Cas 2}
Si $(v',w') = (0,0)$, on obtient une équation du type : 
$$X^2 = h''$$
Nous avons les cas suivants :
\begin{itemize}
 \item[$\rightarrow$] h'' < 0 $\Rightarrow$ ($\Sigma$) = 0
 \item[$\rightarrow$] h'' = 0 $\Rightarrow$ ($\Sigma$) est le plan X=0
 \item[$\rightarrow$] h'' > 0 $\Rightarrow$ ($\Sigma$) est la réunion de deux plans
\end{itemize}

\chapter{Applications linéaires continues, normes subordonnées}
\section{Application linéaires continues}
Soient (E,$\parallel~\parallel$) et (E',$\parallel~\parallel'$) deux K espaces vectoriel (K = $\mathbb{R}$ ou $\mathbb{C}$) et $f \in \mathcal{L}(E,E')$.
\textit{Rappel} : \\
\begin{align*}
\text{f est continue sur E} &\Leftrightarrow\ \text{ f est continue en tout }\overrightarrow{x_0} \in E \\
							&\Leftrightarrow\ \forall \overrightarrow{x_0} \in E~\lim_{\overrightarrow{x}\rightarrow \overrightarrow{x_0}} f(\overrightarrow{x}) = f(\overrightarrow{x}) \\
							&\Leftrightarrow\ \forall \overrightarrow{x_0} \in E, \forall \varepsilon > 0, \exists \alpha > 0~/~ \parallel \overrightarrow{x}-\overrightarrow{x_0} \parallel \leq \alpha \Rightarrow \parallel f(\overrightarrow{x})-f(\overrightarrow{x_0}) \parallel' \leq \varepsilon
\end{align*}
\begin{prop}
Nous avons la propriété suivante : 
$$f \in \mathcal{L}(E,E') \text{ est continue sur E } \Leftrightarrow f \text{ est continue en }\overrightarrow{0}$$
\end{prop}
\begin{prop}
$f\in \mathcal{L}(E,E')$ est continue sur E $\Leftrightarrow$ f est bornée sur $\overline{B}(\overrightarrow{0},1)$ avec :
$$\overline{B}(\overrightarrow{0},1) ) \left\lbrace \overrightarrow{x} \in E~ /~ \parallel \overrightarrow{x} \parallel \leq 1 \right\rbrace $$
\end{prop}
\begin{prop}
Soit $f\in \mathcal{L}(E,E')$, avec E et E' des K espaces vectoriels normés.
$$\text{ f est continue sur E }\Leftrightarrow \text{ f est bornée sur la sphère unité }$$
Avec $S(\overrightarrow{0},1)$ la sphère unité définie par :
$$S(\overrightarrow{0},1) = \left\lbrace \overrightarrow{x} \in E ~/~ \parallel \overrightarrow{x} \parallel = 1 \right\rbrace $$
\end{prop}
\begin{theo}
Si E est un K espace vectoriel normé de dimension finie, et E' un K espace vectoriel normé, alors toute application linéaire $\in \mathcal{E,E'}$ est continue.
\end{theo}
\textbf{Rappel} : \\
La somme de deux applications continues sur E est continue sur E (qu'elles soient ou non linéaire).\\
Le produit par $\lambda \in K$ (K = $\mathbb{R}$ ou $\mathbb{C}$) d'une application continue sur E est continue sur E (toujours que l'application soit linéaire ou non).\\
Si : $f E \rightarrow E'$ et $E' \rightarrow E''$ sont des applications continues alors gof aussi (encore une fois que f et g soient linéaire ou non).\\
Il en résulte, dans le cas des applications linéaires, les propriétés suivantes : 
\begin{prop}
L'ensembe $\mathcal{L}_C(E,E')$ des applications linéaires continues de l'espace vectoriel normé E dans l'espace vectoriel normé E' est un K sous espace vectoriel de $\mathcal{L}(E,E')$ [pour les lois + et $\lambda.$]
\end{prop}
\begin{prop}
$\mathcal{L}_C(E) = \mathcal{L}_C(E,E)$ est une K-algèbre de $\mathcal{L}(E)$ [pour les lois +,$\lambda.$ et o]
\end{prop}
\section{Normes subordonnées}
\subsection{Propriété et définition}
\begin{de}
Soient (E,$\parallel~ \parallel$) et (E',$\parallel~\parallel'$) deux K espace vectoriel normé (K = $\mathbb{R}$ ou $\mathbb{C}$), alors : 
\begin{align*}
\mathcal{L}_C(E,E') &\rightarrow \mathbb{R} \\
f &\mapsto \parallel f\parallel_* = \underset{\overrightarrow{x} \in \overline{B}(\overrightarrow{0},1)}\sup \parallel f(\overrightarrow{x}) \parallel'
\end{align*}
est une norme sur $\mathcal{L}_C(E,E')$ appelé norme subordonnée aux normes $\parallel~\parallel$ sur E et $\parallel~\parallel'$ sur E'.
\end{de}
\textbf{NB} :\\
f étant linéaire contiue, f est bornée sur $\overline{B}(\overrightarrow{0},1)$, donc  $\underset{\overrightarrow{x} \in \overline{B}(\overrightarrow{0},1)}\sup \parallel f(\overrightarrow{x}) \parallel'$ existe dans $\mathbb{R}_+$
\begin{prop}
Si $f \in \mathcal{L}_C(E,E')$ : \\
\begin{align*}
\underset{\overrightarrow{x} \in \overline{B}(\overrightarrow{0},1)}\sup \parallel f(\overrightarrow{x}) \parallel' = \underset{\overrightarrow{x} \in S(\overrightarrow{0},1)}\sup \parallel f(\overrightarrow{x}) \parallel' & = \underset{\overrightarrow{x} \in E - \left\lbrace \overrightarrow{0}\right\rbrace}\sup \dfrac{\parallel f(\overrightarrow{x})\parallel'}{\parallel \overrightarrow{x}\parallel} \\
         &= \underset{\overrightarrow{x} \in \overline{B}(\overrightarrow{0},1), \overrightarrow{x} \neq \overrightarrow{0}}\sup \dfrac{\parallel f(\overrightarrow{x})\parallel'}{\parallel \overrightarrow{x}\parallel}
\end{align*}
\end{prop}
\subsection{Normes Matricielle subordonnée}
Soit $A \in \mathcal{M}_{n,p}(\mathbb{R})$ et f : 
\begin{align*}
f : \mathbb{R}^p &\rightarrow \mathbb{R}^n\\
	X &\mapsto AX \\
\end{align*}
l'application linéaire canoniquement associé à A.\\
Munissons $\mathbb{R}^p$ d'une norme $\parallel~\parallel$ et $\mathbb{R}^n$ d'une norme $\parallel~\parallel'$. On note alors : $\parallel A\parallel_* = \parallel f \parallel_*$ la norme de f subordonnée aux normes $\parallel~\parallel$ et $\parallel~\parallel'$. Donc : 
$$\parallel A \parallel_* = \underset{\overrightarrow{x} \in E - \left\lbrace \overrightarrow{0}\right\rbrace}\sup \dfrac{\parallel AX \parallel'}{\parallel X\parallel} =  \underset{\overrightarrow{x} \in \overline{B}(\overrightarrow{0},1)}\sup \parallel AX \parallel' = \underset{\overrightarrow{x} \in S(\overrightarrow{0},1)}\sup \parallel AX \parallel'$$
\begin{prop}
L'application : 
\begin{align*}
\mathcal{M}_{n,p}(\mathbb{R}) &\rightarrow \mathbb{R} \\
A &\mapsto \parallel A\parallel_*
\end{align*}
est une norme sur $\mathcal{M}_{n,p}(\mathbb{R})$.
\end{prop}
\subsection{Propriété fondamentale de la norme $\parallel~\parallel_*$}
\begin{prop}
Soit $f \in \mathcal{L}_C(E,E')$, avec toujours les mêmes notations pour E et E'. Alors $\forall \overrightarrow{x} \in E$ :
$$\parallel f(\overrightarrow{x})\parallel \leq \parallel f \parallel_* \parallel \overrightarrow{x}\parallel$$
\end{prop}
\textbf{Corollaire} :\\
Si $f \in \mathcal{L}_C(E,E')$, alors f est $\parallel f \parallel_*$ lipchitzienne et donc uniformement continue.
\textbf{Corollaire fondamentale} : \\
La norme $\parallel~\parallel_*$ est sous multiplicative. C'est à dire que si $f \in \mathcal{L}_C(E,E')$ et $g \in \mathcal{L}_C(E,E')$, avec (E,$\parallel~ \parallel$), (E',$\parallel~\parallel'$) et (E'',$\parallel~\parallel''$) des K espaces vectoriels normés alors : 
$$\parallel gof \parallel_* \leq \parallel f\parallel_* \parallel g\parallel_*$$
Nous avons la propriété analogue pour les matrices.
\subsection{Norme d'Algèbre}
\begin{de}
Soit ($\mathcal{A},+,\lambda.,\times$) une K-algèbre, avec $K = \mathbb{R}$ ou $\mathbb{C}$.\\
Une norme sur le K espace vectoriel ($\mathcal{A},+,\lambda.$) est appelé norme d'Algèbre si elle est sous multiplicative, c'est à dire si :
$$\forall(x,y)\in \mathcal{A}~\parallel x \times y \parallel \leq \parallel x\parallel \parallel y \parallel$$
($\mathcal{A},+,\lambda.,\times,\parallel~\parallel$) est appelé alors une Algèbre normé.
\end{de}
\subsubsection{Norme subordonnée à $\parallel~\parallel_1$}
Considérons le cas ou $K=\mathbb{R}$. Pour obtenir le résultat suivant, comme dans tout les cas suivant, on cherche à majorer la norme $\parallel~\parallel_*$ subordonnée à la norme considéré, puis à montrer un X particulier qui permet d'obtenir l'égalité. Si $\mathbb{R}^n$ est munie de $\parallel~\parallel = \parallel~\parallel_1$, alors : 
$$\parallel A\parallel_* = \underset{1 \leq j \leq n}\max \parallel C_j \parallel_1$$
Dans ce cas, le X particulier à considérer est : Si $\underset{1 \leq j \leq n}\max \parallel C_j \parallel_1 = \parallel C_{j0} \parallel$, alors X = $C_{j0}$.\\
On obtient un résultat équivalent dans le cas où $K=\mathbb{C}$. 
\subsubsection{Norme subordonnée à $\parallel~\parallel_{\infty}$}
Considérons le cas ou $K = \mathbb{R}$. Dans ce cas, on obtient que : 
$$\parallel A\parallel_* = \underset{1 \leq i \leq n}\max\parallel L_i \parallel_1$$
Dans ce cas, le X particulier à considérer est : Si $\underset{1 \leq i \leq n}\max\parallel L_i \parallel_1 = \parallel L_{i0} \parallel_1 = \parallel a_{i0,1} \dots a_{i0,n}$, alors on prend :
$$X = \begin{pmatrix}
       signe(a_{i0,1}) \\
	\vdots \\
	signe(a_{i0,n}) \\
      \end{pmatrix}
$$
\subsubsection{Norme subordonnée à $\parallel~\parallel_2$, la norme euclidienne canonique sur $\mathbb{R}^n$}
Au cours de la démonstration, nous avons énoncé les définitions et propritétés suivantes : 
\begin{de}
Un endomorphisme f d'un $\mathbb{R}$ espace vectoriel euclidien E est dit positif (repectivement défini positif ) si la forme quadratique qui lui est associé est positive (respectivement définie positive)
\end{de} 
\begin{prop}
Si f est un endomorphisme d'un $\mathbb{R}$ espace vectoriel euclidien E :
\begin{align*}
 \text{ f positif } &\Leftrightarrow S_p(f) \subset \mathbb{R}_+ \\
 \text{ f défini positif } &\Leftrightarrow S_p(f) \subset \mathbb{R}_+^*
\end{align*}
On obtient en faisant comme d'habitude : Si $\parallel~\parallel = \parallel~\parallel_2$, la norme euclidienne canonique de $\mathbb{R}^n$, alors $\forall A \in \mathcal{M}_n(\mathbb{R})$ : 
$$\parallel A\parallel_* = \sqrt{\rho(^tA A)}$$
Avec $\rho(^tAA)$ le rayon spectral de $^tAA$.
\end{prop}
\subsection{Suite d'endomorphisme en dimension finie}
\begin{prop}
Soit $(f_k)_{k \in \mathbb{N}}$ une suite d'endomorphisme d'un K espace vectoriel normé (E,$\parallel~\parallel$) de dimension finie et $f \in \mathcal{L}(E)$. Alors les conditions suivantes sont équivalentes : 
\begin{itemize}
 \item[$\rightarrow$] $f_k \rightarrow f$ quand $k \rightarrow \infty$ (de plus, les espaces sont de dimensions finies, donc la convergence ne dépend pas de la norme considéré)
 \item[$\rightarrow$] $f_k$ converge uniformement vers f sur toute parties bornées.
 \item[$\rightarrow$] $f_k$ converge uniformement vers f sur tout compact.
 \item[$\rightarrow$] $\forall \overrightarrow{x} \in E$, $f_k(\overrightarrow{x}) \underset{k \rightarrow +\infty}\rightarrow f(\overrightarrow{x})$ ($|f_k|$ converge simplement vers f sur E)
 \item[$\rightarrow$] $\forall B$ base de E et $\forall \overrightarrow{e} \in B$ $f_k(\overrightarrow{e}) \underset{k\rightarrow \infty}\rightarrow f(\overrightarrow{e})$
 \item[$\rightarrow$] $\exists B$ base de E telque $\forall \overrightarrow{e} \in B$ $f_k(\overrightarrow{e}) \underset{k\rightarrow \infty}\rightarrow f(\overrightarrow{e})$
 \item[$\rightarrow$] $\forall B$ base de E, $mat_B(f_k) \underset{k\rightarrow \infty}\rightarrow mat_B(f)$
 \item[$\rightarrow$] $\exists B$ base de E telque $mat_B(f_k) \underset{k\rightarrow \infty}\rightarrow mat_B(f)$
\end{itemize}
\end{prop}

\appendix                     % Les annexes
\part{Annexe}
\chapter{Intégrales généralisées}
\section{Convergence d'une intégrale}
\subsection{Propriétés}
Il existe trois propriétés qui permettent de prouver la convergence, d'une intégrale complexe, à l'aide d'une intégrale "simple".
\begin{itemize}
 \item[$\rightarrow$] La convergence d'une intégrale de fonction positive par majoration (Implication)
 \item[$\rightarrow$] L'intégration par domination (Implication)
 \item[$\rightarrow$] La convergence des intégrale de fonction positive par equivalence (Équivalence)
\end{itemize}
\subsection{Fonctions "classique"}
Il existe plusieurs "cas" standard, au quel on peut se rapporter pour démontrer la convergence d'une intégrale
\subsubsection{En $\infty$}
Nous retiendrons que dans l'étude en $+\infty$, les $\alpha$ "tendent plus vers l'infini" (utilisation du > dans les relations). \\
Nous avons la règle de Riemann :
\begin{prop}
Soit f fonction continue par morceaux de $\left[a,\infty\right[$ dans K.\\
Si il existe $\alpha > 1$ telque : 
$$t^{\alpha}f(t) \underset{t \rightarrow \infty}\rightarrow 0$$
Alors f est intégrable ( converge absolument) sur $\left[a,\infty\right[$
\end{prop}
Nous avons aussi l'intégrale de Bertrand : 
\begin{prop}
Soit a un réel strictement superieur à 1, et $(\alpha,\beta) \in \Re^2$.
$$\left(\int_a^{\infty} \dfrac{dx}{x^{\alpha}ln(x)^{\beta}} \mbox{ converge }\right) \Leftrightarrow ( \alpha > 1,~ ou~ \alpha=1,\beta>1)$$
\end{prop}
La propriété suivante n'est qu'un cas particulier de l'intégrale de Bertrand, ou $\beta = 0$ :
\begin{prop}
Soit $a \in \Re$, $a>0$ :
$$\left(\int_a^{\infty} \dfrac{dt}{t^{\alpha}}\right) \mbox{ converge } \Leftrightarrow (\alpha > 1)$$
\end{prop}
\subsubsection{En 0}
Nous retiendrons que l'ordre de l'étude en 0, les $\alpha$ "tend en quelque sorte plus vers 0" (utilisation du < dans les relations ).\\
Nous avons la règle de Riemann :
\begin{prop}
Soit f fonction continue par morceaux de $\left]0,a\right]$ dans K.\\
Si il existe $\alpha < 1$ telque : 
$$t^{\alpha}f(t) \underset{t \rightarrow 0^+}\rightarrow 0$$
Alors f est intégrable ( converge absolument) sur $\left]0,a\right]$
\end{prop}
Dans le cas de l'intégralde de Bertrand, nous retiendrons que le second cas, $\alpha =1,~ \beta > 1$, est identique en l'infini et en 0.\\
Nous avons aussi l'intégrale de Bertrand : 
\begin{prop}
Soit a un réel telque $a \in \left]0,1\right[$, et $(\alpha,\beta) \in \Re^2$.
$$\left(\int_0^a \dfrac{dx}{x^{\alpha}ln(x)^{\beta}} \mbox{ converge }\right) \Leftrightarrow ( \alpha < 1 ,~ ou~ \alpha=1,\beta>1)$$
\end{prop}
La propriété suivante n'est qu'un cas particulier de l'intégrale de Bertrand, ou $\beta = 0$ :
\begin{prop}
Soit $a \in \Re$, $a>0$ :
$$\left(\int_0^a \dfrac{dt}{t^{\alpha}}\right) \mbox{ converge } \Leftrightarrow (\alpha < 1)$$
\end{prop}
\section{Divergence d'une intégrale}
\subsection{Les règles de Riemann}
Les règles de Riemann nous donne les moyens de prouver la divergence d'une intégrale.
\subsubsection{En $\infty$}
Soit f fonction de $\left[a,\infty\right[$ dans $\Re$, continue par morceaux. Si :
$$t.f(t) \underset{t\rightarrow\infty}\rightarrow 0$$
Alors :
$$\int_a^{\infty} \mbox{ diverge }$$
\subsubsection{En 0}
Soit f fonction de $\left]0,a\right]$ dans $\Re$, continue par morceaux. Si :
$$t.f(t) \underset{t\rightarrow 0}\rightarrow 0$$
Alors :
$$\int_0^a \mbox{ diverge }$$
\chapter{Développement asymptotiques}
Il existe une méthodologie "classique" à utiliser dans le cas d'un développement asymptotique.
\section{Fonction du type $f^{\alpha}$, ou ln(f)}
Pour obtenir le développement asymtotique de fonction du type $f^{\alpha}$, ou ln(f), on range les termes de façon prépondérant décroissante, puis on met toujours le terme prépondérant en facteur, et enfin on effectue un développement limité avec le reste, qui tend vers 0.\\
\underline{Exemple : }
Considérons le fonction ln($x^2+x+1$). Cette fonction est rangé en considérant la prépondérance en l'infini. On obtient donc : 
$$ln(x^2+x+1) = ln((x^2)(1+\dfrac{1}{x}+\dfrac{1}{x^2})$$
$$ln(x^2+x+1) = ln(x^2) + ln(1+\dfrac{1}{x}+\dfrac{1}{x^2})$$
On effectue un développement limité de la forme ln(1+u), avec u qui tend vers 0.
\section{Développement asymptotique de $S_n$ ou de $R_n$ dans le cas d'une série}
La première question à se poser est de savoir si la série $\underset{n}\sum u_n$ converge.
\subsection{Si la série converge}
Si la série converge, alors :
$$\lim_{n\rightarrow \infty} S_n = S$$
On poursuit en écrivant l'égalité suivante : 
$$S_n = S - R_n$$
Pour continuer le développement asymptotique, il faut donc détérminer un équivalent à $R_n$
\subsection{Si la série diverge}
Alors on cherche directement un équivalent de $S_n$
\subsection{Méthode à suivre}
Pour obtenir un équivalent de $S_n$, dans le cas divergent, ou un équivalent de $R_n$ dans le cas convergent, on simplifie le problème en remplacent $u_k$ par un équivalent $w_k$ plus simple.\\
On obtient alors : 
$$S_n \underset{\infty}\sim \sum_{k=n_0}^n w_k \mbox{ cas non sommable }$$
$$R_n \underset{\infty}\sim \sum_{k=n+1}^{\infty} w_k \mbox{ cas sommable }$$
L'utilisation de ceci ramène le problème de recherche d'équivalent de la somme partielle ou du reste de la série $\underset{k}\sum w_k$, avec $(w_k)$ appartenant à une échelle de comparaison. : 
$$w_k = k^{\alpha}.ln(k)^{\beta}.e^{P(k)}$$
Avec P(k) un pseudo polynome.\\
Pour obtenir un équivalent, on distingue deux cas : 
\begin{itemize}
 \item[$\rightarrow$] Si $(w_k)$ est à variation lente ($P = 0$ ou deg(P)<1), alors on encadre par des intégrales
 \item[$\rightarrow$] Si $(w_k)$ est à variation rapide (deg(P) $\geq$ 1), on utilise une comparaisons asymptotique entre $w_{k+1}-w_k$ et $w_k$.
\end{itemize}
Les relations de comparaions utilisable dans le second cas sont : 
$$\gg;\ll;\sim$$
\chapter{Séries}
\section{Propriétés générales}
Pour montrer la convergence d'une série $\underset{n} \sum u_n$, il faut déjà vérifier que $(u_n) \underset{\infty}\rightarrow 0$. Ceci est une condition necessaire, mais non suffisante.\\
Nous avons trois propriétés générales qui implique la convergence à l'aide d'un terme générale plus simple :
\begin{itemize}
 \item[$\rightarrow$] La convergence d'une série de terme général positive par majoration (Implication)
 \item[$\rightarrow$] L'intégration par domination (Implication)
 \item[$\rightarrow$] La convergence d'une serie de terme générale par équivalence (Équivalence)
\end{itemize}
\section{Règle usuelle}
\subsection{Convergence des séries de Riemann}
Soit une séries de Riemman, de terme général :
$$u_n  = \dfrac{1}{n^{\alpha}}$$
Cette série converge si et seulement si :
$$(\alpha > 1)$$
\subsection{Règle de Riemann - Convergence}
Soit $(u_n)$ une suite à valeur complexe.\\
Si il existe $\beta > 1$ telque :
$$(n^{\beta}.u_n \underset{\infty}\rightarrow 0$$
Alors $(u_n)$ est sommable
\subsection{Règle de Riemann - Divergence}
Soit $(u_n)$ une suite à valeur réelle.\\
Si : 
$$n.u_n \underset{\infty}\rightarrow \infty $$
Alors la série de terme générale $u_n$ diverge.
\subsection{Série de Bertrand}
Soit une série de Bertrand, de terme général :
$$u_n = \dfrac{1}{n^{\alpha}.ln(n)^{\beta}}$$
Cette série converge si et seulement si : 
$$(\alpha > 1~ ou~ \alpha=1~ et~ \beta>1)$$
\subsection{Règle de d'Alembert}
Soit $(u_n)$ une suite de réels telques $\forall n \geq n_0$, $u_n>0$ et telque : 
$$\dfrac{u_{n+1}}{u_n} \underset{\infty}\rightarrow l$$
Alors :
\begin{itemize}
 \item{$\rightarrow$} Si l > 1, alors la série de terme général $u_n$ diverge grossièrement
 \item{$\rightarrow$} Si l < 1, alors la série converge
\end{itemize}
\chapter{Calcul matriciel par blocs}
\section{Produit matriciel par blocs}
Considérons deux matrices A et A' : 
\[A = \begin{blockarray}{cccc}
        & \overset{\beta_1}\leftrightarrow & \dots & \overset{\beta_p}\leftrightarrow  \\
        \begin{block}{c(ccc)}
        \alpha_1 \updownarrow & \mesbloc{A_{11}}  & \dots & \mesbloc{A_{1p}}  \\
        \vdots  & \vdots & \ddots & \vdots      \\
        \alpha_n \updownarrow   & \mesbloc{A_{n1}} & \dots  & \mesbloc{A_{np}}   \\
        \end{block}
        \end{blockarray}
\] 
\[A' = \begin{blockarray}{cccc}
        & \overset{\gamma_1}\leftrightarrow & \dots & \overset{\gamma_q}\leftrightarrow  \\
        \begin{block}{c(ccc)}
        \beta_1 \updownarrow & \mesbloc{A'_{11}}  & \dots & \mesbloc{A'_{1q}}  \\
        \vdots  & \vdots & \ddots & \vdots      \\
        \beta_p \updownarrow   & \mesbloc{A'_{p1}} & \dots  & \mesbloc{A'_{pq}}   \\
        \end{block}
        \end{blockarray}
\] 
Avec :
$$\begin{cases}
A_{ij} \in  \mathcal{M}_{\alpha_i,\beta_j}(K) \\  A'_{kl} \in  \mathcal{M}_{\beta_k,\gamma_l}(K)
\end{cases}$$
On défini de plus : 
$$\begin{cases}
\alpha = \alpha_1 + \dots + \alpha_n \\ \beta = \beta_1 + \dots + \beta_p \\ \gamma = \gamma_1 + \dots + \gamma_q
\end{cases}$$
On obtient, pour le produit de A' par A :
\[A.A' = \begin{blockarray}{ccc}
        \begin{block}{(ccc)}
         \mesbloc{C_{11}}  & \dots & \mesbloc{C_{1q}}  \\
         \vdots & \ddots & \vdots      \\
         \mesbloc{C_{n1}} & \dots  & \mesbloc{C_{nq}}   \\
        \end{block}
        \end{blockarray}
\] 
Avec : 
$$C_{ij} = \sum_{k=1}^p A_{ik}.A'_{kj}$$
L'ordre du produit réalisé dans la somme a une importance à priori, car le produit des matrices n'est pas commuatif, à priori.
\subsection{Cas particuliers}
\subsubsection{Matrices diagonales}
Soient A et A' deux matrices diagonales : 

\[A = \begin{blockarray}{cccc}
         \overset{\beta_1}\leftrightarrow & \dots & \overset{\beta_n}\leftrightarrow  \\
        \begin{block}{(ccc)c}
         \mesbloc{A_{11}}  &  & (0)  & ~~~~~ \\
                            &\ddots  & & ~~~~~     \\
         (0)                &  & \mesbloc{A_{nn}}  & ~~~~~\\
        \end{block}
        \end{blockarray}
\]

\[A' = \begin{blockarray}{cccc}
        \begin{block}{(ccc)c}
         \mesbloc{A'_{11}}  &  & (0) & \updownarrow \beta_1   \\
                            &\ddots  &      &  \\
         (0)                &  & \mesbloc{A'_{nn}} & \updownarrow\beta_n   \\
        \end{block}
        \end{blockarray}
\] 
On obtient, pour le produit : 
\[A.A' = \begin{blockarray}{ccc}
        \begin{block}{(ccc)}
         \mesbloc{A_{11}.A'_{11}}  &  & (0)    \\
                            &\ddots  &        \\
         (0)                &  & \mesbloc{A_{nn}.A'_{nn}}   \\
        \end{block}
        \end{blockarray}
\] 
\subsubsection{Matrices Triangulaires}
Soient A et A' deux matrices triangulaire : 
\[A = \begin{blockarray}{cccc}
         \overset{\beta_1}\leftrightarrow & \dots & \overset{\beta_n}\leftrightarrow  \\
        \begin{block}{(ccc)c}
         \mesbloc{A_{11}}  &  & (A)  & ~~~~~ \\
                            &\ddots  & & ~~~~~     \\
         (0)                &  & \mesbloc{A_{nn}}  & ~~~~~\\
        \end{block}
        \end{blockarray}
\]

\[A' = \begin{blockarray}{cccc}
        \begin{block}{(ccc)c}
         \mesbloc{A'_{11}}  &  & (A') & \updownarrow \beta_1   \\
                            &\ddots  &      &  \\
         (0)                &  & \mesbloc{A'_{nn}} & \updownarrow\beta_n   \\
        \end{block}
        \end{blockarray}
\] 
On obtient, pour le produit : 
\[A.A' = \begin{blockarray}{ccc}
        \begin{block}{(ccc)}
         \mesbloc{A_{11}.A'_{11}}  &  & (B)    \\
                            &\ddots  &        \\
         (0)                &  & \mesbloc{A_{nn}.A'_{nn}}   \\
        \end{block}
        \end{blockarray}
\] 
\section{Calcul de déterminants de matrices triangulaires par blocs}
\subsection{Propriétés}
\begin{prop}
Nous avons la propriété suivantes : 
\[det~ \begin{blockarray}{cc}
        \overset{n_1}\leftrightarrow & \overset{n_2}\leftrightarrow \\
	\begin{block}{(cc)}
         \mesbloc{A_{11}}  &  \mesbloc{A_{12}}    \\
         0   		   & \mesbloc{A_{22}}   \\
        \end{block}
        \end{blockarray} = det(A_{11}).det(A_{22})
\] 
Avec :
$$\begin{cases}
A_{11} \in  \mathcal{M}_{n_1}(K) \\  A_{22} \in  \mathcal{M}_{n_2}(K)
\end{cases}$$
\end{prop}
\subsection{Généralisation}
\begin{prop}
On peut généraliser la propriété de la façon suivante : 
\[det ~ \begin{blockarray}{ccc}
	\overset{n_1}\leftrightarrow & \dots & \overset{n_p}\leftrightarrow \\
        \begin{block}{(ccc)}
         \mesbloc{A_{11}}  &  & (B)    \\
                          &\ddots  &        \\
         (0)                &  & \mesbloc{A_{pp}}   \\
        \end{block}
        \end{blockarray} = det(A_{11})\dots det(A_{pp})
\] 
\end{prop}
\chapter{Points importants pour obtenir l'équation réduit d'un quadrique $(\Sigma)$ et l'identifier}
Nous avons les points suivants :
\begin{itemize}
 \item[$\rightarrow$] Réduire orthonormalement la forme quadratique q associé à $(\Sigma)$. Dans certains cas, pour identifier ($\Sigma$), la connaisance des valeurs propres suffit.
 \item[$\rightarrow$] Si l'équation réduite f(X,Y,Z) = 0 est homogène, donc de degrés 2, $(\Sigma)$ est une surface conique de sommet $\Omega$.
 \item[$\rightarrow$] Si l'équation réduite ne contient pas de Z (par exemple) du type f(X,Y)=0, ($\Sigma$) est une surface cylindrique de génératrice parallèle à $\Omega z$.
 \item[$\rightarrow$] S'il s'agit d'identifier ($\Sigma$), on peut encore simplifier l'équation réduite en transformant ($\Sigma$) par des affinités droites de base l'un des plans de coordonnées.
\end{itemize}


\backmatter
\tableofcontents            % Table des matières
\end{document}
